\documentclass{article}
\usepackage[utf8]{inputenc}
\usepackage[ngerman]{babel}
\usepackage[left=3.5cm,right=3.5cm]{geometry}
\usepackage{csquotes}
\usepackage{enumitem}
\usepackage{amsmath}
\usepackage{amssymb}

\begin{document}

\section*{Anleitung für Autoren}

In diesem Dokument werden stilistische und didaktische Grundsätze, die grundsätzlich im gesamten Buch eingehalten werden sollen, festgelegt. Hingegen werden hier \textbf{keine} Inhalte vorgegeben. Eine inhaltliche Gliederung erfolgt separat.

Um ein einheitliches Erscheinungsbild des Buches zu gewährleisten, sollen alle Kapitel analog zum Kapitel \enquote{Abbildungen} aufgebaut sein, das als Referenzkapitel geschrieben wurde. Folgendes ist zu beachten:

\begin{itemize}[leftmargin=.5cm]
	\item Es wird der Ansatz verfolgt, immer zunächst ein intuitives Verständnis (wenn möglich anhand von Beispielen aus dem Alltag, die die Schüler bereits durchblickt haben) zu erzeugen, bevor formale Schreibweisen und Definitionen eingeführt werden. Dadurch soll, falls möglich, erreicht werden, dass eine Metapher immer implizit mitschwingt, wenn der Schüler über das Thema nachdenkt (solange, bis er daraus ein intuitiv mathematisches Verständnis gewinnt)
	\item Neue Konzepte sollten immer zunächst an einem Beispiel gezeigt werden, bevor sie vertiefend erklärt werden.
	\item Jedes Kapitel beginnt mit einem Abschnitt \enquote{Einleitung}, der Zusammenhänge zum Alltag herstellt und/oder intuitiv und an Beispielen in das Thema einführt, ohne formale Schreibweisen und Definitionen einzuführen. Es können jedoch formale Schreibweisen aus vorausgesetzten Kapiteln kurz wiederholt werden (gerade, wenn eine bestimmte Definition für das Kapitel besonders zentral ist -- etwa Mengen für das Kapitel über Abbildungen).
	\item Der zweite Abschnitt \enquote{Formale Schreibweisen} vertieft die Einleitung, indem dort die im Einleitungsabschnitt intuitiv erklärten Konzepte formal definiert und notiert werden (auf diese Formalien soll Schritt für Schritt mit Beispielen hingearbeitet werden)
	\item Alle weiteren Themen erhalten keinen separaten Abschnitt für eine Einleitung, sollen aber trotzdem im gleichen Stil eingeführt werden
	\item Am Ende jedes Kapitels steht ein Abschnitt \enquote{Technische Beweise} (außer, er würde komplett leer bleiben), in den alles ausgelagert wird, was für die mathematische Vollständigkeit relevant ist, den Lesefluss für die meisten Schüler aber stört (etwa weil der Beweis zu technisch, komplex oder mit regulären Schulmitteln nicht durchführbar ist oder keine Ideen enthält, die für das Verständnis besonders hilfreich sind). 
	\item Beweise innerhalb eines Zusatzwissen-Blocks gehören immer in diesen Abschnitt, damit die Zusatzwissen-Blöcke nicht zu lang werden und das Kapitel nicht unübersichtlich machen.
	\item Schematische Bilder, die allgemein und nicht nur für ein bestimmtes Beispiel ihre Gültigkeit haben, gehören in den normalen Fließtext (mit ausführlicher Bildunterschrift).
	\item Beispiele können 
		\begin{enumerate}
			\item ohne Nennung mathematischer Details einen Zusammenhang aus dem Alltag beschreiben, der sich ähnlich wie die zu erklärende Mathematik verhält
			\item anhand eines Alltagsbeispiels eine formale Schreibweise erläutern
			\item ein Beispielbild liefern, mit dem gearbeitet wird
			\item ein Rechenbeispiel sein
		\end{enumerate}
	\item Es sollte versucht werden, zu so vielen Beispielen wie möglich ein Bild (bestenfalls schematische Darstellung eines Zusammenhangs, z. B. Koordinatensystem, Mengendiagramm) zu finden, damit die Beispiele besser im Kopf bleiben. Findet sich ein solches Bild nicht, kann ein thematisch passendes Bild (etwa Eisenbahn bei einem Eisenbahn-Beispiel) verwendet werden. Solche Bilder sollen einen transparenten Hintergrund haben und keine Fotos sein. Als Bildquelle ist auf Bildrechte zu achten (z. B. Pixabay-Bilder verwenden oder Bilder mit tikz erstellen).
	\item Jeder Abschnitt (außer technische Beweise und Einleitung) wird durch einen Zusammenfassungsblock beendet. Dort sollen alle Inhalte des Kapitels inkl. formaler Schreibweisen zusammengefasst werden, sodass es theoretisch möglich wäre, statt des Abschnitts nur die Zusammenfassung zu lesen (d.h. die Zusammenfassung darf das Wissen aus dem Abschnitt nicht voraussetzen, sondern soll es in sehr hohem Tempo möglichst kompakt und ohne Details darstellen).
	\item Es soll vermieden werden, dass Definitionen und Resultate \enquote{vom Himmel fallen}. Deshalb soll, falls mit dem Schulstoff ohne großen technischen Aufwand möglich, eine formale Herleitung immer mit angegeben werden. Sollte der Eindruck entstehen, dass diese zu komplex und für die meisten Schüler nicht nachvollziehbar ist, dann soll sie innerhalb eines Zusatzwissen-Blocks angegeben werden.
	\item Alle Sätze sollen bewiesen oder hergeleitet werden (siehe letzter Punkt). Ist eine normale Herleitung schwierig und ein eher formaler Beweis nötig, dann gehört dieser in den Abschnitt \enquote{Technische Beweise} am Ende des Kapitels
	\item Wissen, das über den normalen Schulstoff hinausgeht, aber als Ergänzung für motivierte Schüler aufgrund seines starken Zusammenhangs zum Thema sinnvoll ist, soll an möglichst passender Stelle als Zusatzwissen präsentiert werden.
	\item Tabellen werden mithilfe von \emph{booktabs} dargestellt.
	\item Es wird ein Stichwortverzeichnis geführt. Begriffe, die im Kapitel, das du schreibst, eingeführt werden, sollen mit 
	\[\backslash index\{\langle\: \text{Name, der im Index stehen soll}\:\rangle\}\] 
	in den Index aufgenommen werden, wenn der Begriff dort vorkommt. Dabei soll immer die Definition (falls existent) und die Zusammenfassung in den Index aufgenommen werden. Wird ein Begriff im Fließtext eingeführt und in der Zusammenfassung nicht erwähnt, dann wird die Erwähnung im Fließtext zitiert. Ansonsten wird ausschließlich die Zusammenfassungsbox zitiert.
	
	Spielt der Begriff an späterer Stelle eine zentrale Rolle, obwohl er schon früher eingeführt wurde, dann kann diese spätere Stelle zusätzlich in den Index aufgenommen werden.
\end{itemize}

\section*{Arbeitsfluss}

Der normale Arbeitsfluss beim Schreiben von Kapiteln soll folgendermaßen aussehen (optimalerweise sollten nach den meisten Schritten die Ergebnisse besprochen werden, um verschiedene Ansätze berücksichtigen zu können und dann den besten zu wählen):

\begin{itemize}[leftmargin=.5cm]
	\item Sammeln, welche Themen (z.B. welche Definitionen, Sätze) im Rahmen des Schulwissens im Kapitel benötigt werden
	\item Definieren, mit welchen ergänzenden Themen die Darstellung vervollständigt werden kann, um das Schulwissen vernetzter zu vermitteln (diese Themen fallen dann noch nicht in die Kategorie Zusatzwissen)
	\item Definieren, welche vertiefenden Themen in Form von Zusatzwissen präsentiert werden sollen
	\item Metaphern aus dem Alltag (oder graphische Anschauungen) für alle Themen finden
	\item Definieren, welches Vorwissen für das Kapitel besonders zentral ist und im Umfang von ca. $\frac{1}{2}$ Seite zu Beginn der Einleitung daran erinnern
	\item Gliederung erstellen, die alle Themen in sinnvoll aufeinander aufbauender Reihenfolge abbildet
	\item Bevor ein Abschnitt verfasst wird, sollte klar sein, wo Beispiele eingesetzt werden und welche Konzepte sie illustrieren sollen. Es sollte außerdem für jedes Beispiel geklärt sein, ob es ein Rechenbeispiel oder ein Metapher-Beispiel sein soll (am besten bieten sich Rechenbeispiele nach der Definition an, Metaphern vorher). Siehe dazu Abschnitt 1.1 und 1.2 aus dem Abbildungs-Kapitel.
\end{itemize}

\section*{Generelle Dos and Dont's}
\begin{itemize}[leftmargin=.5cm]
    \item 
\end{itemize}

\section*{\LaTeX{} Dos and Don'ts}
\begin{itemize}[leftmargin=.5cm]
    \item Für \enquote{ist definiert als} ist der Befehl \texttt{\textbackslash coloneqq} anstelle von \texttt{:=} zu verwenden.
    \item In Anführungszeichen setzen mit \texttt{\textbackslash enquote\{text\}}.
    \item Stückweise definierte Funktionen mit der \texttt{cases}-Environment setzen.
    \item Für 2D-Koordinaten bitte \texttt{\textbackslash coord\{x\}\{y\}} verwenden.
    \item Für Vektoren bitte \texttt{\textbackslash vec\{v\}} verwenden.
    \item Binäre Operatoren (wie z. B. \texttt{\textbackslash xor}), die \LaTeX{} noch nicht kennt, mit Hilfe von \texttt{\textbackslash mathbin} definieren. Allgemein nützliche Operationen in den entsprechenden Abschnitt von \texttt{mathdef.sty} schreiben.
    \item Funktionen und Operationen (wie z. B. \texttt{\textbackslash ggT}), die \LaTeX{} noch nicht kennt, mit Hilfe von \texttt{\textbackslash DeclareMathOperator} definieren. Allgemein nützliche Funktionen und Operationen in den entsprechenden Abschnitt von \texttt{mathdef.sty} schreiben.
    \item Umrahmende Symbole (wie z. B. \texttt{\textbackslash abs}), die \LaTeX{} noch nicht kennt, mit Hilfe von \texttt{\textbackslash DeclarePairedDelimiter} definieren. Allgemein nützliche Notationen in den entsprechenden Abschnitt von \texttt{mathdef.sty} schreiben.
\end{itemize}

\end{document}