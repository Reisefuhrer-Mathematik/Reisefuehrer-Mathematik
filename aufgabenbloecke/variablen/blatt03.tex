\documentclass[../variablen.tex]{subfiles}

\begin{document}


\begin{exercise}{difficult}
    Im Folgenden sei $k$ eine ganze Zahl. Multipliziere die folgenden Terme aus:
    \begin{enumerate}
        \item $(x-1)(x^k+x^{k-1}+\cdots+x^2+x+1)$
        \item $(x+1)(x^{2k}-b^{2k-1}+b^{2k-2}-\cdots+b^2-b+1)$
        \item $(x^2-1)(x^{2k}+x^{2k-2}+x^{2k-4}+\cdots+x^2+1)$
    \end{enumerate}
\end{exercise}

\begin{exercise}{difficult}
    Zeige, dass der folgende Kopfrechentrick funktioniert.

    Um eine zweistellige Zahl $a$ mit 11 zu multiplizieren, berechne zunächst ihre Quersumme. Schreibe die Quersumme zwischen die beiden Ziffern von $a$.
\end{exercise}

\end{document}