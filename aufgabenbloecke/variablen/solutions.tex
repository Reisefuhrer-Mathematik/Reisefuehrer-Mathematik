\documentclass[../abbildungen.tex]{subfiles}

\begin{document}
    \colorlet{highlight}{violet}
    %\colorlet{highlight}{black!90}
    \textcolor{highlight}{\textbf{Beispielaufgabe 1 -- Formale Beschreibung einer Abbildung}}
    
    Ein Haustierzüchter züchtet Katzen, Kaninchen, Hamster und Wellensittiche. In der letzten Woche hat er je ein Tier jeder dieser Tierarten verkauft: Einen Wellensittich namens Martin, ein Kaninchen namens Andy und zwei Tiere namens Traxx und Naja. Es soll nun eine Abbildung \textsc{Tierart} definiert werden, die zu den Namen dieser vier verkauften Tiere die jeweilige Tierart angibt.
    
    \begin{enumerate}[label=\textcolor{highlight}{\alph*)}]
        \item Welche Definitions- und Bildmenge hat die Abbildung \textsc{Tierart}?
        \item Was bedeutet die Schreibweise $\textsc{Tierart}(\text{Naja})=\text{Katze}$?
        \item Beschreibe Abbildung und Abbildungsvorschrift formal.
        \item Beschreibe die Abbildung als Diagramm.
    \end{enumerate}
    
    \textcolor{highlight}{\emph{Lösung.}} 
    \begin{enumerate}[label=\textcolor{highlight}{\alph*)}]
        \item O.B.d.A. trivial.
    \end{enumerate}
    
    %\textcolor{highlight}{\textbf{Beispielaufgabe 2 -- Wertetabellen für Abbildungen erstellen}}
    
    %\textcolor{highlight}{\textbf{Beispielaufgabe 3 -- Ablesen von Abbildungsvorschriften aus Wertetabellen}}
    
    %\textcolor{highlight}{\textbf{Beispielaufgabe 4 -- Aufstellen von Berechnungsvorschriften}}
    
    %\textcolor{highlight}{\textbf{Beispielaufgabe 5 -- Informationen von einem Abbildungsgraphen ablesen}}
    
    %\textcolor{highlight}{\textbf{Beispielaufgabe 6 -- Zeichnen eines Abbildungsgraphen}}
\end{document}