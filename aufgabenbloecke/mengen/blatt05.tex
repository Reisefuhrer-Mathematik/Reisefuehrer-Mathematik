\documentclass[../mengen.tex]{subfiles}

\begin{document}

\begin{exercise}{difficult}
    Es seien $a,b,c$ drei irrationale Zahlen. Beweise, dass es unter diesen Zahlen zwei Zahlen gibt, deren Summe auch irrational ist.
    %https://www.math.kit.edu/ianmip/lehre/methloesung2005s/media/3.pdf
\end{exercise}

\begin{exercise}{difficult}
    Wir stellen uns ein Hotel vor, dass unendlich viele Zimmer hat und damit unendlich vielen Gästen Platz bieten kann. Damit alle Gäste ihr Zimmer finden, sind alle Zimmer mit einer Zimmernummer versehen. Leider ist das Hotel gerade voll, es sind also alle Zimmer belegt.
    \begin{enumerate}
        \item Es kommt ein neuer Gast an die Rezeption. Wie kann ihm ein Platz im Hotel gegeben werden ohne dass dafür ein anderer Gast das Hotel verlassen muss? Welche Zimmernummer erhält er?
        \item Eine unendlich lange Schlange von Gästen, die alle ein Zimmer haben möchten, kommt ins Hotel. Wie kann jedem von ihnen ein Zimmer zugewiesen werden und welche Zimmernummer erhält jeder von ihnen?
        \item Eine Reisegemeinschaft mit unendlich vielen Bussen kommt am Hotel an. In jedem dieser Busse befinden sich unendlich viele Menschen. Wie können all diese Menschen noch im Hotel untergebracht werden? Wie weiß die $n$-te Person aus dem $m$-ten Bus, welche Zimmernummer sie erhalten wird?
    \end{enumerate}
\end{exercise}

\begin{exercise}{difficult}
    Zeige, dass die Potenzmenge jeder unendlichen Menge überabzählbar ist.
\end{exercise}

\end{document}