\documentclass[../abbildungen.tex]{subfiles}

\begin{document}
\begin{exercise}{intro}
    
\end{exercise}

\begin{exercise}{easy}
    Pandabären essen täglich zwischen 10\,kg und 40\,kg Bambus. Es soll eine Abbildung \[\textsc{Bambusbedarf}\colon\Natural\rightarrow\Natural\] definiert werden, die einer bestimmten Anzahl Pandas zuordnet, wie viel Bambus sie pro Tag essen.
    \begin{enumerate}
        \item Finde eine Berechnungsvorschrift für \textsc{Bambusbedarf} und nehme dafür an, dass Pandas immer genau 25\,kg Bambus täglich fressen.
        \item Zeichne ein Koordinatensystem, das auf der $x$-Achse Werte zwischen $0$ und $4$ darstellen kann und auf der $y$-Achse Werte zwischen $0$ und $100$.
        \item Trage jeweils einen Punkt in das Koordinatensystem ein, der die Abbildungsvorschrift von \textsc{Bambusbedarf} für $0, 1, 2, 3, 4$ Pandas beschreibt.
    \end{enumerate}
\end{exercise}

\begin{exercise}{easy}
    Welche der folgenden Koordinatensysteme enthalten den Graphen einer Abbildung?
    \begin{multicols}{4}
    \centering
    \begin{tikzpicture}
        \begin{axis}[defgrid, domain=0:4, y=.7cm, x=.7cm, xmin=0,xmax=4,ymin=0, ymax=4, xtick={1,...,4}, ytick={1,...,4}]
            \draw[violet] (0,3) -- (4,0);
        \end{axis}
    \end{tikzpicture}
    
    \begin{tikzpicture}
        \begin{axis}[defgrid, domain=0:4, y=.7cm, x=.7cm, xmin=0,xmax=4,ymin=0, ymax=4, xtick={1,...,4}, ytick={1,...,4}]
            \plot[violet] expression{sin(deg(x))+2};
        \end{axis}
    \end{tikzpicture}
    
    \begin{tikzpicture}
        \begin{axis}[defgrid, domain=0:4, y=.7cm, x=.7cm, xmin=0,xmax=4,ymin=0, ymax=4, xtick={1,...,4}, ytick={1,...,4}]
            \draw[violet] (2,2) circle[radius=2];
        \end{axis}
    \end{tikzpicture}
    
    \begin{tikzpicture}
        \begin{axis}[defgrid, domain=0:4, y=.7cm, x=.7cm, xmin=0,xmax=4,ymin=0, ymax=4, xtick={1,...,4}, ytick={1,...,4}]
            \draw[violet] (0,0) -- (1,4) -- (2,0) -- (3,4) -- (4,0);
        \end{axis}
    \end{tikzpicture}
\end{multicols}
\end{exercise}

\begin{exercise}{difficult}
    Für zwei Abbildungen $f\colon U\rightarrow V$ und $g\colon V\rightarrow W$ ist es möglich, das Bild, das $f$ einem Element aus $U$ zuordnet, direkt in die Abbildung $g$ einzusetzen, weil $f(u)\in V$ für alle $u\in U$ gilt und damit $f(u)$ in der Definitionsmenge von $g$ liegt.
    
    Aus welcher der Mengen $U, V$ oder $W$ kommen die folgenden Ausdrücke, falls sie definiert sind?
        \begin{enumerate}
            \item $g(v)$, wobei $u$ ein Element der Definitionsmenge $V$ von $g$ ist
            \item $g(u)$, wobei $u$ ein Element der Definitionsmenge $U$ von $f$ ist
            \item $g(x)$, wobei $x$ das Bild $f(u)$ eines Elements $u\in U$ ist
            \item $g(f(u))$, wobei $u$ ein Element der Definitionsmenge $U$ von $f$ ist
            \item $f(g(v))$, wobei $v$ ein Element der Definitionsmenge $V$ von $g$ ist
            \item $f(g(w))$, wobei $w$ ein Element aus $W$ ist
        \end{enumerate}
\end{exercise}

\begin{exercise}{difficult}
    Eine Abbildung $f\colon U\rightarrow U$, d.h. eine Abbildung, in der Definitions- und Wertebereich übereinstimmen, wird auch \textbf{Selbstabbildung} genannt.
    
    \begin{enumerate}
        \item Ist es möglich, dass es ein Element von $U$ gibt, das kein Urbild bzgl. einer bestimmten Selbstabbildung hat? 
        \item Wie viele Selbstabbildungen auf einer Menge $M$ mit $|M|=m$, bei denen zwei verschiedene Elemente immer auch verschiedene Bilder haben (d.h. $f(u)\neq f(v)$ für alle $u,v\in M$ mit $u\neq v$), gibt es?
    \end{enumerate}
\end{exercise}

\begin{exercise}{advanced}
    \begin{enumerate}
        \item Finde eine Selbstabbildung $f$ auf der Menge $M\coloneqq\{1,2,3,4,5,6\}$ mit der Eigenschaft, dass $f(f(f(n)))=n$ für alle $n\in M$, aber $f(f(n))\neq n$.
        \item Für welche Zahlen $n\in\Natural$ ist es möglich, eine Selbstabbildung auf einer Menge $M$ zu definieren, sodass jedes Element von $M$ nach $n$-maligem Anwenden der Abbildung wie in Aufgabe a) wieder auf sich selbst abgebildet wird? Wir wollen hier die Abbildung, die alle Elemente auf sich selbst abbildet, ausschließen.
        \item Wie ändert sich das Ergebnis aus Aufgabe b), wenn zusätzlich gefordert wird, dass kein $m\in M$ nach weniger als $n$ Anwendungen von $f$ auf sich selbst abgebildet werden darf?
    \end{enumerate}
\end{exercise}

\begin{exercise}{advanced}
    Wir betrachten eine Menge $M$ und die Menge $S_M$ aller Selbstabbildungen auf $M$, für die jedes Element genau ein Urbild hat, also \[S_M\coloneqq\{f\colon M\rightarrow M\,|\,\text{bzgl. $f$ hat jedes Element aus $M$ genau ein Urbild}\}.\]
    
    Beweise die folgenden Aussagen:
    \begin{enumerate}
        \item Für alle Abbildungen $f$ mit $f\in S_M$ existiert eine Abbildung $g\in S_M$, sodass \mbox{$g(f(m))=n$} für alle $f(m)\in M$.
        \item Für alle Abbildungen $f$ mit $f\in S_M$ existiert eine Abbildung $f^{-1}\in S_M$, sodass $f^{-1}(f(m))=m$ für alle $m\in M$.
        \item Für alle Abbildungen $f,g$ und $h$ mit $f,g,h\in S_M$ gilt: Wenn $f(g(m))=f(h(m))$ für alle $m\in M$ gilt, dann ist $g(m)=h(m)$ für alle $m\in M$, also ist $g=h$.
    \end{enumerate}
\end{exercise}
\end{document}