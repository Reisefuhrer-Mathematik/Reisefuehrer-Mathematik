\documentclass[../abbildungen.tex]{subfiles}

\begin{document}
\begin{exercise}{intro}
    Beschreibe, an welcher Stelle in den folgenden Beispielen jeweils eine Abbildung zu finden ist.
    \begin{enumerate}
        \item Ein Kino passt die Farbe seiner Tickets dem Genre des Filmes an. Tickets für Komödien sind grün, Zeichentrickfilme sind blau und Romanzen rot.
            \hint{Welcher Film hat welche Ticketfarbe?}
        \item Ein Schreibwarenladen verkauft Tacker für 12\,\euro{} und Stifte, Zeichenblöcke, Lineale und Radiergummis für je 1\,\euro{}.
            \hint{Welcher Artikel ist wie teuer?}
        \item Wenn eine Fußgängerampel grün zeigt, darf man die Straße überqueren. Zeigt sie rot, muss man warten.
            \hint{Was musst du machen, abhängig davon, was die Ampel anzeigt?}
    \end{enumerate}
\end{exercise}

\begin{exercise}{intro}
    Das folgende Bild stellt eine Abbildung dar. Die linke Menge ist dabei die Menge der \textsc{Grundfarben}, die rechte Menge die Menge \Natural{} der natürlichen Zahlen.
    
    \begin{center}
        \begin{tikzpicture}[scale=.6]
            \draw[grayset] (-1.5,0) ellipse (2.4cm and 2cm);
            \draw[grayset] (2.7,0) ellipse (1.2cm and 2cm);
            \node[label={[blue]left:blau}, blue] (x1) at (-1,0.7) {$\bullet$};
            \node[label={[red]left:rot}, red] (x2) at (-1,-0.2) {$\bullet$};
            \node[label={[yellow!70!black]left:gelb}, yellow!70!black] (x3) at (-1,-1.1) {$\bullet$};
            \node[label=right:4] (y1) at (2.5,0.7) {$\bullet$};
            \node[label=right:3] (y3) at (2.5,-1.1) {$\bullet$};
            \draw[->] (x1) -- (y3);
            \draw[->] (x2) to[bend right] (y1);
            \draw[->] (x3) to[bend right] (y3);
        \end{tikzpicture}
    \end{center}
    
    \begin{enumerate}
        \item Schreibe formal auf, dass die dargestellte Abbildung \textsc{Buchstabenanzahl} von der Menge der Grundfarben in die Menge der natürlichen Zahlen abbildet.
            \hint{Um zu schreiben, dass eine Abbildung $f$ von einer bestimmten \textsc{Definitionsmenge} in eine bestimmte \textsc{Bildmenge} abbildet, schreibt man $f\colon\textsc{Definitionsmenge}\rightarrow\textsc{Bildmenge}$.}
        \item Schreibe die Abbildungsvorschrift mathematisch auf.
            \hint{Um eine Regel der Abbildungsvorschrift aufzuschreiben, benutze die Schreibweise $\textsc{Abbildungsname}(\text{Urbild})=\text{Bild}$.}
    \end{enumerate}
\end{exercise}

\begin{exercise}{intro}
    Die Menge $\textsc{Tiere}\coloneqq\{\text{Pferd},\text{Spinne},\text{Biene},\text{Schnecke}\}$ soll mit der Information versehen werden, welches Tier wie viele Beine hat. Deshalb soll eine Abbildung $\textsc{Beine}$ definiert werden, die jedem Tier aus der obigen Menge die Anzahl seiner Beine zuordnet.
    \begin{enumerate}
        \item Gib für alle Tiere aus der Menge \textsc{Tiere} an, wie viele Beine sie haben, indem du eine Abbildungsvorschrift für die Abbildung \textsc{Beine} aufschreibst.
            \hint{Um eine Regel der Abbildungsvorschrift aufzuschreiben, benutze die Schreibweise $\textsc{Abbildungsname}(\text{Urbild})=\text{Bild}$.}
        \item Gib Definitions- und Bildmenge an und schreibe mathematisch auf, dass \textsc{Beine} von $\textsc{Tiere}$ in die von dir gewählte Bildmenge abbildet.
            \hint{Um zu schreiben, dass eine Abbildung $f$ von einer bestimmten \textsc{Definitionsmenge} in eine bestimmte \textsc{Bildmenge} abbildet, schreibt man $f\colon\textsc{Definitionsmenge}\rightarrow\textsc{Bildmenge}$.}
        
        %\offerHint{9}{a}
        %\offerHint{11}{b}
    \end{enumerate}
\end{exercise}

\begin{exercise}{easy}
    Die Abbildung $\textsc{Theater}$ soll beschreiben, wie oft Alan, Pauline, Noah und Kim pro Jahr ins Theater gehen.
    \begin{enumerate}
        \item Was sind Definitions- und Bildmenge dieser Abbildung?
            \hint{Die Namen Alan, Pauline, Noah und Kim werden auf die Anzahl ihrer jährlichen Theaterbesuche abgebildet.}
        \item Was bedeutet $\textsc{Theater}(\text{Alan})=4$?
        \item Schreibe formal auf, dass Pauline und Noah beide jeweils sechsmal im Jahr ins Theater gehen, Kim jedoch nur dreimal.
        
        %\offerHint{10}{a}
    \end{enumerate}
\end{exercise}

\begin{exercise}{easy}
    Anton und Marie haben Ordnungsdienst und werfen eine Münze, um zu bestimmen, wer nach der Schule den Klassenraum fegen muss. Sie einigen sich, dass bei Kopf Anton und bei Zahl Marie fegen muss. Die Abbildung $\textsc{WerFegt}$ soll abhängig vom Ergebnis des Münzwurfs angeben, wer fegen muss. 
    \begin{enumerate}
        \item Für welche Werte ist die Abbildung definiert?
        \item Wie sieht formal die Abbildungsvorschrift aus?
        \item Stelle die Abbildung $\textsc{WerFegt}$ als Diagramm dar.
    \end{enumerate}
\end{exercise}

\begin{exercise}{easy}
    Herr Meier wohnt in der Bahnhofstraße 17, Herr Müller in der Hauptstraße 6 und Herr Vogel in der Hauptstraße 42.
    \begin{enumerate}
        \item Gib Definitions- und Bildmenge für eine in diesem Zusammenhang sinnvolle Abbildung an.
        \item Halte alle relevanten Informationen über die Abbildung inklusive Abbildungsvorschrift formal fest.
    \end{enumerate}
\end{exercise}

\begin{exercise}{easy}
    Die Abbildung $\textsc{Spielt}\colon\textsc{Mannschaft}\rightarrow\textsc{Sportart}$ ist durch die Abbildungsvorschrift $\text{1. FC Köln}\mapsto\text{Fußball}$, $\text{THW Kiel}\mapsto\text{Handball}$, $\text{Rhein-Neckar-Löwen}\mapsto\text{Handball}$ definiert.
    \begin{enumerate}
        \item Beschreibe, was diese Abbildung macht.
        \item Bestimme Definitions- und Bildmenge der Abbildung.
        \item Zur Definitionsmenge soll der Hamburger SV hinzugefügt werden, der sowohl Fußball als auch Handball spielt. Ist das möglich?
    \end{enumerate}
\end{exercise}

\begin{exercise}{easy}
    Für jede natürliche Zahl $n$ gibt es einen Nachfolger, also eine Zahl, die um genau $1$ größer ist als $n$. 
    \begin{enumerate}
        \item Definiere eine Abbildung, die die Zahlen aus der Menge $M=\{1,2,3,4,5\}$ auf ihre Nachfolger abbildet.
        \item Definiere eine Abbildung, die alle Zahlen aus der Menge $M=\{2,4,6,8,10\}$ auf das Wort \enquote{gerade} abbildet.
    \end{enumerate}
\end{exercise}

\begin{exercise}{difficult}
    \begin{enumerate}
        \item Wie viele unterschiedliche Abbildungen von der Menge $\{1,2,3\}$ in die Menge $\{\text{rot},\text{gelb},\text{blau},\text{grün}\}$ gibt es?
            \hint{Überlege dir zunächst, wie viele Möglichkeiten es gibt, das Element $1$ abzubilden.}
            \hint{Das Element $1$ kann auf jede der Farben grün, rot, blau und gelb abgebildet werden. Für jede dieser Möglichkeiten kann auch die Zahl $2$ beliebig abgebildet werden. Wieviel größer wird so die Anzahl Abbildungen?}
        \item Wir betrachten Mengen $U$ und $V$ mit $|U|,|V|<\infty$. Wie viele verschiedene Abbildungen von $U$ nach $V$ gibt es?
    \end{enumerate}
\end{exercise}

\begin{exercise}{difficult}
    \parpic[r]{
        \begin{tikzpicture}
            \draw (0,0) -- (0,2) -- (4,2) -- (4,0) -- cycle;
            \fill (0,0) circle[radius=.75mm] node[below] {$A$};
            \fill (4,0) circle[radius=.75mm] node[below] {$B$};
            \fill (4,2) circle[radius=.75mm] node[above] {$C$};
            \fill (0,2) circle[radius=.75mm] node[above] {$D$};
        \end{tikzpicture}
    }
    
    Die vier Eckpunkte $A,B,C,D$ des rechts abgebildeten Rechtecks sollen durch eine Abbildung $f$ auf andere Eckpunkte des Rechtecks abgebildet werden, sodass der Abstand zweier beliebiger Eckpunkte genauso groß ist wie der Abstand von deren Bildern. 
    
    Beispielsweise soll der Abstand von $A$ und $B$ genauso groß wie der Abstand zwischen $f(A)$ und $f(B)$ sein. Abbildungen mit dieser Eigenschaft heißen \textbf{Symmetrien}.
    \begin{enumerate}
        \item Gib alle möglichen Abbildungen $f$ an, die diese Bedingung erfüllen.
            \hint{Alle gesuchten Abbildungen sind von der Art $f\colon\{A,B,C,D\}\rightarrow\{A,B,C,D\}$.}
            \hint{Was kannst du folgern, wenn du weißt, auf welchen Punkt $A$ abgebildet wird?}
        \item Finde für jede Symmetrie des dargestellten Rechtecks eine geometrische Beschreibung.
            \hint{Symmetrien beschreiben immer Kombinationen aus Drehungen und Spiegelungen.}
        \item Finde alle Symmetrien eines gleichseitigen Dreiecks, die Spiegelungen beschreiben.
            \hint{Ein gleichseitiges Dreieck hat die Eckpunkte $A,B,C$. Es reicht aus, dir zu überlegen, welcher Eckpunkt auf welchen (ggf. anderen) Eckpunkt abgebildet wird.}
        \item Ist es möglich, das Dreieck so zu drehen, dass $A$ auf $B$, $B$ auf $C$ und $C$ auf $A$ abgebildet wird, indem du ausschließlich die Abbildungen aus Aufgabe c) verwendest und ggf. nacheinander anwendest?
        \item Wie viele Symmetrien hat ein Quadrat?
        \item Finde zwei Symmetrien eines Quadrats, sodass sich jede Symmetrie des Quadrats durch Nacheinanderanwendung dieser beiden Symmetrien darstellen lässt.
    \end{enumerate}
\end{exercise}

\begin{exercise}{difficult}
    Wir betrachten Mengen $U$ und $V$ und eine Abbildung $f\colon U\rightarrow V$. Wann ist es möglich, eine Abbildung $g\colon V\rightarrow U$ zu finden, sodass $g$ die Abbildung $f$ rückgängig macht, d.h. für alle $u\in U$ gilt: Ist $f(u)=v$, dann ist $g(v)=u$?
\end{exercise}

\end{document}