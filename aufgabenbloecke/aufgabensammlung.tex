\def\pathToMain{../buch/}
\documentclass[]{uebungsblatt}
\usepackage[ngerman]{babel}
\usepackage{mathdef}
\usepackage{exercisedef}

\usepackage{subfiles}
\usepackage{wrapfig}

\usepackage{booktabs}
\usepackage{multicol}

\usetikzlibrary{positioning}

\tikzset{>=stealth}

\sheet{Aufgabenblock 0}
\title{Aufgabensammlung}
\topic{Aufgabensammlung}
\chapternum{0}

\begin{document}
    \maketitle
    \makepreamble
    
    \begin{exercise}{advanced}
        THOG (vier Objekte in zwei Farben, zwei Formen). Wähle geheim Farbe und Form. THOG bei genau einer 
        Übereinstellung. Gegeben ein THOG. Finde anderes THOG.
    \end{exercise}

    \begin{exercise}{easy}
        Funktion mit Berechnungsvorschrift gegeben, berechne Funktionswerte für vorgegebene Argumente
    \end{exercise}

    \begin{exercise}{easy}
        Funktion mit Berechnungsvorschrift gegeben, berechne Funktionswerte für vorgegebene Argumente und entscheide,
        welche davon Nullstellen oder $y$-Achsenabschnitt sind
    \end{exercise}

    \begin{exercise}{hard}
        Logik-Aufgabe mit \emph{fast alle}
    \end{exercise}

    \begin{exercise}{hard}
        Vereinfache $(-x)^y$ mit Potenzgesetzen.
    \end{exercise}

    \begin{exercise}{advanced}
        Rotationsvolumen-Aufgabe zu rotierendem Trinkglas (Aufgabe: Rotationskörper um $y$-Achse berechnen)
    \end{exercise}

    \begin{exercise}{easy}
        In jedem der folgenden Koordinatensysteme sind zwei Vektoren eingezeichnet. Entscheide jeweils, 
        ob die eingezeichneten Vektoren gleich sind, indem du ihre Einträge berechnest und miteinander vergleichst.

        \begin{multicols}{3}
            \centering

            \tikz{
                \begin{axis}[defgrid, domain=0:4, y=1cm, x=1cm, ymin=0, ymax=4, xmin=0, xmax=4, xtick={1,...,4}, ytick={1,...,4}]
                    \draw[-latex,ultra thick, violet] (0,1) -- node[above] {$a$} (3,1);
                    \draw[-latex,ultra thick, orange] (4,1) -- node[left] {$b$} (4,4);
                \end{axis} 
            }

            \tikz{
                \begin{axis}[defgrid, domain=0:4, y=1cm, x=1cm, ymin=0, ymax=4, xmin=0, xmax=4, xtick={1,...,4}, ytick={1,...,4}]
                    \draw[-latex, ultra thick, blue!60] (0,0) -- node[above] {$c$} (3,1);
                    \draw[-latex,ultra thick, orange] (1,2) -- node[above] {$d$} (4,3);
                \end{axis} 
            }

            \tikz{
                \begin{axis}[defgrid, domain=0:4, y=1cm, x=1cm, ymin=0, ymax=4, xmin=0, xmax=4, xtick={1,...,4}, ytick={1,...,4}]
                    \draw[-latex, ultra thick, blue!60] (0,4) -- node[right] {$e$} (1,1);
                    \draw[-latex,ultra thick, orange] (3,0) -- node[right] {$f$} (2,3);
                \end{axis} 
            }
        \end{multicols}
    \end{exercise}

    \begin{exercise}{easy}
        Zeichne die Vektoren $a=\vectwo{-2}{1}$ und $b=\vectwo{-1}{6}$ so in ein Koordinatensystem ein, dass sie beide 
        im Punkt $\coord{0}{0}$ beginnen. Bestimme anschließend das Ergebnis der folgenden Rechnungen.
        \begin{enumerate}
            \begin{multicols}{3}
                \item $2\cdot \vectwo{-2}{1}$
                \item $\vectwo{-1}{6}+\vectwo{-2}{1}$
                \item $\vectwo{-1}{6}-2\cdot\vectwo{-2}{1}$
            \end{multicols}
        \end{enumerate}
        Zeichne die Ergebnisse der Rechnungen ebenfalls so in das Koordinatensystem ein, dass die Vektoren im Punkt 
        $\coord{0}{0}$ beginnen. Verbinde die Pfeilspitzen der Vektoren aus den Teilaufgaben b) und c) mit der 
        Pfeilspitze des Vektors $b$. Beschreibe jeweils, was dir auffällt.
    \end{exercise}

    \begin{exercise}{normal}
        \parpic[r]{
            \tikz{
                \begin{axis}[defgrid, domain=0:4, y=1cm, x=1cm, ymin=0, ymax=4, xmin=0, xmax=4, xtick={1,...,4}, ytick={1,...,4}]
                    \draw[-latex, ultra thick, blue!60] (0,0) -- node[right] {$a$} (1,1);
                    \draw[-latex,ultra thick, violet] (0,2) -- node[below] {$b$} (2,3);
                    \draw[-latex,ultra thick, orange] (0,4) -- node[below] {$c$} (4,1);
                \end{axis} 
            }
        }
        \begin{enumerate}
            \item Bestimme für die rechts abgebildeten Vektoren 
            \[a=\vectwo{a_1}{a_2}, b=\vectwo{b_1}{b_2}, 
            c=\vectwo{c_1}{c_2}\] 
            jeweils die passenden Werte
            $a_1,a_2,b_1,b_2,c_1,c_2$.
            \item Zeichne (ohne zu rechnen) die Vektoren $a+b$ und $c-b$ in ein Koordinatensystem ein. berechne
            anschließend ihre Einträge und kontrolliere dein Ergebnis mit deiner Zeichnung.
            \item Zeichne den Vektor $3a$ in das Koordinatensystem aus Aufgabe b) ein und berechne seine Einträge.
        \end{enumerate}
    \end{exercise}

    \begin{exercise}{normal}
        Finde zwei Zahlen $a$ und $b$, sodass $\vectwo{-3}{-4}=a\cdot\vectwo{-3}{4}+b\cdot\vectwo{-2}{2}$ gilt.
    \end{exercise}

    \begin{exercise}{hard}
        \begin{enumerate}
            \item Zeichne den Vektor $a=\vectwo{3}{4}$ in ein Koordinatensystem ein und verwende den Satz des Pythagoras, 
            um seine Länge $\abs{a}$ zu berechnen.
            \item Gib eine allgemeine Formel an, mit der du die Länge $\abs{v}$ eines beliebigen Vektors 
            $v=\vectwo{v_1}{v_2}$ berechnen kannst.
        \end{enumerate}
    \end{exercise}

    \begin{exercise}{advanced}
        \textbf{Die komplexen Zahlen als $\R$-Vektorraum}
        \begin{enumerate}
        \item Zeige, dass die komplexen Zahlen $\C$ ein $\R$-Vektorraum sind (als Vektoraddition verwendet man die
        gewöhnliche Addition von komplexen Zahlen und als Skalarmultiplikation die gewöhnliche Multiplikation mit
        reellen Zahlen).
        \item Bestimme komplexe Zahlen $z_1,\dots,z_n$, sodass für jede komplexe Zahl $z\in\C$ eindeutig bestimmte 
        reelle Zahlen $\lambda_1,\dots\lambda_n$ existieren mit
        \[z=\lambda_1z_1+\lambda_2z_2+\dots+\lambda_nz_n.\]
        \emph{Hinweis: Dies funktioniert nur für ein bestimmtes $n$. Diese Zahl $n$ heißt die \textbf{Dimension} 
        von $\C$ als $\R$-Vektorraum, geschrieben $\dim_{\R}(\C)$.}
        \end{enumerate}
    \end{exercise}

    \begin{exercise}{advanced}
        Finde eine geeignete Vektoraddition und Skalarmultiplikation, um die folgenden Mengen zu $\R$-Vektorräumen zu 
        machen (beweise dafür, dass die Vektorraumaxiome mit den von dir gewählten Operationen erfüllt sind).
        \begin{enumerate}
            \item Die Menge aller Polynome über $\R$, definiert als 
            \[\R[x]:=\Biggl\{\sum_{i=0}^\infty a_ix^i\mid a_i\in\R, a_i\neq 0\text{ für nur endlich viele }i\Biggr\}.\]
            \emph{Beispiele: $x^2+1\in\R[x], x^3-\sqrt{2}x\in\R[x]$.}
            \item Die Menge aller Funktionen $f\colon X\rightarrow \R$, definiert als
            \[\text{Abb}(X,\R):=\{f\mid f\colon X\rightarrow\R\text{ ist eine Funktion}\}\]
            für eine beliebige Menge $X$.
        \end{enumerate}
    \end{exercise}

    \begin{exercise}{easy}
        Finde unter den folgenden Vektoren die Paare von Vektoren, die linear abhängig sind.
        \[\vectwo{-1}{4}, \vectwo{0}{2}, \vectwo{2}{0}, \vectwo{1}{0}, \vectwo{0}{5}, \vectwo{3}{12}\]
    \end{exercise}

    \newpage
    
    %\makesolutions
    
    \newpage
    
    \makehints
\end{document}