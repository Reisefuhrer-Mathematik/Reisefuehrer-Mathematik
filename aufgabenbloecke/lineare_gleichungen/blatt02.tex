\documentclass[../lineare_gleichungen.tex]{subfiles}

\begin{document}
    \begin{exercise}{easy}
        \parpic[r]{
            \begin{linearEquation}
                %Füllung linke Waagschale
                \fill (-1.05,0.25) -- (-1.45,0.25) -- (-1.4,0.6) -- (-1.1,0.6) -- cycle;
                \draw[line width=0.75mm] (-1.25,0.66) circle[radius=0.06cm];
                \node[white] at (-1.25,0.42) {$7$};
                \node[white,marble,inner sep=.12cm] at (-0.95,0.35) {$x$};
                \node[white,marble,inner sep=.12cm] at (-0.7,0.35) {$x$};
                %Füllung rechte Waagschale
                \fill (0.95,0.25) -- (0.55,0.25) -- (0.6,0.6) -- (0.9,0.6) -- cycle;
                \draw[line width=0.75mm] (0.75,0.66) circle[radius=0.06cm];
                \node[white] at (0.75,0.42) {$5$};
                \node[white,marble,inner sep=.12cm] at (1.05,0.35) {$x$};
                \node[white,marble,inner sep=.12cm] at (1.18,0.35) {$x$};
                \node[white,marble,inner sep=.12cm] at (1.31,0.35) {$x$};
            \end{linearEquation}
        }
        Du siehst rechts die Gleichung $7+2x=5+3x$. Welche der folgenden Gleichungen kannst du durch eine Äquivalenzumformung aus der abgebildeten Gleichung erhalten? Gib jeweils die verwendete Äquivalenzumformung an.
        \begin{multicols}{2}
            \begin{enumerate}
                \item $7=5+x$
                \hint{Gibt es einen Term, den du auf beiden Seiten subtrahieren kannst, damit du diese Gleichung erhältst?}
                \item $9x+12=3x+30$
                \hint{Wenn dir kein passender Term einfällt, den du auf beiden Seiten addieren oder subtrahieren kannst, versuch es mal mit Multiplikation.}
                \item $0=0$
                \hint{Nicht jede Multiplikation ist eine Äquivalenzumformung.}
                \item $2+2x=3x$
            \end{enumerate}
        \end{multicols}
    \end{exercise}
    \begin{exercise}{easy}
        \parpic[r]{
            \begin{linearEquation}
                %Füllung linke Waagschale
                \fill (-0.95,0.25) -- (-0.55,0.25) -- (-0.6,0.6) -- (-0.9,0.6) -- cycle;
                \draw[line width=0.75mm] (-0.75,0.66) circle[radius=0.06cm];
                \node[white] at (-0.75,0.42) {$4$};
                \node[white,marble,inner sep=.12cm] at (-1,0.35) {$x$};
                \node[white,marble,inner sep=.12cm] at (-1.18,0.35) {$x$};
                \node[white,marble,inner sep=.12cm] at (-1.36,0.35) {$x$};
                %Füllung rechte Waagschale
                \fill (0.95,0.25) -- (1.45,0.25) -- (1.4,0.65) -- (1,0.65) -- cycle;
                \draw[line width=0.75mm] (1.2,0.71) circle[radius=0.06cm];
                \node[white] at (1.2,0.45) {$10$};
                \node[white,marble,inner sep=.12cm] at (0.75,0.35) {$x$};
            \end{linearEquation}
        }
        Im folgenden Bild siehst du die Gleichung $3x+4=x+10$ als Balkenwaage dargestellt. Gib jeweils die Gleichung an, die entsteht, wenn du auf der Waage
        \begin{enumerate}
            \item auf beiden Seiten eine Kugel entfernst
            \hint{Wenn du einen Gegenstand entfernst, dann ist das eine Subtraktion.}
            \hint{Die Kugel hast das Gewicht $x$. Du subtrahierst also $x$, wenn du sie entfernst.}
            \item auf beiden Seiten 4\,g Gewicht entfernst
            \item auf beiden zwei Kugeln hinzufügst
        \end{enumerate}
        Welche Äquivalenzumformung hast du jeweils verwendet?
    \end{exercise}
    \begin{exercise}{easy}
        Wende auf die folgenden Gleichungen jeweils die angegebene Äquivalenzumformung an.
        %\begin{multicols}{2}
            \begin{enumerate}
                \item $x^2+3=7$ (Äquivalenzumformung $-3$)
                \hint{Du musst von beiden Seiten $3$ subtrahieren.}
                \item $9=\frac{2x+5}{2}$ (Äquivalenzumformung $\cdot 2$)
                \hint{Du musst beiden Seiten mit $2$ multiplizieren.}
                \item $-x=12$ (Äquivalenzumformung $:2$)
                \hint{Du musst beiden Seiten durch $2$ teilen.}
                \item $7x-x^2=7-x^2$ (Äquivalenzumformung $+x^2$)
                \hint{Du musst zu beiden Seiten $x^2$ addieren.}
            \end{enumerate}
        %\end{multicols}
    \end{exercise}
    \begin{exercise}{easy}
        Gib für die folgenden Gleichungen jeweils die Lösungsmenge \Solutions an.
        \begin{multicols}{3}
        \begin{enumerate}
            \item $x=10$
            \hint{Die Lösungsmenge dieser Gleichung enthält alle Werte, die du für $x$ einsetzen kannst, damit $x=10$ erfüllt ist.}
            \hint{Das Symbol für die Lösungsmenge ist \Solutions. Wenn eine Gleichung zum Beispiel die Lösungen 4 und 7 hat, dann ist ihre Lösungsmenge $\Solutions=\{4,7\}$.}
            \item $4=2$
            \item $27+16=x$
        \end{enumerate}
    \end{multicols}
    \end{exercise}
    \begin{exercise}{easy}
        Die linke der beiden Balkenwaagen befindet sich gerade im Gleichgewicht. Das Gewicht der Kugel ist auf beiden Waagen gleich.
        \begin{center}
            \begin{linearEquation}
                %Füllung linke Waagschale
                \fill (-0.95,0.25) -- (-0.55,0.25) -- (-0.6,0.6) -- (-0.9,0.6) -- cycle;
                \draw[line width=0.75mm] (-0.75,0.66) circle[radius=0.06cm];
                \node[white] at (-0.75,0.42) {$6$};
                \node[white,marble,inner sep=.12cm] at (-1.2,0.35) {$x$};
                %Füllung rechte Waagschale
                \fill (0.75,0.25) -- (1.25,0.25) -- (1.2,0.65) -- (0.8,0.65) -- cycle;
                \draw[line width=0.75mm] (1,0.71) circle[radius=0.06cm];
                \node[white] at (1,0.45) {$15$};
            \end{linearEquation}
            \bigarrow{
                \node[circle, inner sep=1mm, fill=white,draw=maincolor] at (0.6,0.2) {\textbf{?}};
            }
            \begin{linearEquation}
                %Füllung linke Waagschale
                \node[white,marble,inner sep=.12cm] at (-1,0.35) {$x$};
                %Füllung rechte Waagschale
            \end{linearEquation}
        \end{center}
        \begin{enumerate}[label=\alph*)]
            \item Welches Gewicht musst du in die freie Waagschale der rechten Waage legen, damit sich auch die rechte Balkenwaage im Gleichgewicht befindet?
            \item Gib für jede der beiden Waagen an, welche Gleichung sie darstellt.
            \item Mit welcher Äquivalenzumformung kommst du von der Gleichung der linken Waage zur Gleichung der rechten Waage?
            \item Gib die Lösungsmenge rechten Gleichung an. Welche Lösungsmenge hat die linke Gleichung?
        \end{enumerate}
    \end{exercise}
    \begin{exercise}{normal}
        Sind die beiden Gleichungen äquivalent? Falls ja, gib die verwendete Äquivalenzumformung an und bestimme die Lösungsmenge.
        \begin{enumerate}
        \begin{multicols}{3}
            \item
            \begin{align*}
                2x&=x+7\\
                x&=7
            \end{align*}

            \item
            \begin{align*}
                4x&=5x\\
                4&=5
            \end{align*}

            \item
            \begin{align*}
                5x&=40\\
                x&=8
            \end{align*}
        \end{multicols}
    \end{enumerate}
    \end{exercise}
    \begin{exercise}{normal}
        Wähle jeweils eine geeignete Äquivalenzumformung aus dem Vorrat auf der rechten Seite, um die Gleichung zu vereinfachen.
        \begin{multicols}{3}
            \begin{enumerate}
                \item $3x+11=2x+11$
                \item $4x+3\cdot (x+1)=7+4x$
                \item $\frac{1}{2}x+\frac{9}{2}=\frac{61}{2}$
                \item $-12=x-12$
                \item $\frac{x+4}{4}=\frac{3x}{2}+\frac{7}{2}$
                \item $4\cdot (x+3)=4x+12$
            \end{enumerate}
        \end{multicols}
    \end{exercise}
    \begin{exercise}{hard}
        Welches der Zeichen $\Rightarrow, \Leftarrow$ und $\Leftrightarrow$ gehört jeweils in die Mitte?
        \begin{multicols}{2}
            \begin{enumerate}
                \item $x=4 ~\Box~ x+3=7$
                \hint{$x=4\Rightarrow x+3=7$ bedeutet, dass für jeden Wert für $x$, für den $x=4$ gilt, auch $x+3=7$ gelten muss.}
                \hint{$x=4\Leftarrow x+3=7$ bedeutet, dass für jeden Wert für $x$, für den $x+3=7$ gilt, auch $x=4$ gelten muss.}
                \hint{$x=4\Leftrightarrow x+3=7$ bedeutet, dass die beiden Gleichungen äquivalent sind. Sie müssen dann also die gleiche Lösungsmenge haben.}
                \item $x=0 ~\Box~ x\cdot 0=0$
                \item $x^2=16 ~\Box~ x=4$
                \item $x=5 ~\Box~ |x|=5$
            \end{enumerate}
        \end{multicols}
    \end{exercise}
    \knobelaufgaben
    \begin{exercise}[Alle Basen]{advanced}
        Aufgabe
    \end{exercise}
\end{document}