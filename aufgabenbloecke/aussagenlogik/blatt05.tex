\documentclass[../aussagenlogik.tex]{subfiles}

\begin{document}

\begin{exercise}{difficult}
    Beweise die folgenden Aussagen mithilfe von vollständiger Induktion.
    \begin{multicols}{3}
        \begin{enumerate}
            \item $\displaystyle\sum_{i=1}^ni=\frac{n(n+1)}{2}$
            \item $\displaystyle\sum_{i=1}^ni^2=\frac{n(n+1)(2n+1)}{6}$
            \item $\displaystyle\sum_{i=1}^ni^3=\biggl(\frac{n(n+1)}{2}\biggr)^2$
        \end{enumerate}
    \end{multicols}
    %https://www.emath.de/Referate/induktion-aufgaben-loesungen.pdf
\end{exercise}
\begin{exercise}{difficult}
    Beweise die folgenden Behauptungen mithilfe der Beweistechniken, die du kennengelernt hast.
    \begin{enumerate}
        \item Die Summe dreier aufeinander folgender positiver ganzer Zahlen ist stets durch 3 teilbar (Direkter Beweis).
        \item Für alle positiven Zahlen $a,b$ ist $\sqrt{ab}\leq\frac{a+b}{2}$ (Widerspruchsbeweis).
        \item Die Zahl $n^5-n$ ist für jede positive ganze Zahl $n$ durch 5 teilbar (Induktion)
        \item Für alle positiven ganzen Zahlen $n$ gilt: Falls $3$ ein Teiler von $n^2$ ist, dann ist $3$ auch ein Teiler von $n$ (Kontraposition).
    \end{enumerate}
\end{exercise}
\begin{exercise}{difficult}
    Unter 12 paarweise verschiedenen zweistelligen Zahlen gibt es zwei, deren Differenz durch 11 teilbar ist (also zweistellige Schnapszahl).
\end{exercise}
\end{document}