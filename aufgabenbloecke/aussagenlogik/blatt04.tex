\documentclass[../aussagenlogik.tex]{subfiles}

\begin{document}

\begin{exercise}{difficult}
    Der Konnektor $\lnand$ (\emph{NAND}, Abkürzung für \enquote{not and}) ist über die folgende Wahrheitstabelle definiert:
    \[\begin{array}{cc s c}\toprule
        A & B & A \lnand B\\\midrule
        \falsch   & \falsch   & \wahr  \\
        \falsch   & \wahr & \wahr\\
        \wahr & \falsch   & \wahr\\
        \wahr & \wahr & \falsch\\\bottomrule
    \end{array}\]
    Zeige, dass jede zusammengesetzte Aussage äquivalent zu einer Aussage ist, die nur den Konnektor $\lnand$ verwendet.
\end{exercise}
\end{document}