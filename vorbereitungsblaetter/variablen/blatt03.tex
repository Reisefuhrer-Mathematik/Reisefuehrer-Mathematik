\def\pathToMain{../../buch/}
\documentclass[solution]{uebungsblatt}
\usepackage[utf8]{inputenc}
\usepackage[T1]{fontenc}
\usepackage[ngerman]{babel}

\usepackage{mathdef}

\sheet{Vorbereitungsblatt 8.3}
\title{Ausmultiplizieren}
\topic{\getchaptername{variablen}}
\chapternum{\getchapternum{variablen}}

\usepackage{multicol}

\begin{document}
\maketitle
\begin{contents}
    Klammern ausmultiplizieren, Minusklammern, Binomische Formeln
\end{contents}

\video{Klammern ausmultiplizieren}{4}{Kapitel \ref{ext:sec:abbildungen_intuition} (ab Seite \pageref{ext:sec:abbildungen_intuition})}{https://www.google.de}

\begin{remark}{}
    Jeder mit jedem (Orgie!!)
\end{remark}

\subsection*{Aufgabenteil}

\begin{exercise}

\end{exercise}

\begin{exercise}

\end{exercise}

\begin{exercise}

\end{exercise}

\video{Minusklammern ausmultiplizieren}{4}{Kapitel \ref{ext:sec:abbildungen_intuition} (ab Seite \pageref{ext:sec:abbildungen_intuition})}{https://www.google.de}

\begin{remark}{}
    Minus dreht Vorzeichen in der Klammer um
\end{remark}

\subsection*{Aufgabenteil}

\begin{exercise}

\end{exercise}

\begin{exercise}

\end{exercise}

\newpage

\video{Die binomischen Formeln}{4}{Kapitel \ref{ext:sec:abbildungen_intuition} (ab Seite \pageref{ext:sec:abbildungen_intuition})}{https://www.google.de}

\begin{theorem}
    Es gilt:
    \begin{align}
        (a+b)^2&=a^2+2ab+b^2\\
        (a-b)^2&=a^2-2ab+b^2\\
        (a+b)(a-b)&=a^2-b^2
    \end{align}
\end{theorem}

\subsection*{Aufgabenteil}

\mandala{mandala/mandala01}

\end{document}
