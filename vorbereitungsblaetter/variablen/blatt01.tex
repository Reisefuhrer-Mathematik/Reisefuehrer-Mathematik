\def\pathToMain{../../buch/}
\documentclass[solution]{uebungsblatt}
\usepackage[utf8]{inputenc}
\usepackage[T1]{fontenc}
\usepackage[ngerman]{babel}

\usepackage{mathdef}

\sheet{Vorbereitungsblatt 8.1}
\title{Einführung: Variablen}
\topic{\getchaptername{variablen}}
\chapternum{\getchapternum{variablen}}

\usepackage{multicol}

\begin{document}
\maketitle
\begin{contents}
    Auswerten von Termen mit Variablen, Aufstellen von Termen für Sachzusammenhänge
\end{contents}

\video{Was ist eine Variable?}{4}{Kapitel \ref{ext:sec:abbildungen_intuition} (ab Seite \pageref{ext:sec:abbildungen_intuition})}{https://www.google.de}

Idee: Lückentext --- Einsetzen mit einer Variable, mehrere Werte --- Einsetzen mit mehreren Variablen --- Entscheiden, ob Terme gleich sind --- Terme selbst aufstellen (bzw. entscheiden, welcher passt)

\subsection*{Aufgabenteil}
\begin{exercise}
    Irgendein besonders intelligenter Lückentext vielleicht
\end{exercise}
\begin{exercise}
    Setze in den Term $15+3\cdot x+y^2$ für $x$ den Wert $4$ und für $y$ den Wert $8$ ein und rechne den Term anschließend aus.
    \begin{answerbox}[0.4in]
        $15+3x+y^2=15+3\cdot 4+8^2=15+12+64=91$
    \end{answerbox}
\end{exercise}
\begin{exercise}
    Berechne den Wert $(371\cdot x)\cdot y+42$ möglichst geschickt, wenn $x=25$ und $y=4$ ist.
    \begin{answerbox}[0.4in]
        $(371\cdot x)\cdot y+42=(371\cdot 25)\cdot 4+42=371\cdot (25\cdot 4)+42=371\cdot 100+42=37100+42=37142$
    \end{answerbox}
\end{exercise}
\begin{exercise}
    Kreuze alle Terme an, die unabhängig davon, welchen Wert du für $x$ einsetzt, denselben Wert wie $6\cdot x+2$ haben. Probiere, wenn du dir nicht sicher bist, ein paar Werte für $x$ aus, für die du die Terme jeweils ausrechnest.
    \begin{multicols}{3}
        \begin{multiplechoice}
            \citem $10+6\cdot x-8$
            \item $6\cdot (x+2)$
            \item $12-3\cdot (3-2\cdot x)$
        \end{multiplechoice}
    \end{multicols}
    \begin{answerbox}[1.2in]
        Die erste Antwortmöglichkeit ist richtig, denn $10+6\cdot x-8=+6\cdot x+10-8=6\cdot x+2$.

        Wir setzen für $x$ den Wert $0$ ein, um zu sehen, dass der zweite und dritte Term dann andere Werte annehmen als der Term $6\cdot x+2$. Mit $x=0$ ist $6\cdot x+2=6\cdot 0+2=2$.
        
        Allerdings ist $6\cdot (x+2)=6\cdot (0+2)=12$. Also ist die zweite Antwortmöglichkeit falsch.

        Die dritte Antwortmöglichkeit ist falsch, denn $12-3\cdot (3-2\cdot 0)=12-3\cdot 3=12-9=3$.
    \end{answerbox}
\end{exercise}

\video{Terme mit Variablen aufstellen}{4}{Kapitel \ref{ext:sec:abbildungen_intuition} (ab Seite \pageref{ext:sec:abbildungen_intuition})}{https://www.google.de}

\subsection*{Aufgabenteil}

\begin{exercise}
    {\itshape Es ist Frühlingsanfang und du möchtest deinen Garten pflegen, sodass er dieses Jahr wieder schön blühen kann.
    \begin{itemize}[nosep]
        \item Du brauchst eine halbe Stunde, um deine Garteninstrumente zusammenzusuchen.
        \item Du brauchst ca. 3 Minuten pro Quadratmeter, um ihn zu Mähen, Harken und zu bepflanzen.
        \item An drei der vier Seiten deines quadratischen Gartens sind Hecken, die du stutzen musst.
        \item Um die Hecke zurechtzustutzen, brauchst du 5 Minuten für jeden Meter
    \end{itemize}}
    Gib' einen Term an, der die Gesamtzeit in Minuten beschreibt, die du an der Gartenarbeit sitzt, wenn dein Garten $x\times x$ Meter groß ist.
    \begin{answerbox}
        $30\,\text{min} + x^2\cdot 3\,\text{min} + 3x\cdot 5\,\text{min} = (3x^2+15x+30)\,\text{min}$
    \end{answerbox}
\end{exercise}
\begin{exercise}
    Sachbeispiel, das formalisiert werden soll. Variablen sind vorgegeben und die richtige Antwort soll per Multiple Choice gewählt werden. (zweites Beispiel)
\end{exercise}

\mandala{mandala/mandala01}

\end{document}
