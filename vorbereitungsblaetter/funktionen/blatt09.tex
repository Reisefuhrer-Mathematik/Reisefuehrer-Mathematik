\def\pathToMain{../../buch/}
\documentclass[]{uebungsblatt}
\usepackage[utf8]{inputenc}
\usepackage[T1]{fontenc}
\usepackage[ngerman]{babel}

\usepackage{mathdef}
\usetikzlibrary{positioning}

\sheet{Vorbereitungsblatt 9}
\title{Abbildungsgraphen strecken und stauchen}
\topic{\getchaptername{abbildungen}}
\chapternum{\getchapternum{abbildungen}}

\usepackage{multicol}

\begin{document}
\maketitle
\begin{contents}
    Auf diesem Vorbereitungsblatt geht es darum, wie sich \textbf{Abbildungsgraphen strecken und stauchen} lassen. Dabei wird unterschieden, wie die \textbf{Veränderung der Berechnungsvorschrift} bei einer Streckung/Stauchung in $x$-Richtung aussehen muss und wie sie sich bei einer Streckung/Stauchung in $y$-Richtung ändern muss.
\end{contents}

\mandala{mandala/mandala09}

\end{document}