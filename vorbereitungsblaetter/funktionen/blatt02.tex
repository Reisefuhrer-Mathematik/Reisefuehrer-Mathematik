\def\pathToMain{../../buch/}
\documentclass[]{uebungsblatt}
\usepackage[utf8]{inputenc}
\usepackage[T1]{fontenc}
\usepackage[ngerman]{babel}

\usepackage{mathdef}
\usepackage{tikzdef}

\usepackage{booktabs}
\usetikzlibrary{positioning}

\sheet{Vorbereitungsblatt 2}
\title{Abbildungsvorschriften beschreiben}
\topic{\getchaptername{abbildungen}}
\chapternum{\getchapternum{abbildungen}}

\usepackage{multicol}

\begin{document}
\maketitle
\begin{contents}
    Dieses Vorbereitungsblatt führt verschiedene Möglichkeiten ein, wie Abbildungsvorschriften angegeben werden können. Es beschäftigt sich mit der \textbf{expliziten Schreibweise}, mit \textbf{Wertetabellen} und mit \textbf{Berechnungsvorschriften}.
\end{contents}

\video{Explizite Abbildungsvorschriften}{2}{Kapitel \ref{ext:sec:abbildungen_explizite_schreibweise} (ab Seite \pageref{ext:sec:abbildungen_explizite_schreibweise})}{https://www.google.de}

\begin{remark}{}
    Eine Abbildungsvorschrift kann angegeben werden, indem man zu jedem Element der Definitionsmenge \emph{explizit} aufschreibt, welches Bildelement ihm zugeordnet wird. Dafür nutzt man die Schreibweise $f(\text{Argument})=\text{Bild}$ und listet für jedes Element der Definitionsmenge eine solche Regel auf.
\end{remark}

\subsection*{Aufgabenteil}
\begin{exercise}
    Wenn du eine \textchoice{Menge, Summe, Abbildung} mit der expliziten Schreibweise beschreibst, dann benötigst du eine Abbildungsregel für jedes Element der \textchoice{Definitionsmenge, Zielmenge}.
\end{exercise}
\begin{exercise}
    Notiere zu jedem der folgenden Diagramme eine Abbildungsvorschrift in expliziter Schreibweise. Links ist die Abbildung $f\colon\textsc{Grundfarben}\rightarrow\textsc{NeueFarben}$ dargestellt, rechts die Abbildung $g\colon\{3,5,6,8,12\}\rightarrow\{0,1,2,3\}$.\\
    \begin{multicols}{2}\centering
        \input{
            \begin{center}
                \begin{tikzpicture}[scale=.6]
                    \draw[grayset] (-1.5,0) ellipse (2.4cm and 2cm);
                    \draw[grayset] (4,0) ellipse (2.4cm and 2cm);
                
                    \node[blue] (x1) at (-1,0.7) {$\bullet$}; 
                    \node[blue,left=1mm of x1] {blau};
                    \node[red] (x2) at (-1,-0.2) {$\bullet$}; 
                    \node[red,left=1mm of x2] {rot};
                    \node[yellow!70!black] (x3) at (-1,-1.1) {$\bullet$}; 
                    \node[yellow!70!black,left=1mm of x3] {gelb};
                    
                    \node[green!70!black] (y1) at (3.5,0.7) {$\bullet$}; 
                    \node[green!70!black,right=1mm of y1] {grün};
                    \node[violet] (y2) at (3.5,-0.2) {$\bullet$}; 
                    \node[violet,right=1mm of y2] {lila};
                    \node[blue] (y3) at (3.5,-1.1) {$\bullet$}; 
                    \node[blue,right=1mm of y3] {blau};
                
                    \draw[->] (x1) to[bend right] (y3);
                    \draw[->] (x2) -- (y2);
                    \draw[->] (x3) to[bend right] (y1);
                \end{tikzpicture}
            \end{center}
        }
        
        \begin{tikzpicture}[scale=.5]
            \draw[grayset] (-1.5,0) ellipse (2cm and 2.5cm);
            \draw[grayset] (3.5,0) ellipse (2cm and 2.5cm);
        
            \node[label=left:8] (x1) at (-1,1.6) {$\bullet$};
            \node[label=left:6] (x2) at (-1,0.7) {$\bullet$};
            \node[label=left:3] (x3) at (-1,-0.2) {$\bullet$};
            \node[label=left:12] (x4) at (-1,-1.1) {$\bullet$};
            \node[label=left:5] (x5) at (-1,-2) {$\bullet$};
            
            \node[label=right:0] (y1) at (3,1.15) {$\bullet$};
            \node[label=right:1] (y2) at (3,0.25) {$\bullet$};
            \node[label=right:2] (y3) at (3,-0.65) {$\bullet$};
            \node[label=right:3] (y4) at (3,-1.55) {$\bullet$};
            
            \draw[->] (x1) -- (y1);
            \draw[->] (x2) to[bend right] (y3);
            \draw[->] (x3) to[bend right] (y4);
            \draw[->] (x4) -- (y1);
            \draw[->] (x5) -- (y2);
        \end{tikzpicture}
    \end{multicols}
    \begin{answerbox}[1in]
        $f(\text{blau})=\text{blau}, f(\text{rot})=\text{lila}, f(\text{gelb})=\text{grün}$\\
        $g(8)=0, g(6)=2, g(3)=3, g(12)=0, g(5)=1$
    \end{answerbox}
\end{exercise}

\video{Arbeiten mit Wertetabellen}{6}{Kapitel \ref{ext:sec:abbildungen_wertetabellen} (ab Seite \pageref{ext:sec:abbildungen_wertetabellen})}{https://www.google.de}
\newpage
\begin{remark}{}
    \parpic[r]{
        \begin{tabular}{cc}\toprule
            $x$ & $f(x)$ \\\midrule
            0 & 1\\
            1 & 2\\
            \dots & \dots\\\bottomrule
        \end{tabular}
    }
        
    Eine Abbildungsvorschrift kann angegeben werden, indem sie als Tabelle mit einer Spalte für die Definitions- und einer Spalte für die Zielmenge notiert wird. In einer Zeile einer solchen Tabelle stehen jeweils ein Argument aus der Definitionsmenge und sein Bild.
    
    Die in einer Wertetabelle notierten Abbildungsregeln sind gleichbedeutend mit Regeln in der expliziten Schreibweise.
\end{remark}

\subsection*{Aufgabenteil}
\begin{exercise}
    Die folgende Wertetabelle gehört zum linken Diagramm aus Aufgabe 2. Ergänze die fehlenden Werte so, dass sie die Abbildungsvorschrift aus Aufgabe 2 darstellt.
    \begin{center}
        \begin{tabular}{cc}\toprule
            Farbe & Neue Farbe\\\midrule
            blau & \answerfield{2cm}{blau}\\
            \answerfield{2cm}{gelb} & grün\\
            rot & \answerfield{2cm}{lila}\\\bottomrule
        \end{tabular}
    \end{center}
\end{exercise}
\begin{exercise}
    Beim Kartenspiel Skat spielt ein einzelner Spieler gegen ein Team aus zwei Spielern und versucht, mehr Punkte als das Team zu sammeln. Die folgende Tabelle stammt aus den Skatregeln.
    \begin{center}
        \begin{tabular}{cc}\toprule
            Farbe & Punkte\\\midrule
            $\clubsuit$ & 12\\
            $\spadesuit$ & 11\\
            $\heartsuit$ & 10\\
            $\diamondsuit$ & 9\\\bottomrule
        \end{tabular}
    \end{center}
    
    In der Tabelle ist die Abbildung $\textsc{Punktwert}\colon\{\diamondsuit,\heartsuit,\spadesuit,\clubsuit\}\rightarrow\{9,10,11,12\}$ beschrieben. Ergänze die fehlenden Werte in der nachfolgenden Liste, um sie in der expliziten Schreibweise aufzuschreiben.
    \[
        \textsc{Punktwert}(\diamondsuit)=\answerfield{.5cm}{9},
        \textsc{Punktwert}(\clubsuit)=\answerfield{.5cm}{12},
        \textsc{Punktwert}(\spadesuit)=\answerfield{.5cm}{11},
        \textsc{Punktwert}(\heartsuit)=\answerfield{.5cm}{10}
    \]
\end{exercise}

\video{Berechnungsvorschriften}{5}{Kapitel \ref{ext:sec:abbildungen_berechnungsvorschriften} (ab Seite \pageref{ext:sec:abbildungen_berechnungsvorschriften})}{https://www.google.de}
\begin{remark}{}
    Eine Abbildungsvorschrift kann angegeben werden, indem man notiert, wie aus einem \emph{beliebigen} Element $x$ der Definitionsmenge bestimmt werden kann, welches Bild es erhält. Dafür schreibt man
    \[f(x)=\langle\textit{Regel, mit der aus $x$ das Bild von $x$ berechnet werden kann}\rangle.\]
\end{remark}

\subsection*{Aufgabenteil}
\begin{exercise}
    Die Abbildungen $f$ und $g$ haben beide die Definitionsmenge 
    $\textsc{Grundfarben}=\{\text{rot},\text{blau},\text{gelb}\}$. Die Abbildung 
    $f\colon\textsc{Grundbfarben}\rightarrow\textsc{Grundfarben}$ hat die Abbildungsvorschrift $f(x)=x$.
    Die Abbildungsvorschrift von $g\colon\textsc{Grundfarben}\rightarrow\Natural$ ist die Berechnungsvorschrift \[g(x)=\text{Anzahl der Buchstaben von $x$}.\] 
    
    \begin{enumerate}
        \item[a)] Beschreibe jeweils in Worten, was die Abbildungen $f$ und $g$ machen.
        \begin{answerbox}[1in]
            $f$ bildet jede Grundfarbe auf sich selbst ab, also rot auf rot, blau auf blau und gelb auf gelb.
            
            $g$ erhält als Argument eine Grundfarbe und zählt dann die Anzahl der Buchstaben, die die Farbe hat. Insgesamt bildet $g$ also jede Farbe auf die Zahl ab, die angibt, wie viele Buchstaben das Wort hat, das für die Farbe steht.
        \end{answerbox}
        \item[b)] Gib die Abbildungsvorschrift für $g$ mit der expliziten Schreibweise an.
        \begin{answerbox}[0.5in]
            $g(\text{blau})=4, g(\text{rot})=3, g(\text{gelb})=4$
        \end{answerbox}
    \end{enumerate}
\end{exercise}
\begin{exercise}
    Die linke Abbildung aus Aufgabe 2 mischt jede Grundfarbe mit blau. Gib die Abbildungsvorschrift dieser Abbildung als Berechnungsvorschrift an.
    
    \emph{Hinweis: Du kannst der Abbildung den Namen \textsc{MitBlauMischen} geben, wenn du einen Namen für die Abbildung benötigst.}
    \begin{answerbox}[0.5in]
        $\textsc{MitBlauMischen}(x)=\text{Farbe, die man erhält, wenn man}~x~\text{mit blau mischt}$
    \end{answerbox}
\end{exercise}
\begin{exercise}
    Eine Abbildung $f\colon\Natural\rightarrow\Natural$ ordnet jeder natürlichen Zahl das Ergebnis der Rechnung, wenn man die Zahl mit $2$ multipliziert, zu. Es ist also zum Beispiel \[f(0)=2\cdot 0=0, f(1)=2\cdot 1=2, f(2)=2\cdot 2=4, f(5)=2\cdot 5=10~\text{usw.}\]
    Gib die Abbildungsvorschrift von $f$ durch eine Berechnungsvorschrift an.
    \begin{answerbox}[0.5in]
        $f(x)=2x$
    \end{answerbox}
\end{exercise}

\mandala{mandala/mandala02}

\end{document}