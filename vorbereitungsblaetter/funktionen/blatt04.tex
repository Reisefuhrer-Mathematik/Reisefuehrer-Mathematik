\def\pathToMain{../../buch/}
\documentclass[]{uebungsblatt}
\usepackage[utf8]{inputenc}
\usepackage[T1]{fontenc}
\usepackage[ngerman]{babel}

\usepackage{mathdef}
\usepackage{tikzdef}
\usepackage{amsmath}
\usetikzlibrary{positioning}

\sheet{Vorbereitungsblatt 4}
\title{Abbildungen verketten}
\topic{\getchaptername{abbildungen}}
\chapternum{\getchapternum{abbildungen}}

\usepackage{multicol}

\begin{document}
\maketitle
\begin{contents}
    Dieses Vorbereitungsblatt führt die \textbf{Verkettung} von Abbildungen ein und stellt übliche Schreibweisen rund um dieses Thema vor.
\end{contents}

\video{Was sind Verkettungen?}{6}{Kapitel \ref{ext:sec:abbildungen_verkettung} (ab Seite \pageref{ext:sec:abbildungen_verkettung})}{https://www.google.de}
\begin{definition}
    \parpic[r]{
        \begin{tikzpicture}[scale=.55]
            \draw[grayset] (-1.5,0) ellipse (0.7cm and 2cm);
            \draw[grayset] (1.5,0) ellipse (0.7cm and 2cm);
            \draw[grayset] (4.5,0) ellipse (0.7cm and 2cm);
            % add elements
            \node (x1) at (-1.5,0.7) {$\bullet$};
            \node (x2) at (-1.5,-0.2) {$\bullet$};
            \node (x3) at (-1.5,-1.1) {$\bullet$};
            \node (y1) at (1.5,0.7) {$\bullet$};
            \node (y2) at (1.5,-0.2) {$\bullet$};
            \node (y3) at (1.5,-1.2) {$\bullet$};
            \node (z1) at (4.5,1.2) {$\bullet$};
            \node (z2) at (4.5,0.4) {$\bullet$};
            \node (z3) at (4.5,-0.4) {$\bullet$};
            \node (z4) at (4.5,-1.2) {$\bullet$};
            %draw f arrows
            \draw[->] (x1) -- (y3);
            \draw[->] (x2) to[bend right] (y1);
            \draw[->] (x3) to[bend right] (y2);
            % draw g arrows
            \draw[->] (y1) -- (z3);
            \draw[->] (y2) -- (z1);
            \draw[->] (y3) -- (z4);
            % labels f, g
            \node at (0,1) {$f$};
            \node at (3,1) {$g$};
        \end{tikzpicture}
    }
    
    Es seien $U,V,W$ Mengen und $f\colon U\rightarrow V$ sowie $g\colon V\rightarrow W$ Abbildungen. Die Abbildung $g\circ f\colon U\rightarrow W$ mit 
    \[x\mapsto (g\circ f)(x)\coloneqq g(f(x))\] 
    heißt die \textbf{Komposition} oder \textbf{Verkettung} von $f$ und $g$.
\end{definition}

\subsection*{Aufgabenteil}
\begin{exercise}
    Bei der Verkettung von zwei Abbildungen $f$ und $g$ nimmt man ein Element aus der \textchoice[0]{Definitionsmenge,Bildmenge} von \textchoice[0]{$f$,$g$} und setzt es in die Abbildung $f$ ein. Wenn sich das Bild $f(x)$ in der  \textchoice[0]{Definitionsmenge,Bildmenge} von \textchoice[0]{$f$,$g$} befindet, kann man es anschließend direkt in \textchoice[1]{$f$,$g$} einsetzen und erhält insgesamt $g(f(x))$. Eine andere Schreibweise dafür ist \textchoice[1]{$(f\circ g)(x)$,$(g\circ f)(x)$}.
\end{exercise}

\begin{exercise}
    Im folgenden Diagramm sind die Abbildungen $f$ und $g$ dargestellt. Zeichne Elemente in die beiden Mengen im rechten Bild ein und füge Pfeile hinzu, sodass dort die Abbildung $g\circ f$ dargestellt wird.
    \begin{multicols}{2}\centering
        \begin{tikzpicture}[scale=.8]
            \draw[grayset] (-1.5,0) ellipse (1.1cm and 2cm);
            \draw[grayset] (1.5,0) ellipse (0.7cm and 2cm);
            \draw[grayset] (4.5,0) ellipse (0.7cm and 2cm);
            % add elements
            \node[label={left:1}] (x1) at (-1.5,0.7) {$\bullet$};
            \node[label={left:2}] (x2) at (-1.5,-0.2) {$\bullet$};
            \node[label={left:3}] (x3) at (-1.5,-1.1) {$\bullet$};
            \node[blue] (y1) at (1.5,0.5) {$\bullet$};
            \node[green!70!black] (y2) at (1.5,-0.9) {$\bullet$};
            \node (z1) at (4.5,1.2) {$\heartsuit$};
            \node (z2) at (4.5,0.4) {$\Diamond$};
            \node (z3) at (4.5,-0.4) {$\clubsuit$};
            \node (z4) at (4.5,-1.2) {$\spadesuit$};
            %draw f arrows
            \draw[->] (x1) -- (y2);
            \draw[->] (x2) to[bend right] (y1);
            \draw[->] (x3) to[bend right] (y2);
            % draw g arrows
            \draw[->] (y1) -- (z3);
            \draw[->] (y2) -- (z4);
            % labels f, g
            \node at (0.1,1) {$f$};
            \node at (3,1) {$g$};
        \end{tikzpicture}
        
        \begin{tikzpicture}[scale=.8]
            \draw[grayset] (-2.3,0) ellipse (1.4cm and 2cm);
            \draw[grayset] (2.3,0) ellipse (1.4cm and 2cm);
            \node at (0,1.4) {$g\circ f$};
        \end{tikzpicture}
    \end{multicols}
\end{exercise}

\begin{exercise}
    Hättest du in der letzten Aufgabe auch eine Abbildung $f\circ g$ einzeichnen können?
    \begin{answerbox}[.5in]
        Nein, denn die Bildmenge von $g$ stimmt nicht mit der Definitionsmenge von $f$ überein. Dadurch lässt sich $f(g(x))$ nicht berechnen, weil $f$ an der Stelle $g(x)$ undefiniert ist.
    \end{answerbox}
\end{exercise}

\video{Abbildungsvorschriften verketten}{6}{Kapitel \ref{ext:sec:abbildungen_verkettung} (ab Seite \pageref{ext:sec:abbildungen_verkettung})}{https://www.google.de}

\subsection*{Aufgabenteil}
\begin{exercise}
    Wir betrachten die Abbildungen $f\colon\Real\rightarrow\Real$ und $g\colon\Real\rightarrow\Real$ mit $f(x)=x^2$ und $g(x)=x+1$.
    \begin{enumerate}
        \item[a)] Um $(g\circ f)(x)$ zu berechnen, muss $x$ zunächst in die Abbildung \textchoice[0]{$f$,$g$} und das Ergebnis anschließend in \textchoice[1]{$f$,$g$} eingesetzt werden.
        \item[b)] Kreuze die zutreffenden Aussagen an.
        \begin{multiplechoice}
            \citem $f(2)=4$ und $g(4)=5$. Also ist $(g\circ f)(2)=5$.
            \item $f(5)=25$ und $g(25)=26$. Also ist $(f\circ g)(5)=26$.
            \item $g(f(x))=(x+1)^2$
            \citem $g(f(x))=x^2+1$
        \end{multiplechoice}
        \item[c)] Berechne $(g\circ f)(4)$ und $(f\circ g)(4)$.
        \begin{answerbox}[.5in]
            $(g\circ f)(4)=g(f(4))=g(4^2)=g(16)=16+1=17$.\\
            $(f\circ g)(4)=f(g(4))=f(4+1)=f(5)=5^2=25$.
        \end{answerbox}
    \end{enumerate}
\end{exercise}

\mandala{mandala/mandala04}

\end{document}