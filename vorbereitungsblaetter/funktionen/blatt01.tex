\def\pathToMain{../../buch/}
\documentclass[]{uebungsblatt}
\usepackage[utf8]{inputenc}
\usepackage[T1]{fontenc}
\usepackage[ngerman]{babel}

\usepackage{mathdef}
\usepackage{tikzdef}

\sheet{Vorbereitungsblatt 11.1}
\title{Einführung: Funktionen}
\topic{\getchaptername{funktionen}}
\chapternum{\getchapternum{funktionen}}

\usepackage{multicol}

\begin{document}
\maketitle
\begin{contents}
    Dieses Vorbereitungsblatt behandelt die \textbf{Definition von Abbildungen} in der Mathematik und
    führt damit verbundene einfache \textbf{Schreibweisen} ein.
\end{contents}

\video{Intuition zu Abbildungen}{4}{Kapitel \ref{ext:sec:abbildungen_intuition} (ab Seite \pageref{ext:sec:abbildungen_intuition})}{https://www.google.de}

\subsection*{Aufgabenteil}
\begin{exercise}
    In der Fußballbundesliga steht jede Mannschaft auf einem bestimmten Tabellenplatz. Wir möchten nun jedem Verein zuordnen, auf welchem Tabellenplatz er steht.
        
    Wenn wir also gefragt werden, auf welchem Platz der FC Bayern München steht, dann antworten wir mit genau einem Tabellenplatz. Wir ordnen Fußballvereinen aus der \textchoice[0]{Definitionsmenge,Bildmenge,Abbildungsvorschrift} -- also Vereinen, die gerade in der Bundesliga spielen -- ihren Tabellenplatz zu. Dies ist eine Zahl zwischen 1 und 18. Die Menge dieser Zahlen ist für diese Abbildung deshalb die \textchoice[1]{Definitionsmenge,Bildmenge,Abbildungsvorschrift}.
\end{exercise}  

\begin{exercise}
    Drei Kinder machen eine Schneeballschlacht. Jedes Kind wirft einen Schneeball auf ein anderes Kind. Den Flug der Schneebälle können wir mit einer Abbildung beschreiben:
        
    \textchoice[2]{Kein Kind, Nur ein Kind, Jedes Kind} trifft mit seinem Schneeball \textchoice[1]{beliebig viele andere Kinder, genau ein anderes Kind}. Fragen wir nun ein Kind, wen es getroffen hat, dann können wir ihm anschließend das Kind zuordnen, das es getroffen hat.
        
    Wir ordnen hier also \textchoice[0]{jedem Kind, einem Kind} dasjenige Kind zu, das von ihm getroffen wurde. Hier entsteht deshalb eine Abbildung, die von den \textchoice[1]{Schneebällen, teilnehmenden Kindern, getroffenen Kindern} auf die \textchoice[2]{Schneebälle, teilnehmenden Kinder, getroffenen Kinder} abbildet.
\end{exercise}

\begin{exercise}
    Beschreibe ein Beispiel aus dem Alltag, das sich wie eine Abbildung verhält. Identifiziere dann die Definitions- und Bildmenge.
    \begin{answerbox}[.5in]
    \end{answerbox}
\end{exercise}

\video{Definition von Abbildungen}{4}{Kapitel \ref{ext:sec:abbildungen_definition} (ab Seite \pageref{ext:sec:abbildungen_definition})}{https://www.google.de}
\begin{definition}
    \parpic[r]{
        %\begin{wrapfigure}{r}{.3\textwidth}
    %\centering
    \begin{tikzpicture}[scale=.6]
        \draw[grayset] (-1.5,0) ellipse (0.7cm and 2cm);
        \draw[grayset] (1.5,0) ellipse (0.7cm and 2cm);
        %
        \node at (-1.5,1.5) {$U$};
        \node at (1.5,1.5) {$V$};
        \node at (0,0.2) {$f$};
        %
        \node (x1) at (-1.5,0.7) {$\bullet$};
        \node (x2) at (-1.5,-0.2) {$\bullet$};
        \node (x3) at (-1.5,-1.1) {$\bullet$};
        \node (y1) at (1.5,0.7) {$\bullet$};
        \node (y2) at (1.5,-0.2) {$\bullet$};
        \node (y3) at (1.5,-1.2) {$\bullet$};
        %
        \draw[->] (x1) -- (y3);
        \draw[->] (x2) to[bend right] (y1);
        \draw[->] (x3) to[bend right] (y2);
    \end{tikzpicture}
%\end{wrapfigure}
    }
    
    Eine \textbf{Abbildung} $f$ ist eine Vorschrift, die allen Elementen aus der Definitionsmenge $U$ genau ein Element aus der Bildmenge $V$ zuordnet, geschrieben $f\colon U\rightarrow V$. 
    
    Wird ein Element $u\in U$ auf $v\in V$ abgebildet, so schreibt man $f(u)=v$ bzw. $u\mapsto v$.
\end{definition}

\subsection*{Aufgabenteil}
\begin{exercise}
    Eine Abbildung bildet \textchoice[2]{ein bestimmtes, jedes, kein} Element einer \textchoice[1]{Menge, Zahl, Vorschrift} auf \textchoice[2]{bestimmte, genau ein, alle} \textchoice[1]{Element, Elemente} einer anderen \textchoice[1]{Menge, Zahl, Vorschrift} ab.
\end{exercise}

\begin{exercise}
    Zeichne jeweils eine Abbildung zwischen den dargestellten Mengen ein.
    \begin{multicols}{2}
        \centering
\begin{tikzpicture}[scale=0.75]
    \draw[grayset] (-1.5,0) ellipse (0.7cm and 2cm);
    \draw[grayset] (1.5,0) ellipse (0.7cm and 2cm);

    \node (x1) at (-1.5,0.75) {$\bullet$};
    \node (x2) at (-1.5,-0.75) {$\bullet$};
    \node (y1) at (1.5,1) {$\bullet$};
    \node (y2) at (1.5,0) {$\bullet$};
    \node (y3) at (1.5,-1) {$\bullet$};
\end{tikzpicture}
        
        \centering
\begin{tikzpicture}[scale=0.75]
    \draw[grayset] (-1.5,0) ellipse (0.7cm and 2cm);
    \draw[grayset] (1.5,0) ellipse (0.7cm and 2cm);

    \node (x1) at (-1.5,1.2) {$\bullet$};
    \node (x2) at (-1.5,0.4) {$\bullet$};
    \node (x3) at (-1.5,-0.4) {$\bullet$};
    \node (x4) at (-1.5,-1.2) {$\bullet$};
    \node (y1) at (1.5,1) {$\bullet$};
    \node (y2) at (1.5,0) {$\bullet$};
    \node (y3) at (1.5,-1) {$\bullet$};
\end{tikzpicture}
    \end{multicols}
\end{exercise}

\begin{exercise}
    Was sagt die Schreibweise $f\colon\Natural\rightarrow\Natural$ aus?
    \begin{answerbox}[0.6in]
        $f$ ist eine Abbildung von den natürlichen Zahlen in die natürliche Zahlen.
    \end{answerbox}
\end{exercise}

\begin{exercise}
    Stelle die Abbildung $f\colon\{1,2,3\}\rightarrow\{\clubsuit,\heartsuit\}$ mit $f(1)=\clubsuit, f(2)=\heartsuit, f(3)=\heartsuit$ als Diagramm dar.
    \begin{answerbox}[1in]
    \end{answerbox}
\end{exercise}

\mandala{mandala/mandala01}

\end{document}