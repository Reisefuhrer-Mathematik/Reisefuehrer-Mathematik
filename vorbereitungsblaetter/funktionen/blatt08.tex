\def\pathToMain{../../buch/}
\documentclass[]{uebungsblatt}
\usepackage[utf8]{inputenc}
\usepackage[T1]{fontenc}
\usepackage[ngerman]{babel}

\usepackage{mathdef}
\usepackage{tikzdef}
\usepackage{pgfplots}
\usetikzlibrary{positioning}

\sheet{Vorbereitungsblatt 8}
\title{Abbildungsgraphen verschieben}
\topic{\getchaptername{abbildungen}}
\chapternum{\getchapternum{abbildungen}}

\usepackage{multicol}
\usepackage{todonotes}

\begin{document}
\maketitle
\begin{contents}
    Auf diesem Vorbereitungsblatt geht es darum, wie sich \textbf{Abbildungsgraphen verschieben} lassen. Dabei wird thematisiert, wie sich die \textbf{Berechnungsvorschrift ändern} muss, damit sich der Graph nach oben, unten, links oder rechts verschiebt.
\end{contents}
\video{Abbildungegraphen nach oben und unten verschieben}{4}{Kapitel \ref{ext:sec:abbildungen_verschieben_y} (ab Seite \pageref{ext:sec:abbildungen_verschieben_y})}{https://www.google.de}
\begin{remark}{}
    Es sei $f$ eine Abbildung und $c>0$. Dann ist der Graph der Abbildung $g(x)=f(x)+c$ im Vergleich zum Graphen von $f$ um $c$ Einheiten nach oben und der Graph der Abbildung $g(x)=f(x)-c$ um $c$ Einheiten nach unten verschoben.
\end{remark}
\subsection*{Aufgabenteil}
\begin{exercise}
    \begin{enumerate}[label = \alph*)]
        \item Zeichne in die folgenden Koordinatensysteme jeweils einen Graphen ein, der im Vergleich zum bereits eingezeichneten Graphen eine Einheit nach oben verschoben ist.
        \begin{multicols}{2}\centering
            \begin{tikzpicture}
                \begin{axis}[defgrid, domain=-2:2, y=1cm, x=1cm, xtick={-4,...,4}, ytick={-2,...,2},xmin=-2,xmax=2,ymin=-2,ymax=2, samples=40]
                        \addplot[color=violet] expression{x^2*(x-1)};
                \end{axis}
            \end{tikzpicture}
        
        Graph der Funktion $f$
        
        \begin{tikzpicture}
            \begin{axis}[defgrid, domain=-2:2, y=1cm, x=1cm, xtick={-4,...,4}, ytick={-2,...,2},xmin=-2,xmax=2,ymin=-2,ymax=2, samples=40]
                    \addplot[color=violet] expression{0.2*(x+2)*(x+0.5)*(x-1)*(x-3)};
            \end{axis}
        \end{tikzpicture}

        Graph der Funktion $h$
    \end{multicols}

    
        \item Die Abbildung $g$, deren Graph wie der von dir eingezeichnete aussieht, muss die Berechnungsvorschrift $g(x)=$\textchoice[2]{$f(x+1)$,$f(x-1)$,$f(x)+1$,$f(x+c)+1$} haben.
        \item Es gilt $f(x+c)=$\textchoice[2]{$x^2\cdot (x-1)+c$,$x^2\cdot (x+c)$,$(x+c)^2\cdot (x+c-1)$} und $f(x)+c=$\textchoice[0]{$x^2\cdot (x-1)+c$,$x^2\cdot (x+c)$,$(x+c)^2\cdot (x+c-1)$}.
        \item Es gilt deshalb $g(x)=$\answerfield{3cm}{$x^2\cdot (x-1)+1$}.
    \end{enumerate}
\end{exercise}
\begin{exercise}
    Der Graph einer Abbildung $f(x)$ wird um 4 Einheiten nach \textchoice{oben,unten,links,rechts} verschoben, wenn man stattdessen die Berechnungsvorschrift $f(x)-4$ verwendet.
\end{exercise}
\video{Abbildungegraphen nach links und rechts verschieben}{4}{Kapitel \ref{ext:sec:abbildungen_verschieben_x} (ab Seite \pageref{ext:sec:abbildungen_verschieben_x})}{https://www.google.de}
\begin{remark}{}
    Es sei $f$ eine Abbildung und $c>0$. Dann ist der Graph der Abbildung $g(x)=f(x+c)$ im Vergleich zum Graphen von $f$ um $c$ Einheiten nach links und der Graph der Abbildung $g(x)=f(x-c)$ um $c$ Einheiten nach rechts verschoben.
\end{remark}
\subsection*{Aufgabenteil}

\newcommand{\startingGraph}{
    \begin{tikzpicture}
        \begin{axis}[defgrid, domain=-2:2, y=1cm, x=1cm, xtick={-4,...,4}, ytick={-2,...,2},xmin=-2,xmax=2,ymin=-2,ymax=2, samples=40]
                \addplot[color=violet] expression{x^3-x};
        \end{axis}
    \end{tikzpicture}
}

\begin{exercise}
    In den folgenden zwei Koordinatensystemen findest du jeweils den Graphen, den die Funktion \mbox{$f(x)=x^3-x$} beschreibt.
    \begin{multicols}{2}\centering
        \startingGraph
        
        \startingGraph
    \end{multicols}
    \begin{enumerate}
        \item[a)] Die Abbildung $g(x)=x^3-x+1$ hat eine sehr ähnliche Berechnungsvorschrift wie $f$. Zu $f(x)$ wird jetzt zusätzlich noch 1 addiert, also ist $g(x)=f(x)+1$. 
        
        Zeichne in das linke Koordinatensystem die Abbildung $g$ ein. Berechne dafür $g(-1),g(-\frac{1}{2}),g(\frac{1}{2})$ und $g(1)$, bis du siehst, wie sich der Graph verändert. Was fällt dir auf?
        \begin{answerbox}[.5in]
        \end{answerbox}
        \item[b)] Die Abbildung $h(x)=(x+1)^3-(x+1)$ hat ebenfalls eine ähnliche Berechnungsvorschrift wie $f$. Überall, wo vorher $x$ stand, steht jetzt $x+1$. Man berechnet hier also $f(x+1)$.
        
        Zeichne den Graphen von $h$ auf die gleiche Art wie in a) in das rechte Koordinatensystem. Wie hat sich der Graph verändert?
        \begin{answerbox}[.5in]
        \end{answerbox}
    \end{enumerate}
\end{exercise}
\mandala{mandala/mandala08}

\todo{Work in Progress: Unzufrieden mit den Aufgaben}

\end{document}