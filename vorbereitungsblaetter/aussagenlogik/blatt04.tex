\def\pathToMain{../../buch/}
\documentclass{uebungsblatt}
\usepackage[utf8]{inputenc}
\usepackage[T1]{fontenc}
\usepackage[ngerman]{babel}

\usepackage{multicol}
\usepackage{mathdef}
\usepackage{booktabs}
\usepackage{multirow}
\usepackage{todonotes}

\sheet{Vorbereitungsblatt 9.4}
\title{Äquivalenzumformungen}
\topic{\getchaptername{aussagenlogik}}
\chapternum{\getchapternum{aussagenlogik}}

\begin{document}
\maketitle
\begin{contents}
    Semantische Äquivalenz, Äquivalenzumformungen, Involution, Kontraposition, De Morgansche Gesetze, Auflösen von Implikation und Äquivalenz
\end{contents}

\video{Semantisch äquivalente Aussagen}{4}{Kapitel \ref{ext:sec:abbildungen_intuition} (ab Seite \pageref{ext:sec:abbildungen_intuition})}{https://www.google.de}

\begin{definition}
    Zwei Aussagen $A,B$ heißen \textbf{semantisch äquivalent}, wenn diese für alle möglichen Wahrheitswerte ihrer
        atomaren Unteraussagen den gleichen Wahrheitswert haben. Wir notieren dies durch $A \equiv B$.
\end{definition}

\begin{theorem}
    Sind $A$ und $B$ Aussagen, dann gilt:
    \begin{align*}
        \tag{Involution}
        \lnot \lnot A \equiv& A\\
        \tag{1. De Morgansches Gesetz}
        \lnot (A \land B) \equiv& (\lnot A) \lor (\lnot B)\\
        \tag{2. De Morgansches Gesetz}
        \lnot (A \lor B) \equiv& (\lnot A) \land (\lnot B)\\
        \tag{Kontraposition}
        A \implies B \equiv& (\lnot B) \implies (\lnot A)\\
        \tag{Auflösen der Implikation}
        A \implies B \equiv & \lnot (A \land (\lnot B))\\
        \tag{Auflösen der Äquivalenz}
        A \iff B \equiv & (A \implies B) \land (B \implies A)
    \end{align*}
\end{theorem}

\subsection*{Aufgabenteil}

\begin{exercise}
    Zwei Aussagen $A,B$ heißen \textchoice{syntaktisch,semantisch} äquivalent, 
    wenn diese für
    \textchoice{alle,einige,keine} Wahrheitswerte ihrer atomaren 
    \textchoice{Überaussagen,Unteraussagen} 
    \textchoice{den gleichen,einen unterschiedlichen} 
    Wahrheitswert haben.
\end{exercise}

\begin{exercise}
    Wir haben zwei atomare Aussagen $X,Y$. Zeige mithilfe einer Wahrheitstabelle folgende semantische Äquivalenz: \[ (\lnot X) \implies Y \equiv X \lor Y\] 
    
    Ergänze dafür die unausgefüllten Spalten in folgender Tabelle:

    \begin{center}
    \begin{tabular}{ccccc}\toprule
        $X$ &$Y$ & $ \lnot X$ & $(\lnot X) \implies Y$ & $X \lor Y$\\\midrule
        falsch & falsch &  &  & \\ 
        falsch & wahr&  &  &   \\
        wahr & falsch &  &  &  \\
        wahr & wahr &  &  & \\
        \bottomrule
    \end{tabular}
    \end{center}

\end{exercise}

\begin{exercise}

    Gezeigt ist eine Folge von semantischen Äquivalenzumformungen. 
    Nenne die Namen der Äquivalenzen,
    die angewendet wurden. Auf die rot markierten Unteraussagen wurden die jeweiligen 
    Äquivalenzumformungen angewendet.
        \\
        \begin{center}
        \begin{tabular}{ccc}
            & $\lnot (A \land \color{red}(B \land \lnot C)\color{black})$& Regel: \answerfield{3cm}{9} \\
            $\equiv$ & \color{red} $\lnot (A \land (\lnot \lnot (B \land \lnot C)))$ \color{black} & Regel: \answerfield{3cm}{9} \\
            $\equiv$ & $A \implies (\color{red}\lnot (B \land (\lnot C))\color{black})$ & Regel: \answerfield{3cm}{9}\\ 
            $\equiv$ & $A \implies (B \implies C)$ &
        \end{tabular}
    \end{center}

\end{exercise}

\begin{exercise}
    Zeige mithilfe von Äquivalenzumformungen, dass die Aussagen $(\lnot A) \iff (\lnot B)$ und 
    $A \iff B$ semantisch äquivalent sind.
    \begin{answerbox}[1in]
    \end{answerbox}
\end{exercise}


\begin{exercise}
    Wir erinnern uns wieder an Tim und seine Haustiere:
    \begin{itemize}
        \item $W$ steht für die Aussage \enquote{Tim hat einen Wellensittich}
        \item $K$ steht für die Aussage \enquote{Tim hat eine Katze}
        \item $H$ steht für die Aussage \enquote{Tim hat einen Hund}
    \end{itemize}

    Wir hatten dann eine Wahrheitstabelle für die Aussagen $W \land (K \lor H)$ und $(W \land K) \lor H$
    aufgestellt. Ergänze hier zunächst die fehlenden Einträge:

    \begin{center}
        \begin{tabular}{ccc cc}\toprule
            $W$ &$K$ & $H$ & $W \land (K \lor H)$ & $(W \land K) \lor H$\\\midrule
            falsch & falsch & falsch & falsch & \\ 
            falsch & falsch & wahr & falsch &  \\
            falsch & wahr & falsch &  & falsch\\
            falsch & wahr & wahr &  & wahr\\
            wahr & falsch & falsch & falsch &  falsch \\
            wahr & falsch & wahr & wahr & \\
            wahr & wahr & falsch &  & wahr\\
            wahr & wahr & wahr & wahr &  \\\bottomrule
        \end{tabular}
    \end{center}

    Entscheide nun mithilfe der Wahrheitstabelle, ob die Aussagen $W \land (K \lor H)$ und $(W \land K) \lor H$ semantisch
    äquivalent sind.

    \begin{answerbox}[1in]
    \end{answerbox}
\end{exercise}

\mandala{mandala/mandala01}

\end{document}
