\def\pathToMain{../../buch/}
\documentclass{uebungsblatt}
\usepackage[utf8]{inputenc}
\usepackage[T1]{fontenc}
\usepackage[ngerman]{babel}

\usepackage{mathdef}
\usepackage{multicol}

\sheet{Vorbereitungsblatt 9.1}
\title{Einführung: Aussagenlogik}
\topic{\getchaptername{aussagenlogik}}
\chapternum{\getchapternum{aussagenlogik}}

\begin{document}
\maketitle
\begin{contents}
    Aussagen, Wahrheitswerte
\end{contents}

\video{Was ist eine Aussage?}{4}{Kapitel \ref{ext:sec:abbildungen_intuition} (ab Seite \pageref{ext:sec:abbildungen_intuition})}{https://www.google.de}

\begin{definition}
    Eine Aussage ist ein Satz, der keine Frage ist, den man aber entweder mit 
    \enquote{Ja, das ist \emph{wahr}} bejahen oder mit \enquote{Nein, das ist \emph{falsch}} 
    verneinen kann.
\end{definition}

\begin{definition}
    Eine Aussage ist \emph{wahr}, falls die Antwortmöglichkeit \enquote{Ja, das ist \emph{wahr}}
    passend ist. Ist jedoch \enquote{Nein, das ist \emph{falsch}} die passende
    Antwortmöglichkeit, dann ist die Aussage \emph{falsch}.
\end{definition}

\subsection*{Aufgabenteil}

\begin{exercise}
    Eine tolle Aufgabenstellung.
    \begin{multicols}{2}
        \begin{enumerate}[label=\alph*)]
            \item Ich bin wahr.
                \begin{multiplechoice}
                    \item Wahr
                    \item Falsch
                \end{multiplechoice}
            \item Ich bin wahr.
                \begin{multiplechoice}
                    \item Wahr
                    \item Falsch
                \end{multiplechoice}
        \end{enumerate}
    \end{multicols}
\end{exercise}

\mandala{mandala/mandala01}

\end{document}
