\def\pathToMain{../../buch/}
\documentclass{uebungsblatt}
\usepackage[utf8]{inputenc}
\usepackage[T1]{fontenc}
\usepackage[ngerman]{babel}

\usepackage{mathdef}
\usepackage{booktabs}
\usepackage{multirow}

\sheet{Vorbereitungsblatt 9.3}
\title{Wahrheitstabellen}
\topic{\getchaptername{aussagenlogik}}
\chapternum{\getchapternum{aussagenlogik}}

\begin{document}
\def\wahr{\text{\color{green!50!black}wahr}}
\def\falsch{\text{\color{red!80!black}falsch}}

\maketitle
\begin{contents}
    Atomare Aussagen, Wahrheitstabellen, Wahrheitswerte von $\land, \lor, \implies, \iff, \lnot$
\end{contents}

\video{Wahrheitswerte aussagenlogischer Konnektoren}{4}{Kapitel \ref{ext:sec:abbildungen_intuition} (ab Seite \pageref{ext:sec:abbildungen_intuition})}{https://www.google.de}

\begin{definition}
    Eine \textbf{Atomare Aussage} ist eine Aussage, die keine Konnektoren enthält.
\end{definition}

\begin{definition}
    \label{whw}
    Seien $A,B$ zwei beliebige Aussagen. Der Wahrheitswerte der Aussagen $A \land B$, $A \lor B$, $A \implies B$, $A \iff B$ und $\lnot A$ sind dann wie folgt definiert.
    
    \[\begin{array}{cc s ccccc}\toprule
        A & B & A \land B & A\lor B & A\implies B & A\iff B & \lnot A\\\midrule
        \falsch & \falsch & \falsch & \falsch & \wahr & \wahr & \multirow{2}{*}{\wahr}\\
        \falsch & \wahr & \falsch & \wahr & \wahr & \falsch &  \\
            \wahr & \falsch & \falsch & \wahr & \falsch & \falsch & \multirow{2}{*}{\falsch}
        \\
        \wahr & \wahr & \wahr & \wahr & \wahr & \wahr & 
            \\\bottomrule
    \end{array}\]
\end{definition}

\subsection*{Aufgabenteil}

\newpage

\video{Wahrheitstabellen}{4}{Kapitel \ref{ext:sec:abbildungen_intuition} (ab Seite \pageref{ext:sec:abbildungen_intuition})}{https://www.google.de}

\begin{definition}
    Ist $A$ eine Aussage und kommen in $A$ die atomaren Unteraussagen $x_1,\dots,x_n$ vor, dann nennen wir eine Tabelle, die für alle möglichen Kombinationen der Wahrheitswerte von $x_1,\dots,x_n$ den Wahrheitswert von $A$ angibt, eine \textbf{Wahrheitstabelle} für $A$.
\end{definition}

\subsection*{Aufgabenteil}

\mandala{mandala/mandala01}

\end{document}
