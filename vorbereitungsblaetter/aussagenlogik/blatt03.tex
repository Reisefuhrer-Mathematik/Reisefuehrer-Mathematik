\def\pathToMain{../../buch/}
\documentclass{uebungsblatt}
\usepackage[utf8]{inputenc}
\usepackage[T1]{fontenc}
\usepackage[ngerman]{babel}

\usepackage{multicol}
\usepackage{mathdef}
\usepackage{booktabs}
\usepackage{multirow}
\usepackage{todonotes}

\sheet{Vorbereitungsblatt 9.3}
\title{Wahrheitstabellen}
\topic{\getchaptername{aussagenlogik}}
\chapternum{\getchapternum{aussagenlogik}}

\begin{document}
\def\wahr{\text{\color{green!50!black}wahr}}
\def\falsch{\text{\color{red!80!black}falsch}}

\maketitle
\begin{contents}
    Atomare Aussagen, Wahrheitstabellen, Wahrheitswerte von $\land, \lor, \implies, \iff, \lnot$
\end{contents}

\video{Wahrheitswerte aussagenlogischer Konnektoren}{4}{Kapitel \ref{ext:sec:abbildungen_intuition} (ab Seite \pageref{ext:sec:abbildungen_intuition})}{https://www.google.de}

\begin{definition}
    Eine \textbf{atomare Aussage} ist eine Aussage, die keine Konnektoren enthält.
\end{definition}

\begin{definition}
    \label{whw}
    Seien $A,B$ zwei beliebige Aussagen. Der Wahrheitswert der Aussagen $A \land B$, $A \lor B$, $A \implies B$, $A \iff B$ und $\lnot A$ ist dann wie folgt definiert.
    
    \[\begin{array}{cc s ccccc}\toprule
        A & B & A \land B & A\lor B & A\implies B & A\iff B & \lnot A\\\midrule
        \falsch & \falsch & \falsch & \falsch & \wahr & \wahr & \multirow{2}{*}{\wahr}\\
        \falsch & \wahr & \falsch & \wahr & \wahr & \falsch &  \\
            \wahr & \falsch & \falsch & \wahr & \falsch & \falsch & \multirow{2}{*}{\falsch}
        \\
        \wahr & \wahr & \wahr & \wahr & \wahr & \wahr & 
            \\\bottomrule
    \end{array}\]
\end{definition}

\subsection*{Aufgabenteil}

\begin{exercise}

    Kreuze an, welche Aussagen atomar sind. 

    $A$ ist eine Abkürzung für \enquote{Es regnet} und $B$ für \enquote{Die Sonne scheint nicht}.

    \begin{multicols}{2}
        \begin{multiplechoice}
            \item $B$
            \citem Schildkröten sind die schnellsten Tiere.
            \citem In Wäldern gibt es Bäume.
            \item Es schneit oder es schneit nicht.
            \citem $A$
            \item $A \implies B$
        \end{multiplechoice}
    \end{multicols}

\end{exercise}

\begin{exercise}

    Liste alle atomaren Aussagen der folgenden Aussage auf.
    \\ 

    \enquote{Delfine sind Säugetiere und Delfine leben nicht an Land}

    \begin{answerbox}[0.5in]
    \end{answerbox}

\end{exercise}

\todo{spacing}
\begin{exercise}

    Wir gehen davon aus, dass die atomare Aussage \enquote{Tobias ist hungrig} (abgekürzt: $H$)
    \emph{wahr} ist und die atomare Aussage \enquote{Tobias hat gefrühstückt} (abgekürzt: $F$)
    \emph{falsch} ist.
    \\

    Gib den Wahrheitswert dieser Aussagen an.
    
    \begin{center}
    \enquote{Wenn Tobias gefrühstückt hat, dann ist Tobias nicht hungrig}
    \end{center}

    
    \begin{multicols}{2}
        \begin{multiplechoice}
            \citem Wahr
            \item Falsch
        \end{multiplechoice}
    \end{multicols}

    \begin{center}
        $H \land \lnot F$
    \end{center}
    
        
        \begin{multicols}{2}
            \begin{multiplechoice}
                \citem Wahr
                \item Falsch
            \end{multiplechoice}
        \end{multicols}

\end{exercise}

\newpage

\video{Wahrheitstabellen}{4}{Kapitel \ref{ext:sec:abbildungen_intuition} (ab Seite \pageref{ext:sec:abbildungen_intuition})}{https://www.google.de}

\begin{definition}
    Ist $A$ eine Aussage und kommen in $A$ die atomaren Unteraussagen $x_1,\dots,x_n$ vor, dann nennen wir eine Tabelle, die für alle möglichen Kombinationen der Wahrheitswerte von $x_1,\dots,x_n$ den Wahrheitswert von $A$ angibt, eine \textbf{Wahrheitstabelle} für $A$.
\end{definition}

\subsection*{Aufgabenteil}

\begin{exercise}
    Gib Wahrheitswerte der atomaren Aussagen $A,B$ an, sodass folgende Aussage \emph{wahr} wird
    und weitere Wahrheitswerte, sodass die Aussage \emph{falsch} wird.

    \begin{center}
     $\lnot (A \lor \lnot B)$
    \end{center}
    \[
        \textrm{Wahrheitswerte für \emph{falsch}:}\answerfield{2cm}{9},
        \textrm{Wahrheitswerte für \emph{wahr}:}\answerfield{2cm}{12},
    \]
\end{exercise}


\begin{exercise}

    Dies ist die Wahrheitstabelle zu der Aussage $\lnot (A \iff B)$ ($A,B$ sind atomar).

    \begin{center}
        \begin{tabular}{cc c}\toprule
            $A$ & $B$ & $\lnot (A \iff B)$\\\midrule
            falsch & falsch & falsch\\
            falsch & wahr & wahr \\
            wahr & falsch & wahr\\
            wahr & wahr & falsch \\\bottomrule
        \end{tabular}
    \end{center}

    Lies aus der Wahrheitstabelle ab, welchen Wahrheitswert 
    die Aussage $\lnot (A \iff B)$ hat, wenn $A$ \emph{falsch} ist und $B$ \emph{wahr}
    ist.
    \begin{center}
    Wahrheitswert = \answerfield{2cm}{9}
    \end{center}

\end{exercise}

\todo{zwei parallele tabellen drausmachen?}
\begin{exercise}
    Wir erinnern uns an Tim und seine Haustiere. Wir kürzen jetzt
    \enquote{Tim hat einen Wellensittich} durch $W$, \enquote{Tim hat eine Katze} durch $K$ und
    \enquote{Tim hat einen Hund} durch $H$ ab.

    Vervollständige die Wahrheitstabelle.

    \begin{center}
        \begin{tabular}{ccc cc}\toprule
            $W$ &$K$ & $H$ & $W \land (K \lor H)$ & $(W \land K) \lor H$\\\midrule
            falsch & falsch & falsch & falsch & falsch\\
            falsch & falsch & wahr \\
            falsch & wahr & falsch & falsch & falsch\\
            falsch & wahr & wahr\\
            wahr & falsch & falsch\\
            wahr & falsch & wahr & wahr & wahr\\
            wahr & wahr & falsch\\
            wahr & wahr & wahr & wahr & wahr\\\bottomrule
        \end{tabular}
    \end{center}


\end{exercise}

\mandala{mandala/mandala01}

\end{document}
