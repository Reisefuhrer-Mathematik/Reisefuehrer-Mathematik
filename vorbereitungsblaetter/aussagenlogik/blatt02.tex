\def\pathToMain{../../buch/}
\documentclass{uebungsblatt}
\usepackage[utf8]{inputenc}
\usepackage[T1]{fontenc}
\usepackage[ngerman]{babel}

\usepackage{mathdef}
\usepackage{booktabs}

\sheet{Vorbereitungsblatt 9.2}
\title{Aussagen in der Mathematik}
\topic{\getchaptername{aussagenlogik}}
\chapternum{\getchapternum{aussagenlogik}}

\begin{document}
\maketitle
\begin{contents}
    Zusammengesetzte Aussagen, Konnektoren, Unteraussagen, logisches \emph{und}, logisches \emph{oder}, Negation, Implikation, Äquivalenz
\end{contents}

\video{Zusammengesetzte Aussagen}{4}{Kapitel \ref{ext:sec:abbildungen_intuition} (ab Seite \pageref{ext:sec:abbildungen_intuition})}{https://www.google.de}

\begin{definition}
    Ein Wort oder eine Kombination von mehreren Wörtern, die Aussagen miteinander zu einer neuen Aussage verknüpfen, nennt man einen \textbf{Konnektor}.
\end{definition}

\begin{definition}
    Eine Aussage, die Teil einer Aussage ist, nennen wir \textbf{Unteraussage}.
\end{definition}

\subsection*{Aufgabenteil}

\newpage

\video{Aussagen formalisieren}{4}{Kapitel \ref{ext:sec:abbildungen_intuition} (ab Seite \pageref{ext:sec:abbildungen_intuition})}{https://www.google.de}

\begin{definition}
    Die folgende Tabelle definiert, welche Konnektoren durch welches Symbol repräsentiert werden.
    \begin{center}
        \begin{tabular}{cc}\toprule
                        Konnektor & Symbol\\\midrule
                    und &  $\land$\\
                        oder&  $\lor$ \\
                        nicht & $\lnot$ \\
                        wenn, dann& $\implies$\\
                        genau dann, wenn&  $\iff$ \\
                        \bottomrule
        \end{tabular}
    \end{center}
\end{definition}

\subsection*{Aufgabenteil}

\mandala{mandala/mandala01}

\end{document}
