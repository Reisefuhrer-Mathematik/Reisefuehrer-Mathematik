\def\pathToMain{../../buch/}
\documentclass{uebungsblatt}
\usepackage[utf8]{inputenc}
\usepackage[T1]{fontenc}
\usepackage[ngerman]{babel}

\usepackage{multicol}
\usepackage{mathdef}
\usepackage{booktabs}

\sheet{Vorbereitungsblatt 9.2}
\title{Aussagen in der Mathematik}
\topic{\getchaptername{aussagenlogik}}
\chapternum{\getchapternum{aussagenlogik}}

\begin{document}
\maketitle
\begin{contents}
    Zusammengesetzte Aussagen, Konnektoren, Unteraussagen, logisches \emph{und}, logisches \emph{oder}, Negation, Implikation, Äquivalenz
\end{contents}

\video{Zusammengesetzte Aussagen}{4}{Kapitel \ref{ext:sec:abbildungen_intuition} (ab Seite \pageref{ext:sec:abbildungen_intuition})}{https://www.google.de}

\begin{definition}
    Ein Wort oder eine Kombination von mehreren Wörtern, die Aussagen miteinander zu einer neuen Aussage verknüpfen, nennt man einen \textbf{Konnektor}.
\end{definition}

\begin{definition}
    Alle Aussagen, die in einer Aussage $A$ vorkommen, nennen wir \textbf{Unteraussagen} von $A$.
    \end{definition}

\subsection*{Aufgabenteil}

\begin{exercise}

    Unterstreiche alle Konnektoren in den folgenden Aussagen.

        \begin{enumerate}[label=\alph*)]
            \item Wenn ich eine 1 schreibe, dann erhalte ich 5 Euro.
            \item Manner sind Menschen und Frauen sind Menschen.
            \item Ich gehe nicht zum Fußballtraining.
            \item Pizza schmeckt genau dann, wenn die Pizza nicht mit Ananas belegt ist.
        \end{enumerate}

\end{exercise}

\begin{exercise}

    Liste alle Unteraussagen der folgenden Aussage auf.

    \begin{center}
        \emph{Wenn es mir nicht gut geht, dann bleib ich zuhause}
    \end{center}

    \begin{answerbox}[1in]

    \end{answerbox}

\end{exercise}

\begin{exercise}

    Wähle für die zwei Aussagen ein Symbol und verknüpfe diese Symbole mit einem Konnektor, sodass die verknüpfte Aussage \emph{wahr} wird.

    \begin{enumerate}
        \item Es regnet.
        \item Die Straße wird nass.
    \end{enumerate}

    \begin{answerbox}[1in]

    \end{answerbox}

\end{exercise}

\newpage

\video{Aussagen formalisieren}{4}{Kapitel \ref{ext:sec:abbildungen_intuition} (ab Seite \pageref{ext:sec:abbildungen_intuition})}{https://www.google.de}

\begin{definition}
    Die folgende Tabelle definiert, welche Konnektoren durch welches Symbol repräsentiert werden.
    \begin{center}
        \begin{tabular}{cc}\toprule
                        Konnektor & Symbol\\\midrule
                    und &  $\land$\\
                        oder&  $\lor$ \\
                        nicht & $\lnot$ \\
                        wenn, dann& $\implies$\\
                        genau dann, wenn&  $\iff$ \\
                        \bottomrule
        \end{tabular}
    \end{center}
\end{definition}

\subsection*{Aufgabenteil}

\begin{exercise}

    Wir kürzen folgende drei Aussagen mit diesen Bildern ab.

    \begin{enumerate}
        \item Der Zauberer verwandelt dich in ein Kaninchen BILD
        \item Du siehst eine Katze BILD
        \item Du läufst weg BILD
    \end{enumerate}

    a) Ersetze in den folgenden Aussagen, den Konnektor durch sein Symbol
    \\ \\
        wenn WEGLAUFEN dan nicht VERWANDELT \ \ \ \ \ \ \ \ \ \ \ \  \ \ \
         \ \ \  \ \ \ \ \ \   WEGLAUFEN oder KANINCHEN  
    

    \begin{answerbox}[1in]
    \end{answerbox}


    b) Formuliere folgende Aussage natürlichsprachlich.

     VERWANDELT $\implies$ (katze $\implies$ weglaufen)

    \begin{answerbox}[1in]
    \end{answerbox}

\end{exercise}

\begin{exercise}

    \emph{Du weißt, dass dein Kumpel Tim einen Wellensittich hat und du weißt, dass 
    Tim noch ein weiteres Haustier besitzt. Du bist dir aber nicht sicher was für
    eins. Du weißt nur, dass es entweder ein Hund oder eine Katze ist.}

    Welche der zwei Aussagen ist \emph{wahr} bezügliches des Textes?

    \begin{multicols}{2}
        \begin{multiplechoice}
            \citem formalisiert: Wellsensittich und (Katze oder Hund)
            \item formalisiert: (Wellsensittich und Katze) oder Hund
            \item formalisiert: (Wellensittich und Hund) oder Katze
            \citem formalisiert: Wellsensittich und (Hund oder Katze)
        \end{multiplechoice}
    \end{multicols}

\end{exercise}

\mandala{mandala/mandala01}

\end{document}
