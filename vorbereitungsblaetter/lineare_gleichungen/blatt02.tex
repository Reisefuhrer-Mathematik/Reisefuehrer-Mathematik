\def\pathToMain{../../buch/}
\documentclass[solution]{uebungsblatt}
\usepackage[utf8]{inputenc}
\usepackage[T1]{fontenc}
\usepackage[ngerman]{babel}

\usepackage{mathdef}
\usepackage{multicol}

\sheet{Vorbereitungsblatt 15.2}
\title{Äquivalenzumformungen}
\topic{\getchaptername{lineare_gleichungen}}
\chapternum{\getchapternum{lineare_gleichungen}}

\begin{document}
\maketitle
\begin{contents}
    Lösungsmengen, Äquivalente Gleichungen, Äquivalenzumformungen
\end{contents}

\video{Die Lösungsmenge einer Gleichung}{4}{Kapitel \ref{ext:sec:abbildungen_intuition} (ab Seite \pageref{ext:sec:abbildungen_intuition})}{https://www.google.de}

\begin{definition}
    Die \textbf{Lösungsmenge} einer Gleichung $G$ ist die Menge 
    \[\Solutions:=\{x\:|\:x~\text{erfüllt die Gleichung~}G\},\]
    also die Menge aller Werte, die für die vorkommenden Unbekannten eingesetzt werden können, sodass die Gleichung erfüllt ist.
\end{definition}

\subsection*{Aufgabenteil}

\newpage

\video{Äquivalenzumformungen}{4}{Kapitel \ref{ext:sec:abbildungen_intuition} (ab Seite \pageref{ext:sec:abbildungen_intuition})}{https://www.google.de}

\begin{theorem}
    Die Lösungsmenge einer Gleichung bleibt unverändert, wenn man
    \begin{itemize}
        \item zu jeder Seite eine Zahl $a\in\mathbb{R}$ addiert.
        \item jede Seite mit einer Zahl $a\neq 0$ multipliziert.
    \end{itemize}
\end{theorem}

\subsection*{Aufgabenteil}

\begin{exercise}{}
    Welches der Zeichen $\Rightarrow, \Leftarrow$ und $\Leftrightarrow$ gehört jeweils in die Mitte?
    \begin{multicols}{2}
        \begin{enumerate}
            \item[a)]
            \begin{multiplechoice}
                \item $x=4 \Rightarrow x+3=7$
                \item $x=4 \Leftarrow x+3=7$
                \citem $x=4 \Leftrightarrow x+3=7$
            \end{multiplechoice}
            \item[b)]
            \begin{multiplechoice}
                \citem $x=0 \Rightarrow x\cdot 0=0$
                \item $x=0 \Leftarrow x\cdot 0=0$
                \item $x=0 \Leftrightarrow x\cdot 0=0$
            \end{multiplechoice}
            \item[c)]
            \begin{multiplechoice}
                \item $x^2=16 \Rightarrow x=4$
                \citem $x^2=16 \Leftarrow x=4$
                \item $x^2=16 \Leftrightarrow x=4$
            \end{multiplechoice}
            \item[d)]
            \begin{multiplechoice}
                \citem $x=5 \Rightarrow |x|=5$
                \item $x=5 \Leftarrow |x|=5$
                \item $x=5 \Leftrightarrow |x|=5$
            \end{multiplechoice}
        \end{enumerate}
    \end{multicols}
\end{exercise}
\if 0
\begin{exercise}
    Sind die beiden Gleichungen äquivalent? Falls ja, gib die verwendete Äquivalenzumformung an.
    \begin{multicols}{3}
        \begin{align*}
            3x+7&=16\\
            3x&=9
        \end{align*}
        \begin{multiplechoice}
            \citem Ja, Umformung: \answerfield{1cm}{$-7$}
            \item Nein
        \end{multiplechoice}

        \begin{align*}
            3x+7&=16\\
            3x&=9
        \end{align*}
        \begin{multiplechoice}
            \item Ja, Umformung: \answerfield{1cm}{}
            \item Nein
        \end{multiplechoice}

        \begin{align*}
            3x+7&=16\\
            3x&=9
        \end{align*}
        \begin{multiplechoice}
            \item Ja, Umformung: \answerfield{1cm}{}
            \item Nein
        \end{multiplechoice}
    \end{multicols}
\end{exercise}
\fi
\mandala{mandala/mandala01}

\end{document}
