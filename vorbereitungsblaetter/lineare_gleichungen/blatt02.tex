\def\pathToMain{../../buch/}
\documentclass[]{uebungsblatt}
\usepackage[utf8]{inputenc}
\usepackage[T1]{fontenc}
\usepackage[ngerman]{babel}

\usepackage{mathdef}
\usepackage{tikzdef}
\usepackage{multicol}

\sheet{Vorbereitungsblatt 15.2}
\title{Äquivalenzumformungen}
\topic{\getchaptername{lineare_gleichungen}}
\chapternum{\getchapternum{lineare_gleichungen}}

\begin{document}
\maketitle
\begin{contents}
    Lösungsmengen, Äquivalente Gleichungen, Äquivalenzumformungen
\end{contents}

\video{Die Lösungsmenge einer Gleichung}{4}{Kapitel \ref{ext:sec:abbildungen_intuition} (ab Seite \pageref{ext:sec:abbildungen_intuition})}{https://www.google.de}

\begin{definition}
    Die \textbf{Lösungsmenge} einer Gleichung $G$ ist die Menge 
    \[\Solutions:=\{x\:|\:x~\text{erfüllt die Gleichung~}G\},\]
    also die Menge aller Werte, die für die vorkommenden Unbekannten eingesetzt werden können, sodass die Gleichung erfüllt ist.
\end{definition}
%
\subsection*{Aufgabenteil}
%
\begin{exercise}
    Die Lösungsmenge der Gleichung $x+4=10-x$ enthält \textchoice[0]{alle Werte,einen Wert}, durch \textchoice[0]{die,den} wir das $x$ überall ersetzen können, sodass die Gleichung stimmt. Wenn wir für $x$ den Wert \answerfield{1cm}{3} einsetzen, erhalten wir $3+4=10-3$. Diese Gleichung ist \textchoice[0]{erfüllt,nicht erfüllt}, also ist \textchoice[0]{$3$,$x$} eine Lösung der Gleichung $x+3=10-x$. Alle anderen Werte, die wir für $x$ einsetzen könnten, erfüllen die Gleichung nicht. Die Lösungsmenge der Gleichung ist also $\Solutions=\{\answerfield{1cm}{3}\}$.
\end{exercise}
\begin{exercise}
    Wenn wir sagen, dass zwei Gleichungen \emph{äquivalent} sind, dann meinen wir damit, dass sie \textchoice[1]{eine beliebige,die gleiche,keine} Lösungsmenge haben. Wenn wir einen Wert für $x$ in eine der beiden Gleichungen einsetzen und die Gleichung dann erfüllt ist, dann erfüllt \textchoice[3]{kein,mindestens ein,ein anderer,der gleiche} Wert für $x$ auch die andere Gleichung.
\end{exercise}
\begin{exercise}
    Kreuze jeweils an, welche Aussagen über die Lösungsmenge der gegebenen Gleichungen stimmen.
    \begin{multicols}{4}
        \begin{enumerate}[label=\alph*)]
        \item $2x+5=9$
        \begin{multiplechoice}
            \citem $2\in\Solutions$
            \item $4\in\Solutions$
            \item $6\in\Solutions$
        \end{multiplechoice}

        \item $2x+3x=5x$
        \begin{multiplechoice}
            \citem $3\in\Solutions$
            \citem $7\in\Solutions$
            \item $\Solutions=\{7\}$
        \end{multiplechoice}

        \item $x^2=4$
        \begin{multiplechoice}
            \citem $2\in\Solutions$
            \citem $-2\in\Solutions$
            \item $\Solutions=\{2\}$
        \end{multiplechoice}

        \item $2x+4=1+2x$
        \begin{multiplechoice}
            \citem $0\in\Solutions$
            \item $\Solutions=\{0\}$
            \citem $\Solutions=\emptyset$
        \end{multiplechoice}
    \end{enumerate}
    \end{multicols}
\end{exercise}
\begin{exercise}
    \begin{enumerate}[label=\alph*)]
        \item Bestimme die Lösungsmenge der Gleichung $11-x=9$.\\
            \emph{Lösungsmenge:} $\Solutions=$\answerfield{1cm}{$\{3\}$}
        \item Zu welcher Gleichung aus der letzten Aufgabe ist diese Gleichung äquivalent?
        \begin{multicols}{4}
            \begin{multiplechoice}
                \item Gleichung a)
                \item Gleichung b)
                \item Gleichung c)
                \item Gleichung d)
            \end{multiplechoice}
        \end{multicols}
    \end{enumerate}
\end{exercise}

\newpage

\video{Äquivalenzumformungen}{4}{Kapitel \ref{ext:sec:abbildungen_intuition} (ab Seite \pageref{ext:sec:abbildungen_intuition})}{https://www.google.de}

\begin{theorem}
    Die Lösungsmenge einer Gleichung bleibt unverändert, wenn man
    \begin{itemize}
        \item zu jeder Seite eine Zahl $a\in\mathbb{R}$ addiert.
        \item jede Seite mit einer Zahl $a\neq 0$ multipliziert.
    \end{itemize}
\end{theorem}

\subsection*{Aufgabenteil}
\parpic[r]{
        \begin{linearEquation}
            %Füllung linke Waagschale
            \fill (-0.95,0.25) -- (-0.55,0.25) -- (-0.6,0.6) -- (-0.9,0.6) -- cycle;
            \draw[line width=0.75mm] (-0.75,0.66) circle[radius=0.06cm];
            \node[white] at (-0.75,0.42) {$8$};
            \node[white,marble,inner sep=.12cm] at (-1.05,0.35) {$x$};
            \node[white,marble,inner sep=.12cm] at (-1.35,0.35) {$x$};
            %Füllung rechte Waagschale
            \fill (0.75,0.25) -- (1.25,0.25) -- (1.2,0.65) -- (0.8,0.65) -- cycle;
            \draw[line width=0.75mm] (1,0.71) circle[radius=0.06cm];
            \node[white] at (1,0.45) {$14$};
        \end{linearEquation}
    }
\begin{exercise}
    \begin{minipage}{.6\textwidth}
        Die rechts dargestellte Balkenwaage befindet sich gerade im Gleichgewicht. 
        \begin{enumerate}
            \item[a)] Welche der folgenden Waagen befinden sich dann ebenfalls im Gleichgewicht und welche kippen nach links oder rechts?
            \item[b)] Begründe jeweils, warum die Waage im Gleichgewicht geblieben ist oder in eine Richtung gekippt ist.
        \end{enumerate}
    \end{minipage}
    \begin{multicols}{3}
        \begin{linearEquation}
            %Füllung linke Waagschale
            \fill (-0.95,0.25) -- (-0.55,0.25) -- (-0.6,0.6) -- (-0.9,0.6) -- cycle;
            \draw[line width=0.75mm] (-0.75,0.66) circle[radius=0.06cm];
            \node[white] at (-0.75,0.42) {$3$};
            \node[white,marble,inner sep=.12cm] at (-1.05,0.35) {$x$};
            \node[white,marble,inner sep=.12cm] at (-1.35,0.35) {$x$};
            %Füllung rechte Waagschale
            \fill (0.75,0.25) -- (1.25,0.25) -- (1.2,0.65) -- (0.8,0.65) -- cycle;
            \draw[line width=0.75mm] (1,0.71) circle[radius=0.06cm];
            \node[white] at (1,0.45) {$9$};
        \end{linearEquation}
        \begin{multiplechoice}
            \citem ist im Gleichgewicht
            \item kippt nach links
            \item kippt nach rechts
        \end{multiplechoice}
        \vspace{4mm}
        \emph{Begründung}:
        \begin{answerbox}[.75in]
            Auf beiden Seiten wurde das gleiche Gewicht entfernt.
        \end{answerbox}

        \begin{linearEquation}
            %Füllung linke Waagschale
            \node[white,marble,inner sep=.12cm] at (-1.15,0.35) {$x$};
            \node[white,marble,inner sep=.12cm] at (-0.85,0.35) {$x$};
            %Füllung rechte Waagschale
            \fill (0.75,0.25) -- (1.25,0.25) -- (1.2,0.65) -- (0.8,0.65) -- cycle;
            \draw[line width=0.75mm] (1,0.71) circle[radius=0.06cm];
            \node[white] at (1,0.45) {$14$};
        \end{linearEquation}
        \begin{multiplechoice}
            \item ist im Gleichgewicht
            \item kippt nach links
            \citem kippt nach rechts
        \end{multiplechoice}
        \vspace{4mm}
        \emph{Begründung}:
        \begin{answerbox}[.75in]
            Auf der linken Seite wurde ein Gewicht entfernt, auf der rechten aber nicht.
        \end{answerbox}

        \begin{linearEquation}
            %Füllung linke Waagschale
            \fill (-0.95,0.25) -- (-0.55,0.25) -- (-0.6,0.6) -- (-0.9,0.6) -- cycle;
            \draw[line width=0.75mm] (-0.75,0.66) circle[radius=0.06cm];
            \node[white] at (-0.75,0.42) {$4$};
            \node[white,marble,inner sep=.12cm] at (-1.2,0.35) {$x$};
            %Füllung rechte Waagschale
            \fill (0.75,0.25) -- (1.25,0.25) -- (1.2,0.65) -- (0.8,0.65) -- cycle;
            \draw[line width=0.75mm] (1,0.71) circle[radius=0.06cm];
            \node[white] at (1,0.45) {$7$};
        \end{linearEquation}
        \begin{multiplechoice}
            \citem ist im Gleichgewicht
            \item kippt nach links
            \item kippt nach rechts
        \end{multiplechoice}
        \vspace{4mm}
        \emph{Begründung}:
        \begin{answerbox}[.75in]
            Das Gewicht wurde auf beiden Seiten halbiert.
        \end{answerbox}
    \end{multicols}
\end{exercise}

\end{document}
