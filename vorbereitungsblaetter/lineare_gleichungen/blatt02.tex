\def\pathToMain{../../buch/}
\documentclass{uebungsblatt}
\usepackage[utf8]{inputenc}
\usepackage[T1]{fontenc}
\usepackage[ngerman]{babel}

\usepackage{mathdef}

\sheet{Vorbereitungsblatt 15.2}
\title{Äquivalenzumformungen}
\topic{\getchaptername{lineare_gleichungen}}
\chapternum{\getchapternum{lineare_gleichungen}}

\begin{document}
\maketitle
\begin{contents}
    Lösungsmengen, Äquivalente Gleichungen, Äquivalenzumformungen
\end{contents}

\video{Die Lösungsmenge einer Gleichung}{4}{Kapitel \ref{ext:sec:abbildungen_intuition} (ab Seite \pageref{ext:sec:abbildungen_intuition})}{https://www.google.de}

\begin{definition}
    Die \textbf{Lösungsmenge} einer Gleichung $G$ ist die Menge 
    \[\Solutions:=\{x\:|\:x~\text{erfüllt die Gleichung~}G\},\]
    also die Menge aller Werte, die für die vorkommenden Unbekannten eingesetzt werden können, sodass die Gleichung erfüllt ist.
\end{definition}

\subsection*{Aufgabenteil}

\newpage

\video{Äquivalenzumformungen}{4}{Kapitel \ref{ext:sec:abbildungen_intuition} (ab Seite \pageref{ext:sec:abbildungen_intuition})}{https://www.google.de}

\begin{theorem}
    Die Lösungsmenge einer Gleichung bleibt unverändert, wenn man
    \begin{itemize}
        \item zu jeder Seite eine Zahl $a\in\mathbb{R}$ addiert.
        \item jede Seite mit einer Zahl $a\neq 0$ multipliziert.
    \end{itemize}
\end{theorem}

\subsection*{Aufgabenteil}

\mandala{mandala/mandala01}

\end{document}
