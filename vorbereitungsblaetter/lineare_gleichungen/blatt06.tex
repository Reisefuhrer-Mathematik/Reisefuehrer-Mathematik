\def\pathToMain{../../buch/}
\documentclass{uebungsblatt}
\usepackage[utf8]{inputenc}
\usepackage[T1]{fontenc}
\usepackage[ngerman]{babel}

\usepackage{mathdef}

\sheet{Vorbereitungsblatt 15.6}
\title{Punkte auf Funktionsgeraden}
\topic{\getchaptername{lineare_gleichungen}}
\chapternum{\getchapternum{lineare_gleichungen}}

\begin{document}
\maketitle
\begin{contents}
    Nullstellen einer Funktionsgerade bestimmen, Funktionsgeraden durch vorgegebene Punkte
\end{contents}

\video{Punkte auf Funktionsgeraden ermitteln}{4}{Kapitel \ref{ext:sec:abbildungen_intuition} (ab Seite \pageref{ext:sec:abbildungen_intuition})}{https://www.google.de}

\begin{remark}
    Vlt auch kein remark. Ist echt trivial
\end{remark}

\subsection*{Aufgabenteil}

\newpage

\video{Funktionsgeraden durch vorgegebene Punkte}{4}{Kapitel \ref{ext:sec:abbildungen_intuition} (ab Seite \pageref{ext:sec:abbildungen_intuition})}{https://www.google.de}

\begin{remark}
    Zusammenfassen, was Phase ist
\end{remark}

\subsection*{Aufgabenteil}

\mandala{mandala/mandala01}

\end{document}
