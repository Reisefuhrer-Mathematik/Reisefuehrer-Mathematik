\def\pathToMain{../../buch/}
\documentclass{uebungsblatt}
\usepackage[utf8]{inputenc}
\usepackage[T1]{fontenc}
\usepackage[ngerman]{babel}

\usepackage{mathdef}

\sheet{Vorbereitungsblatt 15.3}
\title{Auflösen linearer Gleichungen}
\topic{\getchaptername{lineare_gleichungen}}
\chapternum{\getchapternum{lineare_gleichungen}}

\begin{document}
\maketitle
\begin{contents}
    Lineare Gleichungen, Verfahren zum Lösen von linearen Gleichungen
\end{contents}

\video{Was ist eine lineare Gleichung?}{4}{Kapitel \ref{ext:sec:abbildungen_intuition} (ab Seite \pageref{ext:sec:abbildungen_intuition})}{https://www.google.de}

\begin{definition}
    Eine \textbf{lineare Gleichung} ist eine Gleichung, deren Seiten durch Termumformung in die Form $ax+b$ mit $a,b\in\Real$ gebracht werden können.
\end{definition}

\subsection*{Aufgabenteil}

\newpage

\video{Lineare Gleichungen auflösem}{4}{Kapitel \ref{ext:sec:abbildungen_intuition} (ab Seite \pageref{ext:sec:abbildungen_intuition})}{https://www.google.de}

\begin{remark}
    Lösungsalgorithmus für lineare Gleichungen
\end{remark}

\subsection*{Aufgabenteil}

\mandala{mandala/mandala01}

\end{document}
