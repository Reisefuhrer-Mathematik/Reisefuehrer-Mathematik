\def\pathToMain{../../buch/}
\documentclass[solution]{uebungsblatt}
\usepackage[utf8]{inputenc}
\usepackage[T1]{fontenc}
\usepackage[ngerman]{babel}

\usepackage{mathdef}
\usepackage{multicol}

\sheet{Vorbereitungsblatt 15.3}
\title{Auflösen linearer Gleichungen}
\topic{\getchaptername{lineare_gleichungen}}
\chapternum{\getchapternum{lineare_gleichungen}}

\begin{document}
\maketitle
\begin{contents}
    Lineare Gleichungen, Verfahren zum Lösen von linearen Gleichungen
\end{contents}

\video{Was ist eine lineare Gleichung?}{4}{Kapitel \ref{ext:sec:abbildungen_intuition} (ab Seite \pageref{ext:sec:abbildungen_intuition})}{https://www.google.de}

\begin{definition}
    Eine \textbf{lineare Gleichung} ist eine Gleichung, deren Seiten durch Termumformung in die Form $ax+b$ mit $a,b\in\Real$ gebracht werden können.
\end{definition}

\subsection*{Aufgabenteil}

\begin{exercise}
    Forme den Term $4\cdot (x+2)-10\cdot (5+2x)$ um, um herauszufinden, welche Zahlen du in die Boxen einsetzen musst, damit $4\cdot (x+2)-10\cdot (5+2x)=\Box\cdot x+\Box$ gilt.
    \begin{answerbox}[.5in]
        $4\cdot (x+2)-10\cdot (5+2x)=4x+8-50-20x=-16x-42$. 
        
        In die erste Box gehört also eine $-16$ und in die zweite eine $-42$.
    \end{answerbox}
\end{exercise}

\begin{exercise}
    Welche der folgenden Gleichungen sind linear?
    \begin{multicols}{3}
        \begin{multiplechoice}
            \citem $7x+1=4x+13$
            \citem $0=0$
            \item $\sqrt{4x}+3=x-5$
            \item $3x(x+3)=4$
            \citem $1=1$
            \citem $3x=2$
            \item $x^2-3x=4$
            \citem $4\cdot (x+2)-30=15x$
            \citem $4x=6x$
        \end{multiplechoice}
    \end{multicols}
\end{exercise}

\newpage

\video{Lineare Gleichungen auflösem}{4}{Kapitel \ref{ext:sec:abbildungen_intuition} (ab Seite \pageref{ext:sec:abbildungen_intuition})}{https://www.google.de}

\begin{remark}
    Lösungsalgorithmus für lineare Gleichungen
\end{remark}

\subsection*{Aufgabenteil}

\mandala{mandala/mandala01}

\end{document}
