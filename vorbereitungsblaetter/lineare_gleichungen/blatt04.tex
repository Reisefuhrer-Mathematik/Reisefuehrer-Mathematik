\def\pathToMain{../../buch/}
\documentclass{uebungsblatt}
\usepackage[utf8]{inputenc}
\usepackage[T1]{fontenc}
\usepackage[ngerman]{babel}

\usepackage{mathdef}

\sheet{Vorbereitungsblatt 15.4}
\title{Auflösen linearer Ungleichungen}
\topic{\getchaptername{lineare_gleichungen}}
\chapternum{\getchapternum{lineare_gleichungen}}

\begin{document}
\maketitle
\begin{contents}
    Lineare Ungleichungen
\end{contents}

\video{Lineare Ungleichungen}{4}{Kapitel \ref{ext:sec:abbildungen_intuition} (ab Seite \pageref{ext:sec:abbildungen_intuition})}{https://www.google.de}

\begin{theorem}
    Die Lösungsmenge einer Ungleichung bleibt unverändert, wenn man
    \begin{itemize}
        \item zu jeder Seite eine Zahl $a\in\mathbb{R}$ addiert.
        \item jede Seite mit einer Zahl $a>0$ multipliziert.
    \end{itemize}
\end{theorem}

\subsection*{Aufgabenteil}

\mandala{mandala/mandala01}

\end{document}
