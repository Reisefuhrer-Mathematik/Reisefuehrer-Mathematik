\def\pathToMain{../../buch/}
\documentclass{uebungsblatt}
\usepackage[utf8]{inputenc}
\usepackage[T1]{fontenc}
\usepackage[ngerman]{babel}

\usepackage{mathdef}

\sheet{Vorbereitungsblatt 15.5}
\title{Geradengleichungen}
\topic{\getchaptername{lineare_gleichungen}}
\chapternum{\getchapternum{lineare_gleichungen}}

\begin{document}
\maketitle
\begin{contents}
    Geradengleichungen, Steigung einer Geraden
\end{contents}

\video{Geradengleichungen}{4}{Kapitel \ref{ext:sec:abbildungen_intuition} (ab Seite \pageref{ext:sec:abbildungen_intuition})}{https://www.google.de}

\begin{definition}
    Es sei $f$ eine Funktion und $f(x)=ax+b$. Dann heißt $a$ die \textbf{Steigung} des Graphen von $f$.
\end{definition}

\begin{theorem}
    Der Graph einer Funktion $f(x)=ax+b$ ist eine Gerade mit der Steigung $a$ und dem Achsenabschnitt $b$.
\end{theorem}

\subsection*{Aufgabenteil}

\mandala{mandala/mandala01}

\end{document}
