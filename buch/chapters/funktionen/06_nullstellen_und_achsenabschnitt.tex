\documentclass[../../main.tex]{subfiles}

\begin{document}

Die wichtigste Eigenschaft einer Funktion ist ihre Zuordnungsvorschrift. Du weißt bereits, wie sich Zuordnungsvorschriften beschreiben lassen und wie du dir Funktionen graphisch vorstellen kannst. Mithilfe des Funktionsgraphen, der in einem vorherigen Abschnitt erklärt wurde, können diese Vorschriften sehr gut veranschaulicht werden.

In diesem Abschnitt werden die Zuordnungsvorschriften weiter untersucht: Du lernst Begriffe kennen, mit denen sich Zuordnungsvorschriften beschreiben lassen und mit denen sie sich besser verstehen lassen. 

Mittlerweile solltest du gut verstanden haben, was eine Funktion ist. Für das Bild $f(x)$, das einem Argument $x$ durch die Anwendung einer Funktion $f$ zugeordnet wird, verwenden wir im Folgenden eine kompaktere Bezeichnung: Man nennt $f(x)$ auch den \textbf{Funktionswert von $f$ an der Stelle $x$}.

Während es in den einführenden Abschnitten sehr wichtig war, zu erwähnen, welche Funktion welche Definitions- und Zielmenge hatte (weil diese wegen der vielen anschaulichen Beispiele immer unterschiedlich waren), rückt die Frage, welche Definitions- und Zielmenge eine Funktion hat, ab sofort eher in den Hintergrund.

Wenn nichts anderes gesagt ist, wird deshalb ab sofort davon ausgegangen, dass eine Funktion einfach Zahlen auf Zahlen abbildet. Jede Funktion, bei der Definitions- und Zielmenge nicht in der Schreibweise $f\colon U\rightarrow V$ angegeben sind, sondern weggelassen werden, soll einfach als eine Funktion von den reellen Zahlen in die reellen Zahlen, also als eine Funktion $f\colon\Real\rightarrow\Real$, verstanden werden.

\begin{example}{}
    \parpic[r]{
        \begin{tikzpicture}
            \begin{axis}[defgrid, domain=0:3, y=.8cm, x=.8cm, ymin=0, ymax=2, xmin=0, xmax=3, samples=2]
                \addplot[color=violet] expression{(x+1)/2};
            \end{axis}
        \end{tikzpicture}
    }
    Die Funktion $f(x)=\frac{1}{2}(x+1)$, die den rechts abgebildeten Graphen besitzt, ordnet reellen Zahlen andere reelle Zahlen zu. Wir sparen es uns allerdings ab sofort, dies durch die Schreibweise $f\colon\Real\rightarrow\Real$ aufzuschreiben, sondern nehmen an, dass dies einfach der Normalfall ist.

    Interessanter als die Zielmenge einer Funktion ist daher meist eher die \emph{Wertemenge}, also die Menge der Zahlen, die von $f$ wirklich erreicht werden können.
\end{example}

\subsection{Merkmale des Funktionsgraphen}

Funktionsgraphen können eine Vielzahl von möglichen Eigenschaften haben. Sie können vollkommen verschieden aussehen und beispielsweise die Form einer Gerade haben, ständig die $x$-Achse schneiden oder einfach nach rechts immer weiter nach oben verlaufen.

\begin{example}{}
    Die folgenden drei Bilder zeigen jeweils den Graphen einer bestimmten Funktion. Der linke Graph schneidet die $y$-Achse in der Höhe $4$ beim Punkt $\coord{0}{4}$. Es ist kein Punkt zu sehen, an dem er die $x$-Achse schneidet. Da der Graph aber in gerader Linie nach unten steigt, könnte man erwarten, dass er das weiter rechts tun würde.
    \begin{multicols}{3}
        \centering
        \begin{tikzpicture}
            \begin{axis}[defgrid, domain=0:4, y=.7cm, x=.7cm, ymin=0, ymax=4, xtick={1,...,4}, ytick={1,...,4}]
                \addplot[color=violet] coordinates{(0,4) (4,2)};
            \end{axis}
        \end{tikzpicture}
        
        \begin{tikzpicture}
            \begin{axis}[defgrid, domain=0:4, y=.7cm, x=.7cm, ymin=0, ymax=4, xtick={1,...,4}, ytick={1,...,4}, samples=\ifdraft{5}{20}]
                \addplot[color=violet] expression{-0.06*(x-4)^3};
            \end{axis}
        \end{tikzpicture}
        
        \begin{tikzpicture}
            \begin{axis}[defgrid, domain=0:4, y=.7cm, x=.7cm, ymin=0, ymax=4, xtick={1,2,3,4}, ytick={1,...,4}, samples=\ifdraft{7}{30}]
                \addplot[color=violet] expression{2*sin(3.14159*deg(x))+2};
            \end{axis}
        \end{tikzpicture}
    \end{multicols}
    Der rechte Graph schneidet die $x$-Achse hingegen zweimal im Bild. Da er sich ständig auf- und abbewegt, wird er das vermutlich auch noch häufiger machen. Die $y$-Achse schneidet er beim Punkt $(0\,|\,2)$.
    
    Schließlich ist im mittleren Bild gar nicht klar, ob er die $x$- oder $y$-Achse schneidet, da er sich nicht in gerader Linie darauf zubewegt. Dafür sieht man im mittleren Bild (ebenso wie im linken), dass die $y$-Werte immer kleiner werden, wenn man dem Graphen nach rechts folgt.
\end{example}

Um vernünftig beschreiben zu können, was jeweils die Besonderheiten der Funktionsgraphen aus dem letzten Beispiel sind, lernst du hier ein paar Begrifflichkeiten kennen, die sich eignen, um einfache Merkmale von Funktionsgraphen zu untersuchen. Für den Moment interessieren uns vor allem die folgenden Fragen:
\begin{itemize}[noitemsep]
    \item Wo schneidet der Graph die $y$-Achse?
    \item Wo schneidet der Graph die $x$-Achse?
    \item Ist der Funktionsgraph symmetrisch?
\end{itemize}

Diese drei Fragen sollen in diesem Kapitel untersucht werden. Wir werden gegen Ende dieses Kapitels außerdem untersuchen, wie sich der Graph einer Funktion gezielt durch Änderungen an der Berechnungsvorschrift modifizieren lässt. Vor allem im Kapitel über \textbf{Differentialrechnung in \Real}, dessen Hauptziel eine genauere Beschreibung von Funktionsgraphen ist, lernst du später einige fortgeschrittenere Möglichkeiten kennen, Funktionsgraphen systematisch zu untersuchen.

\subsection{$y$-Achsenabschnitt}
\label{sec:abbildungen_ordinatenabschnitt}

Wenn man den Graphen einer Funktion sieht, ist die Frage, wo dieser die $y$-Achse schneidet, meistens sehr leicht zu beantworten. Durch einfaches Hinsehen lässt sich schnell der gesuchte $y$-Wert ablesen. Im Beispiel weiter oben hat man sofort gesehen, dass die linken beiden Graphen die $y$-Achse beim $y$-Wert $4$ schneiden. Der $y$-Wert des Punktes, an dem der Graph die $y$-Achse schneidet, wird \textbf{$y$-Achsenabschnitt} genannt.

\begin{definition}{$y$-Achsenabschnitt}
    Für eine Funktion $f$ heißt der $y$-Wert, an dem der Graph von $f$ die $y$-Achse schneidet, der \textbf{$y$-Achsenabschnitt} (oder \textbf{Ordinatenabschnitt}) von $f$.
\end{definition}

Es hat allerdings einige Nachteile, wenn man den $y$-Achsenabschnitt nur durch das Ansehen des Funktionsgraphen bestimmen kann. Erstens erhält man so nicht immer genaue Werte und zweitens muss der gesuchte Schnittpunkt nicht immer ein kleiner Wert sein. In diesem Fall würde man ein sehr großes Koordinatensystem benötigen, um den $y$-Achsenabschnitt durch Hinsehen zu bestimmen. Auch wenn die Funktion nicht als Graph, sondern -- wie es normal der Fall ist -- als Berechnungsvorschrift vorliegt, ist es nicht hilfreich, auf den Funktionsgraphen angewiesen zu sein. Deshalb untersuchen wir nun, wie sich der $y$-Achsenabschnitt auch rechnerisch ohne das Benutzen eines Funktionsgraphen bestimmen lässt.

\begin{example}{}
    \parpic[r]{
        \begin{tikzpicture}
            \begin{axis}[defgrid, domain=0:4, y=1cm, x=1cm, xmin=0, xmax=4, ymin=0,ymax=4,xtick={1,...,4}, ytick={1,...,4}]
                \addplot[color=violet] expression{x+1};
                \addplot[mark=*, only marks, fill=violet] coordinates {(2,3)};
            \end{axis}
        \end{tikzpicture}
    }
    Der rechts abgebildete Graph gehört zur Funktion \mbox{$f(x)=x+1$}. Der eingezeichnete Punkt hat die $x$-Koordinate $2$ und damit insgesamt die Koordinate $\coord{2}{f(2)}=\coord{2}{3}$. 
    
    Weil jedes Argument nur ein Bild haben kann, gibt es nur einen Punkt auf dem Graphen von $f$, der den $x$-Wert $2$ hat.
    
    \picskip{3}
    Der Graph schneidet die $y$-Achse beim Wert $1$. Der dort liegende Punkt hat die $x$-Koordinate $0$ und insgesamt die Koordinate $\coord{0}{f(0)}=\coord{0}{1}$. Damit hat die Funktion $f$ den $y$-Achsenabschnitt $1$.
    
    Der $y$-Achsenabschnitt von $f$ entspricht somit dem Wert $f(0)$ -- denn auf genau dieser Höhe liegt der Punkt über $x=0$, der die entsprechende Zuordnungsvorschrift darstellt.
\end{example}

Um einen Funktionsgraphen zu erhalten, hatten wir über jedem möglichen Argument (also jedem möglichen $x$-Wert) einen Punkt eingezeichnet. Jeder Punkt auf dem Funktionsgraphen steht für eine Zuordnungsregel, die einem bestimmten Argument ein bestimmtes Bild zuordnet und hat die Koordinate $\coord{x}{f(x)}$ für ein $x\in\Real$.

Wir suchen nun einen Punkt, der auf dem Graphen einer Funktion und gleichzeitig auch auf der $y$-Achse liegt. Jeder Punkt, der auf der $y$-Achse liegt, hat den $x$-Wert $0$ (die $y$-Achse geht nämlich vom Ursprung, dessen Koordinate $\coord{0}{0}$ ist, gerade nach oben und hat deshalb denselben $x$-Wert wie der Ursprung). 

Im Beispiel haben wir bereits gesehen, dass $\coord{0}{f(0)}$ der einzige Punkt auf dem Graphen ist, der auf der $y$-Achse liegt. Um den $y$-Achsenabschnitt einer beliebigen Funktion auszurechnen, genügt es aus diesem Grund, einfach $f(0)$ zu berechnen.

\begin{example}{}
    \sloppy
    Die Funktion $f(x)=4(x+1)$ hat den $y$-Achsenabschnitt $4$, denn um den $y$-Achsenabschnitt zu berechnen, muss einfach nur $f(0)$ berechnet werden. Es gilt $f(0)=4\cdot (0+1)=4$.
\end{example}

\fussy

\subsection{Nullstellen}
\label{sec:abbildungen_nullstelle}

Wie bereits beim $y$-Achsenabschnitt ist es, wenn man einen Graphen sieht, natürlich möglich, durch Hinsehen zu bestimmen, wo der Graph die $x$-Achse schneidet -- allerdings auch hier mit denselben Problemen. Um also zu \emph{berechnen}, wo ein Funktionsgraph die $x$-Achse schneidet, müssen Punkte berechnet werden, deren $y$-Koordinate $0$ ist (das ist die Bedingung dafür, dass ein Punkt auf der $x$-Achse liegt).

Jeder Punkt auf dem Funktionsgraphen hat die Koordinate $\coord{x}{f(x)}$, also eine Kombination aus einem Element $x$ aus der Definitionsmenge und dem Bild $f(x)$, das $x$ zugeordnet wird. Damit die $y$-Koordinate $0$ wird, muss der rechte Teil, also $f(0)$, null werden. Für Schnittpunkte mit der $x$-Ache müssen deshalb einfach Werte für $x$ gefunden werden, für die $f(x)=0$ gilt.

\begin{example}{}
    \parpic[r]{
        \begin{tikzpicture}[samples=\ifdraft{5}{30}]
            \begin{axis}[defgrid, domain=0:4, y=.75cm, x=.75cm, xtick={1,...,4}, ytick={-1,...,3}, samples=\ifdraft{5}{20}]
                \addplot[color=violet] expression{(x-2)^2-1};
            \end{axis}
        \end{tikzpicture}
    }
    Rechts ist der Graph der Funktion $f(x)=(x-2)^2-1$ abgebildet. Der Graph schneidet die $x$-Achse bei $x=1$ und $x=3$. 
    
    Das liegt daran, dass die Funktion für beide $x$-Werte null wird: Es gilt \[f(3)=(3-2)^2-1=1-1=0\] und \[f(1)=(1-2)^2-1=1-1=0.\]
    
    \picskip{0}
    Wie du darauf kommen kannst, dass die Funktion ausgerechnet an den Stellen $1$ und $3$ den Wert $0$ annimmt, ist zunächst nicht wichtig (das wird Gegenstand der Kapitel über lineare und quadratische Gleichungen sein).
    
    Weil der $y$-Wert eines Punktes auf dem Graphen immer $f(x)$ (also dem Bild eines gewissen Arguments) entspricht, liegt für diese beiden $x$-Werte der Punkt auf der $x$-Achse (denn der $y$-Wert ist $0$).
\end{example}

Der Graph einer Funktion schneidet die $x$-Achse immer dann, wenn das Bild des betreffenden $x$-Wertes $0$ ist, also wenn $f(x)=0$ gilt. Wegen dieser Eigenschaft nennt man die $x$-Werte, für die eine Funktion den Wert $0$ annimmt (und damit die $x$-Achse schneidet) auch \textbf{Nullstellen}.

\begin{definition}{Nullstelle\index{Nullstelle}}
    Ist $f$ eine Funktion, dann heißen alle $x\in\Real$ mit $f(x)=0$ \textbf{Nullstellen} von $f$.
\end{definition}

Um herauszufinden, welche Werte man für $x$ wählen kann, damit $f(x)=0$ gilt, ist es nötig, eine Gleichung aufzulösen. $f(x)$ sollte dir immer in Form einer Berechnungsvorschrift vorliegen, z.B. in der Art \enquote{$f(x)=2x$}. Nun wäre es erforderlich, die Gleichung $2x=0$ aufzulösen, also alle $x$ zu finden, für die $2x=0$ gilt. 

Das lernst du vor allem in den Kapiteln über \textbf{lineare und quadratische Gleichungen}. Zunächst wird es nicht erforderlich sein, dass du systematisch Gleichungen lösen kannst -- stattdessen werden die Nullstellen in der Regel so einfach sein, dass sie sich leicht durch Ausprobieren finden lassen.

\begin{example}{}
    Die Nullstellen der Funktion $f(x)=x+5$ sind alle Werte, die sich für $x$ einsetzen lassen, damit $f(x)=0$ gilt. Es muss die Gleichung $\colorbrace{x+5}{f(x)}=0$ gelöst werden. Der einzige Wert, den man für $x$ wählen kann, damit $f(x)=x+5=0$ gilt, ist $x=-5$.
    
    Es folgt, dass $f$ genau eine Nullstelle hat, und zwar bei $x=-5$.
\end{example}

\begin{nutshell}{Nullstellen und $y$-Achsenabschnitt}
    Die Schnittpunkte eines Funktionsgraphen mit den beiden Koordinatenachsen (also der $x$-Achse und der $y$-Achse) werden besonders bezeichnet.
    
    Ein Schnittpunkt des Graphen einer Funktion $f$ mit der $x$-Achse wird als \textbf{Nullstelle} von $f$ bezeichnet, weil die Funktion hier den Wert 0 hat: Damit der Graph die $x$-Achse an einem Punkt $\coord{x}{0}$ schneidet, muss $f(x)=0$ sein. Grundsätzlich kann eine Funktion mehrere Nullstellen haben, nämlich dann, wenn es mehrere verschiedene Werte für $x$ gibt, sodass $f(x)=0$ gilt.
    
    Ein Schnittpunkt des Graphen von $f$ mit der $y$-Achse heißt \textbf{$y$-Achsenabschnitt} von $f$. Jede Funktion hat nur einen $y$-Achsenabschnitt, weil jeder Punkt auf der $y$-Achse eine Koordinate $\coord{0}{y}$ haben muss, deren $x$-Wert 0 ist. Der $y$-Achsenabschnitt ist genau der Wert $f(0)$, weil das die einzige Zuordnungsregel ist, die einen Punkt auf der $y$-Achse erzeugen kann.
\end{nutshell}

\end{document}