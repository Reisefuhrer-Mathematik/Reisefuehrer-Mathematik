\documentclass[../../main.tex]{subfiles}

\begin{document}

\subsection{Die Identische Funktion}
\label{sec:abbildungen_identitaet}

In der Regel verändern Funktionen ihre Argumente, indem sie diesen Bilder zuordnen. Eine solche Zuordnung von Elementen der Definitionsmenge zu Elementen der Zielmenge ist schließlich der Zweck von Funktionen. 

Es ist jedoch nicht verboten, mit Funktionen, die gar nichts tun, zu arbeiten. Damit sind Funktionen gemeint, die ihr Argument nicht verändern. Das heißt, jedes Argument, das man in die Funktion einsetzt, wird sich selbst zugeordnet.

\begin{example}{}
    \parpic[r]{%\begin{center}
    \begin{tikzpicture}[scale=.6]
        \fill[black!10] (-1.5,0) ellipse (2.4cm and 2cm);
        \fill[black!10] (4,0) ellipse (2.4cm and 2cm);
        %
        \draw (-1.5,0) ellipse (2.4cm and 2cm);
        \draw (4,0) ellipse (2.4cm and 2cm);
        %
        \node[blue] (x1) at (-1,0.7) {$\bullet$}; 
        \node[blue,left=1mm of x1] {blau};
        \node[red] (x2) at (-1,-0.2) {$\bullet$}; 
        \node[red,left=1mm of x2] {rot};
        \node[yellow!70!black] (x3) at (-1,-1.1) {$\bullet$}; 
        \node[yellow!70!black,left=1mm of x3] {gelb};
        %
        \node[blue] (y1) at (3.5,0.7) {$\bullet$}; 
        \node[blue,right=1mm of y1] {blau};
        \node[red] (y2) at (3.5,-0.2) {$\bullet$}; 
        \node[red,right=1mm of y2] {rot};
        \node[yellow!70!black] (y3) at (3.5,-1.1) {$\bullet$}; 
        \node[yellow!70!black,right=1mm of y3] {gelb};
        %
        \draw[->] (x1) to[bend left] (y1);
        \draw[->] (x2) -- (y2);
        \draw[->] (x3) to[bend right] (y3);
    \end{tikzpicture}
%\end{center}}
    Wir haben bereits die Funktion $\textsc{MitGelbMischen}$ gesehen, die Grundfarben erhält und mit gelb mischt. Dadurch verändern sich die Farben blau zu grün und rot zu orange. Deutlich langweiliger ist es, eine Grundfarbe mit sich selbst zu mischen, also blau mit blau, rot mit rot und gelb mit gelb. Natürlich ändert eine Farbe sich nicht, wenn man sie mit sich selbst mischt.
    
    \picskip{0}
    Es lässt sich eine Funktion $\textsc{MitSichSelbstMischen}$ definieren, bei der jede Grundfarbe der Farbe zugeordnet wird, die entsteht, wenn man die Grundfarbe mit sich selbst mischt. Weil das wie bereits beschrieben die Farbe nicht ändert, erhält man dadurch eine Funktion
    \[\textsc{MitSichSelbstMischen}\colon\textsc{Grundfarben}\rightarrow\textsc{Grundfarben}\]
    mit
    \[\textsc{MitSichSelbstMischen}(x)=x.\]
\end{example}

Eine Funktion mit dieser Eigenschaft wird \textbf{identische Funktion} genannt. Der Name kommt daher, dass Urbild und Bild immer identisch sind. Eine identische Funktion gibt es für jede Definitionsmenge (das ist dann immer die Funktion, die allen Elementen dieser Menge sich selbst zuordnet).

\begin{definition}{Identische Funktion}
    Es sei $M$ eine Menge. Dann heißt die Funktion, die jedes Element von $M$ auf sich selbst abbildet, die \textbf{identische Funktion} oder \textbf{Identität} auf $M$, geschrieben $\ident[M]$. Es gilt also \[\ident[M](x)=x\text{~für alle~}x\in M.\]
\end{definition}

Die identische Funktion auf einer Menge $M$ notiert man also mit $\ident[M]$. Meistens lässt man die Menge im Index weg, schreibt also einfach nur $\ident$, wenn klar ist, um welche Menge es geht. Wir werden die Menge hier allerdings zur Klarheit weiterhin mit aufschreiben.

\subsection{Die Umkehrfunktion}
\label{sec:abbildungen_umkehrabbildung}

Im letzten Abschnitt hast du gesehen, dass identische Funktionen solche Funktionen sind, die ihre Argumente nicht verändern, sondern einfach sich selbst zuordnen. Eine solche Funktion anzuwenden, hat keinen Effekt, denn sie verändert nichts.

Dieser Abschnitt thematisiert die Frage, ob es möglich ist, wenn man bereits eine Funktion $f$ angewandt hat, anschließend eine andere Funktion $g$ anzuwenden, die den Effekt von $f$ umkehrt. Das heißt, dass wenn man zuerst $f$ und dann $g$ anwendet, insgesamt nichts passiert. Ist $f$ eine Funktion von $U$ nach $V$ ($U,V$ sind Mengen), dann ist also eine Funktion $g\colon V\rightarrow U$ gesucht, sodass bei der Verkettung von $f$ und $g$ die identische Funktion herauskommt.

\parpic[r]{
    \begin{tikzpicture}[scale=.75]
    \draw[grayset] (-1.5,0) ellipse (0.7cm and 2cm);
    \draw[grayset] (1.5,0) ellipse (0.7cm and 2cm);

    \node (x1) at (-1.5,0.7) {$\bullet$};
    \node (x2) at (-1.5,-0.2) {$\bullet$};
    \node (x3) at (-1.5,-1.1) {$\bullet$};
    \node (y1) at (1.5,0.7) {$\bullet$};
    \node (y2) at (1.5,-0.2) {$\bullet$};
    \node (y3) at (1.5,-1.2) {$\bullet$};

    \draw[blue,->] (x1) -- (y3);
    \draw[blue,->] (x2) to[bend right] (y1);
    \draw[blue,->] (x3) to[bend right] (y2);
    
    \draw[dashed,->] (y3) to[bend right] (x1);
    \draw[dashed,->] (y1) to[bend right] (x2);
    \draw[dashed,->] (y2) to[bend right] (x3);
\end{tikzpicture}
}

Rechts ist die Funktion $f$ mit blauen Pfeilen dargestellt. Die gestrichelte Funktion dreht alle Pfeile von $f$ um. Dadurch läuft man immer im Kreis, wenn man erst einem blauen und dann einem gestrichelten Pfeil folgt. Führt man die beiden Funktionen hintereinander aus, bekommt man deshalb die identische Funktion auf der linken Menge.

\begin{example}{}
    Die Funktion $f\colon\Real\rightarrow\Real$ mit der Zuordnungsvorschrift $f(x)=x+1$ addiert zu jeder Zahl 1. Wenn man vom Ergebnis wieder $1$ subtrahiert, kommt man bei der Zahl an, bei der man angefangen hat. Für die Funktion $g\colon\Real\rightarrow\Real$ mit $g(x)=x-1$ gilt also $(g\circ f)(x)=x$. Das lässt sich nachrechnen:
    \[g(f(x))=g(x+1)=(x+1)-1=x.\]
    Die Funktion $g$ kehrt die Funktion $f$ daher um und es gilt $g\circ f=\ident[\Real]$.
\end{example}

Eine Funktion $g$, die eine Funktion $f\colon U\rightarrow V$ rückgängig macht bzw. umkehrt, wird Umkehrfunktion von $f$ genannt. Wie eben erklärt, drückt sich das dadurch aus, dass bei der Hintereinanderausführung von $f$ und $g$ die identische Funktion herauskommt, also $g\circ f=\ident[U]$. Dass $g$ eine Funktion von $V$ nach $U$ sein muss, ist erzwungen: Die Definitionsmenge muss $V$ sein, damit die Funktionen überhaupt hintereinander ausgeführt werden dürfen. Die Zielmenge von $g$ muss $U$ sein, weil man bei Elementen aus $U$ begonnen hat und entsprechend auch wieder zu Elementen aus $U$ zurück möchte.

\begin{definition}{Umkehrfunktion}
    Es seien $U,V$ Mengen und $f\colon U\rightarrow V$ eine Funktion. Eine Funktion $g\colon V\rightarrow U$ mit $g\circ f=\ident[U]$ und $f\circ g=\ident[V]$ heißt \textbf{Umkehrfunktion} von $f$. In diesem Fall schreibt man $f^{-1}$ statt $g$.
\end{definition}

Die Umkehrfunktion einer Funktion $f$ notiert man mit $f^{-1}$. Im nächsten Beispiel wird noch einmal erläutert, warum $g$ im letzten Beispiel eine Umkehrfunktion von $f$ war. Anschließend werden zwei weitere Beispiele präsentiert.

\begin{example}{}
    Im letzten Beispiel haben wir gesehen, dass für $f\colon\Real\rightarrow\Real$ mit $f(x)=x+1$ die Umkehrfunktion durch $f^{-1}(x)=x-1$ bestimmt ist. Es gilt nämlich $g\circ f=\ident[\Real]$. Außerdem gilt $f\circ g=\ident[\Real]$, weil $f(g(x))=f(x-1)=(x-1)+1$ gilt.
\end{example}

\begin{example}{}
    Wenn ein mit Butter beschmiertes Toast von einem normalen Tisch nach unten fällt, landet es leider oft mit der beschmierten Seite unten. Während des Falls kann das Toast sich prinzipiell mehrmals drehen. Nach jedem Drehen zeigt entweder die Seite mit Butter oder die Seite ohne Butter nach oben.
    
    Die Funktion $\textsc{Umdrehen}\colon\textsc{Toastseiten}\rightarrow\textsc{Toastseiten}$ gibt nun an, welche Seite des Toasts nach dem Umdrehen oben ist, abhängig davon, welche vorher oben war (wobei $\textsc{Toastseiten}$ die Menge der Seiten ist, die sich prinzipiell oben befinden können, d.h. $\textsc{Toastseiten}=\{\text{Butter},\text{keine Butter}\}$). Es gilt (weil sich nach einmaligem Umdrehen jeweils die andere Seite oben befindet)
    \[\textsc{Umdrehen}(\text{Butter})=\text{keine Butter}\] und \[\textsc{Umdrehen}(\text{keine Butter})=\text{Butter}.\]
    Wenn sich das Toast zweimal umdreht, ist wieder die Seite oben, die vorher oben war, d.h. \[\textsc{Umdrehen}(\textsc{Umdrehen}(\text{keine Butter}))=\textsc{Umdrehen}(\text{Butter})=\text{keine Butter}\]
    und auf die gleiche Weise liegt auch die Seite mit Butter oben, wenn vorher die Seite mit Butter oben war und sich das Toast zweimal umdreht. Damit ist das zweimalige Umdrehen des Toasts die identische Funktion auf der Menge \textsc{Toastseiten}. Weil $\textsc{Umdrehen}\circ \text{Umdrehen}=\ident[\textsc{Toastseiten}]$ gilt, ist \textsc{Umdrehen} die Umkehrfunktion von sich selbst.
\end{example}

\begin{example}{}
    Die Funktion $f\colon\Real\rightarrow\Real$ mit $f(x)=4x$ hat die Umkehrfunktion $f^{-1}\colon\Real\rightarrow\Real$ mit $f^{-1}(x)=\frac{1}{4}x$: Um eine Multiplikation mit $4$ rückgängig zu machen, muss man durch $4$ teilen (oder mit $\frac{1}{4}$ multiplizieren).
    
    Zur Überprüfung lässt sich schnell nachrechnen, dass \[(f^{-1}\circ f)(x)=f^{-1}(f(x))=f^{-1}(4x)=\frac{1}{4}\cdot 4x=x.\] 
    Damit ist $f^{-1}\circ f=\ident[\Real]$. Natürlich kann ebenso gezeigt werden, dass $f\circ f^{-1}=\ident[\Real]$ ist.
\end{example}

\begin{advanced}{Injektionen, Surjektionen und Bijektionen}
    Wenn man die Funktionsvorschrift einer Funktion weiter analysieren möchte, ist eine wichtige Eigenschaft, wie viele Urbilder die Elemente der Zielmenge jeweils haben. Man stellt sich in diesem Zusammenhang die Frage, ob
    \begin{itemize}[noitemsep]
        \item jedes Element der Zielmenge mindestens ein Urbild hat oder
        \item jedes Element der Zielmenge maximal ein Urbild hat oder
        \item jedes Element der Zielmenge genau ein Urbild hat.
    \end{itemize}
    
    Eine Funktion heißt im ersten Fall \textbf{surjektiv}, im zweiten Fall \textbf{injektiv} und im dritten Fall (wenn die Funktion also sowohl injektiv als auch surjektiv ist) heißt sie \textbf{bijektiv}.
    
    Es ist auch möglich, dass eine Funktion keine dieser Eigenschaften hat.
    Eine Erläuterung dieser Eigenschaften, eine formale Definition sowie Beispiele findest du am Ende dieses Kapitels auf Seite \pageref{advanced:bijektion}.
\end{advanced}

\begin{nutshell}{Identität und Umkehrfunktion}
    \sloppy
    Eine \textbf{identische Funktion} auf einer Menge $M$ (geschrieben $id_M$) ist eine Funktion, die als Definitions- und Zielmenge $M$ hat und alle Argumente unverändert lässt, d.h. $f(x)=x$ für alle $x\in M$. Damit handelt es sich um eine Funktion, die nichts tut.
    
    Wenn es möglich ist, zu einer Funktion $f\colon U\rightarrow V$ eine Funktion $g\colon V\rightarrow U$ zu finden, sodass diese sich gegenseitig rückgängig machen (also $g\circ f=\ident[U]$ und $f\circ g=\ident[V]$), dann nennt man $g$ die \textbf{Umkehrfunktion} von $f$ und schreibt für $g$ auch $f^{-1}$.
\end{nutshell}
\fussy

\end{document}