\documentclass[../../main.tex]{subfiles}

\begin{document}

\parpic[r]{
    \tikz{
        \begin{axis}[defgrid, domain=0:4, y=1cm, x=1cm, ymin=0, ymax=4, xmin=0, xmax=4, xtick={1,...,4}, ytick={1,...,4}]
            \filldraw[opacity=0.5,violet!20, draw=violet] (1,0) -- (3,2) -- (0,3) -- cycle;
            \filldraw[opacity=0.5,orange!20, draw=orange] (3,1) -- (3,4) -- (4,4) -- (4,1) -- cycle;
        \end{axis}    
    }
}

Die hast bereits gelernt, wie du den Flächeninhalt von Quadraten, Dreiecken, Trapezen, Kreisen und einigen weiteren
Flächen ausrechnen kannst -- etwa von den Flächen, die rechts im Koordinatensystem eingezeichnet sind.

\picskip{4}
Das funktioniert, weil wir uns dafür früher einmal Formeln für den Flächeninhalt überlegt haben, die wir nun anwenden 
können. Im Falle des orangefarbenen Rechtecks ist der Flächeninhalt $A_R$ etwa das Produkt aus seiner Breite und seiner
Höhe, also
\[A_R=1\cdot 3=3.\]

\parpic[l]{
    \begin{tikzpicture}
        \begin{axis}[
            defgrid, y=1cm, x=1cm, ymin=0, ymax=4, xmin=0, xmax=4, xtick={1,...,4}, ytick={1,...,4}
            ]
            \addplot[name path=poly, domain=1:4, violet] {0.33333*(x-1)*(x-4)*(x-6)};
            \addplot[name path=line, domain=1:4, violet] {0};
            \addplot[fill opacity=0.5, fill=violet!20] fill between[ 
            of = poly and line,
            soft clip={domain=1:4},
            ];
        \end{axis}
    \end{tikzpicture}
}
Mit unseren Formeln haben wir aber bei weitem nicht alle Flächen abgedeckt, die es gibt. Zum Beispiel hat die linke
Fläche oben eine seltsame Krümmung, die uns nicht erlaubt, sie mit uns bekannten Formeln auszurechnen. Was wir allerdings
wissen, ist, dass die obere Begrenzung der Fläche zur Funktion
\[f(x)=\frac{(x-1)(x-4)(x-6)}{3}\] 
gehört. In diesem Kapitel wollen untersuchen, wie wir beliebige Flächen berechnen können, wenn wir zumindest wissen,
durch welche Funktionen sie begrenzt sind.

\begin{example}{}
    \parpic[r]{
    \begin{tikzpicture}
        \begin{axis}[
            defgrid, y=0.15cm, x=1cm, ymin=0, ymax=25, xmin=0, xmax=4, xtick={1,...,4}, ytick={5,10,15,20,25}
            ]
            \addplot[name path=f, domain=0:4, violet] {20};
            \addplot[name path=g, domain=0:4, violet] {0};
            \addplot[fill opacity=0.5, fill=violet!20] fill between[ 
            of = f and g,
            soft clip={domain=0:4},
            ];
        \end{axis}
    \end{tikzpicture}
    }
    Wir kennen aus der Physik die Formel $v=\frac{s}{t}$, wobei $v$ die Geschwindigkeit zum Beispiel eines Fahrrades ist,
    das sich bewegt, $s$ die zurückgelegte Strecke und $t$ die dafür benötigte Zeit.

    \picskip{6}
    Indem wir die Gleichung mit $t$ multiplizieren, erhalten wir die Formel $v\cdot t=s$. Wenn du zum Beispiel weißt, 
    dass du auf dem Fahrrad normalerweise mit $20\,\frac{\text{km}}{\text{h}}$ (dargestellt auf der $y$-Achse) unterwegs 
    bist und eine Radtour für 4 Stunden planst, dann kannst du nun ausrechnen, welche Strecke du in der Zeit schaffen
    wirst, wenn du keine Pausen machst und die Radtour entsprechend planen. Dafür setzt du Zeit und Geschwindigkeit in
    die Gleichung $s=v\cdot t$ sein und erhältst
    \[s=20\,\frac{\text{km}}{\text{h}}\cdot 4\,\text{h}=80\,\text{km}.\]
    Dies ist auch genau die Fläche des rechts markierten Rechtecks, also der Fläche unter der Funktion 
    $v(t)=20\,\frac{\text{km}}{\text{h}}$, die deine Geschwindigkeit beschreibt. Es gibt einige weitere Zusammenhänge
    (nicht nur in der Physik), die sich durch das Berechnen der Fläche unter einer Funktion ausdrücken lassen.
\end{example}
Flächen wie im letzten Beispiel oder in der Abbildung weiter oben links werden wir in Zukunft als \textbf{Integral} 
bezeichnen. Ein Integral ist also die Fläche zwischen einer Funktion und der $x$-Achse (sofern die Fläche oberhalb der 
$x$-Achse liegt; Flächen unterhalb der $x$-Achse wird ein negativer Flächeninhalt zugeordnet).
\begin{multicols}{2}
    \begin{center}
    \begin{tikzpicture}
        \begin{axis}[
            defgrid, y=0.8cm, x=1cm, ymin=-2, ymax=2, xmin=-3, xmax=3, xtick={-3,...,3}, ytick={-2,...,2}
            ]
            \addplot[name path=f, domain=-3:3, violet] {-0.5*(x-1)*(x+1)*(x-3)};
            \addplot[name path=g, domain=1:3, violet] {0};
            \addplot[fill opacity=0.5, fill=violet!20] fill between[ 
            of = f and g,
            soft clip={domain=1:3},
            ];
        \end{axis}
    \end{tikzpicture}

    \begin{tikzpicture}
        \begin{axis}[
            defgrid, y=0.8cm, x=1cm, ymin=-2, ymax=2, xmin=-3, xmax=3, xtick={-3,...,3}, ytick={-2,...,2}
            ]
            \addplot[name path=f, domain=-3:3, violet] {-0.5*(x-1)*(x+1)*(x-3)};
            \addplot[name path=g, domain=-1:1, violet] {0};
            \addplot[fill opacity=0.5, fill=violet!20] fill between[ 
            of = f and g,
            soft clip={domain=-1:1},
            ];
        \end{axis}
    \end{tikzpicture}
\end{center}
\end{multicols}
Von den Integralen, die hier abgebildet sind, hat das linke also einen positiven Wert (da die Fläche oberhalb der 
$x$-Achse liegt) und das rechte einen negativen (da die Fläche unter der $x$-Achse liegt).

Wenn wir ein Integral berechnen wollen, müssen wir das Integral zunächst einmal beschreiben:
\begin{enumerate}
    \item Welche Funktion begrenzt die Fläche, die wir berechnen wollen?
    \item In welchem Bereich möchten wir das Integral berechnen?
\end{enumerate}
\begin{example}{}
    \parpic[r]{
    \begin{tikzpicture}
        \begin{axis}[
            defgrid, y=0.15cm, x=0.5cm, ymin=0, ymax=25, xmin=-5, xmax=5, xtick={-4,-2,...,4}, ytick={5,10,15,20,25}
            ]
            \addplot[name path=f, domain=-5:5, violet] {20};
            \addplot[name path=g, domain=-5:5, violet] {0};
            \addplot[fill opacity=0.5, fill=violet!20] fill between[ 
            of = f and g,
            soft clip={domain=0:4},
            ];
        \end{axis}
    \end{tikzpicture}
    }
    Als wir im letzten Beispiel die Fläche zwischen dem Graphen $v(t)$ und der $x$-Achse berechnet haben, haben wir
    als linke Grenze $0$ und als rechte Grenze $4\,\text{h}$ festgelegt. Das sieht man besser, wenn wir einmal einen 
    größeren Bereich im Koordinatensystem darstellen (siehe rechts). Offensichtlich haben wir nicht die \emph{gesamte} 
    Fläche unter dem Graphen ausgerechnet, denn die geht nach links und rechts noch beliebig weiter. Wir haben uns
    stattdessen auf einen bestimmten Bereich festgelegt.
\end{example}
Die Fläche zwischen einem Funktionsgraphen und der $x$-Achse lässt sich in der Regel nach links und rechts beliebig
fortsetzen. Deshalb ist es wichtig, eine linke und eine rechte Grenze des Integrals anzugeben, um eine ganz bestimmte
Fläche festzulegen, die ausgerechnet werden kann. Haben wir eine Funktion $f$ festgelegt, deren Integral wir berechnen
möchten sowie eine linke Grenze $a$ und eine rechte Grenze $b$ festgelegt, so schreiben wir das entsprechende Integral 
als
\[\int_a^bf(x)\diff x.\]
(\enquote{\emph{das Integral im Bereich von $a$ bis $b$ über $f(x)\diff x$}}). Das \enquote{$x$} bei der Schreibweise 
\enquote{$\diff x$} am Ende beschreibt vereinfacht gesagt den Namen 
der Achse, die unsere Fläche nach unten beschränkt (unsere Fläche wird von der $x$-Achse beschränkt). Das Zeichen $\int$
ist das Zeichen für Integrale.

\begin{example}{}
    Wenn wir im letzten Beispiel einmal die Einheiten ignorieren, so haben wir das Integral
    \[\int_0^4v(t)\diff t=\int_0^420\diff t=80\]
    ausgerechnet. Dass hier am Ende ein \enquote{$dt$} steht, kommt daher, dass in unserem Beispiel die $x$-Achse die
    Zeit beschreibt, die vergeht -- und diese wird in der Regel mit $t$ statt mit $x$ beschrieben. Wir haben hier also
    eigentlich eine $t$-Achse und keine $x$-Achse. Abgesehen davon
    ändert dies aber überhaupt nichts an unserem Integral.
\end{example}

Damit du ein besseres Gefühl dafür bekommst, welche Integrale zu welchen Flächen gehören, schauen wir uns zum Ende dieses
Abschnitts noch zwei weitere Beispiele für Flächen an, die wir als Integral aufschreiben und berechnen können.
\begin{example}{}
    \parpic[r]{
    \begin{tikzpicture}
        \begin{axis}[
            defgrid, y=1cm, x=1cm, ymin=0, ymax=4, xmin=-3, xmax=3, xtick={-3,...,3}, ytick={1,...,4}
            ]
            \addplot[name path=f, domain=-3:3, violet] {0.5*x+1};
            \addplot[name path=g, domain=-3:3, violet] {0};
            \addplot[fill opacity=0.5, fill=violet!20] fill between[ 
            of = f and g,
            soft clip={domain=0:2},
            ];
        \end{axis}
    \end{tikzpicture}
    }
    Rechts zu sehen ist der Graph der Funktion $f(x)=\frac{1}{2}x+1$. Wir möchten die markierte Fläche berechnen. Zu
    welchem Integral gehört diese Fläche?

    \picskip{6}
    Zunächst einmal siehst du, dass die Fläche links bei der $0$ und rechts bei der $2$ endet. Damit kennen wir schon
    mal die linke und die rechte Grenze des Integrals. Das sind die Zahlen, die wir unten und oben an das Integralzeichen
    schreiben müssen: $\displaystyle \int_0^2$ (die linke Grenze steht unten, die rechte oben). Nun fehlt noch die Information,
    welche Funktion unsere Fläche beschränkt, also $f$. Wir können das eingezeichnete Integral also als
    \[\int_0^2f(x)\diff x=\int_0^2\frac{1}{2}x+1\diff x\]
    beschreiben. Um diese Fläche zu berechnen, können wir sie zum Beispiel in das Rechteck aus den unteren beiden 
    Kästchen und das Dreieck darüber aufteilen. Dann hat das Dreieck eine Fläche von 
    $\frac{g\cdot h}{2}=\frac{2\cdot 1}{2}=1$ und das Rechteck hat eine Fläche von $2\cdot 1=2$. Es gilt also
    \[\int_0^2\frac{1}{2}x+1\diff x=3.\]
\end{example}
\begin{example}[ex:integrale-unter-achse]{}
    \parpic[r]{
    \begin{tikzpicture}
        \begin{axis}[
            defgrid, y=1cm, x=1cm, ymin=-2, ymax=2, xmin=-3, xmax=3, xtick={-3,...,3}, ytick={-2,...,2}
            ]
            \addplot[name path=f, domain=-3:3, violet] {0.5*x};
            \addplot[name path=g, domain=-3:3, violet] {0};
            \addplot[fill opacity=0.5, fill=violet!20] fill between[ 
            of = f and g,
            soft clip={domain=-2:2},
            ];
        \end{axis}
    \end{tikzpicture}
    }
    Das rechts abgebildete Integral besteht aus zwei gleich großen Dreiecken, die beide eine Breite von $2$ und eine 
    Höhe von $1$ haben. Damit ist ihr Flächeninhalt jeweils
    \[\frac{2\cdot 1}{2}=1.\]
    Weil das linke Dreieck allerdings unter der $x$-Achse liegt, wird sein Flächeninhalt für das eingezeichnete Integral
    negativ gewichtet. Für die rechts dargestellte Funktion $f$ gilt also
    \[\int_{-2}^2f(x)\diff x=-\colorbrace{1}{\text{Linkes Dreieck}}+\colorobrace{1}{\text{Rechtes Dreieck}}=0.\]
    Du musst also aufpassen, wenn du versuchst, Flächen unter der $x$-Achse mit Integralen zu beschreiben, weil du für
    diese Flächen nicht vergessen darfst, dass ihre Fläche für das Integral negativ gezählt wird.
\end{example}
\begin{nutshell}{Integrale}
    \parpic[r]{
    \begin{tikzpicture}[scale=0.7]
        \begin{axis}[
            defgrid, y=1cm, x=1cm, ymin=0, ymax=4, xmin=-3, xmax=3, xtick={-3,...,3}, ytick={1,...,4}
            ]
            \addplot[name path=f, domain=-3:3, violet] {0.5*x+1};
            \addplot[name path=g, domain=-3:3, violet] {0};
            \addplot[fill opacity=0.5, fill=violet!20] fill between[ 
            of = f and g,
            soft clip={domain=0:2},
            ];
            \node at (1.75,3) {$\displaystyle\int_0^2f(x)\diff x$};
            \draw[-latex,very thick] (1.75,2.5) to[bend left] (1.2,0.8);
        \end{axis}
    \end{tikzpicture}
    }
    Für eine Funktion $f$, deren Graph im Intervall $[a,b]$ oberhalb der $x$-Achse verläuft, beschreibt das \textbf{Integral}
    \[\int_a^bf(x)\diff x\]
    \picskip{3}
    (gelesen \enquote{\emph{das Integral im Bereich von $a$ bis $b$ über $f(x)$ dx}}) die \emph{Fläche zwischen dem Graphen von $f$ und der $x$-Achse}, die links durch $x=a$ und rechts durch $x=b$ 
    beschränkt ist. Flächen unterhalb der $x$-Achse werden bei Integralen negativ gewichtet.
\end{nutshell}
\end{document}