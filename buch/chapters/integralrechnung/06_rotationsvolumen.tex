\documentclass[../../main.tex]{subfiles}

\begin{document}

Der Graph einer Funktion schließt, wie du bereits weißt, mit der $x$-Achse eine Fläche ein. In diesem Abschnitt
untersuchen wir eine bestimmte Art von dreidimensionalen Körpern, deren Volumen sich mithilfe von Integralen ausrechnen
lässt.

\parpic[r]{
    \begin{tikzpicture}
        \pgfmathsetmacro{\u}{1}
        \begin{axis}[defgrid,
            set layers,
            x={(1*\u cm,0.0cm)}, 
            y={(0cm,\u cm)},
            z={(\u*0.3535cm,\u*0.3535cm)},  
            axis lines=center,
            hide z axis,
            xmin=0,xmax=5,ymin=-2,ymax=2,xtick={1,...,5},ytick={-2,...,2}
        ]
            \addplot3[opacity=0.2,surf,shader=flat,on layer=axis background,
            samples=20,
            domain=0:5,y domain=0:2*pi,
            z buffer=sort]
            (x,{2*cos(deg(y))}, {2 * sin(deg(y))});
            \draw[latex-latex,thick] (5,0,0) -- node[rotate=60,left,above] {\scriptsize $2$} (5,{2*cos(75)},{2*sin(75)});
            \draw[latex-latex,thick] (0,0,0) -- node[rotate=60,left,above] {\scriptsize $2$} (0,{2*cos(75)},{2*sin(75)});
        \end{axis}
    \end{tikzpicture}
}
Rechts siehst du einen Zylinder mit dem Radius 2, der so um die $x$-Achse gelegt ist, dass seine Mitte genau die 
$x$-Achse ist. Wie bei jedem Zylinder ist sein Radius natürlich überall gleich. Die linke Kreisfläche hat also ebenso
wie die rechte Kreisfläche den Radius 2.

Wenn wir nun das Volumen dieses Zylinders berechnen möchten, dann müssen wir die Fläche der Kreise mit der Breite des
Zylinders multiplizieren. Wenn $K$ die Fläche der Kreise und $b$ die Breite ist, lautet die Volumenformel also
\[V=K\cdot b=\pi r^2\cdot b,\]
denn die Formel für die Kreisfläche bei einem Radius von $r$ ist $K=\pi r^2$. In unserem Fall ist der Zylinder 5 Einheiten
breit und hat einen Radius von 2. Das Volumen ist also $\pi\cdot 2^2\cdot 5=20\pi$.
\if 0
\parpic[l]{
    \begin{tikzpicture}
        \pgfmathsetmacro{\u}{1}
        \begin{axis}[defgrid,
            set layers,
            x={(1*\u cm,0.0cm)}, 
            y={(0cm,\u cm)},
            z={(\u*0.3535cm,\u*0.3535cm)},  
            axis lines=center,
            hide z axis,
            xmin=0,xmax=5,ymin=-2,ymax=2,xtick={1,...,5},ytick={-2,...,2}
        ]
            \addplot3[opacity=0.2,surf,shader=flat,on layer=axis background,
            samples=20,
            domain=0:5,y domain=0:2*pi,
            z buffer=sort]
            (x,{sin(deg(x))*cos(deg(y))}, {sin(deg(x)) * sin(deg(y))});
        \end{axis}
    \end{tikzpicture}
}
Wie bereits erwähnt, ist der Radius des Zylinders überall 2. Die Oberfläche hat also überall einen Abstand von 2 zur
$x$-Achse. 

Wir werden in diesem Abschnitt lernen, ganz ähnlich auch das Volumen von Körpern zu berechnen, deren
Radius nicht überall gleich ist, sondern sich auf dem Weg von links nach rechts ändert. Einen solchen Körper siehst du
zum Beispiel links. Sowohl ganz links als auch bei $x=\pi$ hat der Körper einen Radius von $0$. Dazwischen ist der
Radius jedoch offensichtlich größer.
Wie groß der Radius dieses Körpers ist, hängt also davon ab, wie weit links oder rechts du hinschaust.

\begin{example}{}
    \parpic[r]{
    \begin{tikzpicture}
        \pgfmathsetmacro{\u}{1}
        \begin{axis}[defgrid,
            set layers,
            x={(1*\u cm,0.0cm)}, 
            y={(0cm,\u cm)},
            z={(\u*0.3535cm,\u*0.3535cm)},  
            axis lines=center,
            hide z axis,
            xmin=0,xmax=5,ymin=-2,ymax=2,xtick={1,...,5},ytick={-2,...,2}
        ]
            \addplot3[opacity=0.2,surf,shader=flat,on layer=axis background,
            samples=20,
            domain=0:5,y domain=0:2*pi,
            z buffer=sort]
            (x,{(0.3333*x+0.3333)*cos(deg(y))}, {(0.3333*x+0.3333) * sin(deg(y))});
            \draw[latex-latex,thick] (5,0,0) -- node[rotate=-75,right,above] {\scriptsize $f(5)$} (5,{2*cos(-30)},{2*sin(-30)});
        \end{axis}
    \end{tikzpicture}
    }
    Der rechts eingezeichnete Körper ist dadurch entstanden, dass für jeden Wert für $x$ ein Kreis um die $x$-Achse
    gezeichnet worden ist. Ähnlich wie beim Zylinder weiter oben liegt er also so, dass die $x$-Achse sich in seiner
    Mitte befindet. 
    
    \picskip{3}
    Der Radius des Körpers ist allerdings nicht überall gleich. Schaust du dir zum Beispiel die Kreisscheibe
    an, die die $x$-Koordinate $0$ hat, dann erhältst du einen Kreis mit dem Radius $\frac{1}{3}$. Schaust du hingegen
    weiter rechts die Kreisscheibe mit $x$-Koordinate $5$ an, so hat diese einen Radius von $2$.
    
    Der Radius dieser Kreisscheibe hängt somit von $x$ ab. Tatsächlich ist der rechts abgebildete Körper so gezeichnet,
    dass der Radius des Kreises für ein bestimmtes $x$ genau $\frac{1}{3}x+\frac{1}{3}$ ist. Du kannst den Radius
    auch als Funktion ansehen, in die du einen bestimmten Wert für $x$ einsetzt und den Radius des Körpers an dieser
    Stelle bekommst:
    \[\texttt{radius}(x)=\frac{1}{3}x+\frac{1}{3}\]
    \parpic[l]{
    \begin{tikzpicture}
        \begin{axis}[
            defgrid, y=1cm, x=1cm, ymin=-2, ymax=2, xmin=0, xmax=5.5, xtick={1,...,5}, ytick={-2,...,2},samples=50
            ]
            \addplot[name path=poly, domain=0:5, violet] {0.3333*x+0.3333};
            \addplot[name path=poly, domain=0:5, violet, dashed] {-0.3333*x-0.3333};
            \draw[violet,dashed] (0,.3333) arc[start angle=30,end angle=-30,radius=0.6667] (0,-.3333);
            \draw[violet,dashed] (0,-.3333) arc[start angle=210,end angle=150,radius=0.6667] (0,.3333);
            \draw[violet,dashed] (5,2) arc[start angle=30,end angle=-30,radius=4] (5,-2);
            \draw[violet,dashed] (5,-2) arc[start angle=210,end angle=150,radius=4] (5,2);
        \end{axis}
    \end{tikzpicture}
    }
    Du kannst dir das auch so vorstellen, dass du den Graphen der Funktion $f(x)=\frac{1}{3}x+\frac{1}{3}$ zeichnest
    und diesen um die $x$-Achse rotieren lässt, indem du einen Faden der Länge $f(x)$ an einem Punkt auf der
    $x$-Achse befestigst und dann wie mit einem Zirkel einen Kreis zeichnest. 
    
    \picskip{2}
    Wenn du das für alle Werte für $x$ machst, dann kommt genau der Körper, den du siehst, dabei heraus. Der Körper
    entsteht also, wenn du den Graphen der Funktion $f$ um die $x$-Achse rotieren lässt. Deswegen nennt man einen
    solchen Körper auch \textbf{Rotationskörper}.
\end{example}
Wenn wir den Graphen einer Funktion $f$ nehmen und um die $x$-Achse rotieren lassen, dann entsteht ein Körper, der wie
in den vorherigen Abbildungen aussieht. Ein solcher Körper wird \textbf{Rotationskörper} genannt.
Lassen wir zum Beispiel einen Punkt $\coord{x}{f(x)}$ um die $x$-Achse rotieren, 
so erhalten wir einen Kreis mit dem Radius $f(x)$, da der Punkt genau $f(x)$ Einheiten von der $x$-Achse entfernt ist.
\begin{example}{}
    Lässt man die Funktion $f(x)=\frac{x^2}{8}$ um die $x$-Achse rotieren, dann erhält man den links abgebildeten
    Rotationskörper. Der rechts abgebildete Körper gehört zur Funktion $g(x)=\cos(x)$.
    \begin{multicols}{2}
        \begin{center}\begin{tikzpicture}
            \pgfmathsetmacro{\u}{1}
            \begin{axis}[defgrid,
                set layers,
                x={(1*\u cm,0.0cm)}, 
                y={(0cm,\u cm)},
                z={(\u*0.3535cm,\u*0.3535cm)},  
                axis lines=center,
                hide z axis,
                xmin=0,xmax=4,ymin=-2,ymax=2,xtick={1,...,4},ytick={-2,...,2}
            ]
                \addplot3[opacity=0.2,surf,shader=flat,on layer=axis background,
                samples=20,
                domain=0:4,y domain=0:2*pi,
                z buffer=sort]
                (x,{0.125*x^2*cos(deg(y))}, {(0.125*x^2) * sin(deg(y))});
            \end{axis}
        \end{tikzpicture}

        \begin{tikzpicture}
            \pgfmathsetmacro{\u}{1}
            \begin{axis}[defgrid,
                set layers,
                x={(1*\u cm,0.0cm)}, 
                y={(0cm,\u cm)},
                z={(\u*0.3535cm,\u*0.3535cm)},  
                axis lines=center,
                hide z axis,
                xmin=0,xmax=4,ymin=-2,ymax=2,xtick={1,...,4},ytick={-2,...,2}
            ]
                \addplot3[opacity=0.2,surf,shader=flat,on layer=axis background,
                samples=20,
                domain=0:4,y domain=0:2*pi,
                z buffer=sort]
                (x,{cos(deg(x))*cos(deg(y))}, {cos(deg(x)) * sin(deg(y))});
            \end{axis}
        \end{tikzpicture}\end{center}
    \end{multicols}
\end{example}
Um das Volumen solcher Körper auszurechnen, versuchen wir, uns an der Formel für den Zylinder zu orientieren, die wir
zu Beginn dieses Abschnitts verwendet haben. Wie bereits bei der Berechnung gewöhnlicher Integrale ist unser Ziel, den
Rotationskörper, dessen Volumen wir suchen, in viele kleinere Körper aufzuteilen, deren Volumen wir problemlos berechnen 
können.
\begin{example}{}
    In den beiden oberen Abbildungen siehst du, wie der Rotationskörper, der durch die Funktion $f(x)=\frac{x^2}{8}$ aus
    dem letzten Beispiel erzeugt wird, durch viele kleinere Zylinder ausgefüllt wurde. Die Zylinder liegen so im
    Rotationskörper, dass ihr Radius am linken Rand dem Radius des Rotationskörpers entspricht.
    \begin{center}
        \begin{tikzpicture}
            \pgfmathsetmacro{\u}{1}
            \begin{axis}[defgrid,
                set layers,
                x={(1*\u cm,0.0cm)}, 
                y={(0cm,\u cm)},
                z={(\u*0.3535cm,\u*0.3535cm)},  
                axis lines=center,
                hide z axis,
                xmin=0,xmax=4,ymin=-2,ymax=2,xtick={1,...,4},ytick={-2,...,2}
                ]
                \addplot3[opacity=0.05,surf,shader=flat,on layer=axis background,
                    samples=20,
                    domain=0:4,y domain=0:2*pi,
                    z buffer=sort]
                    (x,{0.125*x^2*cos(deg(y))}, {(0.125*x^2) * sin(deg(y))});
                \addplot3[opacity=0.2,surf,shader=flat,on layer=axis background,
                samples=20,
                domain=0:4,y domain=0:2*pi,
                z buffer=sort]
                (x,{(0.125*4 * (x > 2) * (x < 3) + 0.125 * (x < 2) * (x > 1) + 0.125 * 9 * (x > 3))*cos(deg(y))}, 
                {(0.125*4 * (x > 2) * (x < 3) + 0.125 * (x < 2) * (x > 1) + 0.125 * 9 * (x > 3)) * sin(deg(y))});
            \end{axis}
        \end{tikzpicture}
        \begin{tikzpicture}
            \pgfmathsetmacro{\u}{1}
            \begin{axis}[defgrid,
                set layers,
                x={(1*\u cm,0.0cm)}, 
                y={(0cm,\u cm)},
                z={(\u*0.3535cm,\u*0.3535cm)},  
                axis lines=center,
                hide z axis,
                xmin=0,xmax=4,ymin=-2,ymax=2,xtick={1,...,4},ytick={-2,...,2}
                ]
                \addplot3[opacity=0.05,surf,shader=flat,on layer=axis background,
                    samples=20,
                    domain=0:4,y domain=0:2*pi,
                    z buffer=sort]
                    (x,{0.125*x^2*cos(deg(y))}, {(0.125*x^2) * sin(deg(y))});
                \addplot3[opacity=0.2,surf,shader=flat,on layer=axis background,
                samples=20,
                domain=0:4,y domain=0:2*pi,
                z buffer=sort]
                (x,{(0.125*0.25*(x<0.5) + 0.125*2.25*(x > 1.5)*(x < 2)+0.125*6.25*(x > 2.5)*(x < 3)+0.125*12.25*(x > 3.5)*(x < 4) + 0.125*4 * (x > 2) * (x < 2.5) + 0.125 * (x < 1.5) * (x > 1) + 0.125 * 9 * (x > 3) * (x < 3.5))*cos(deg(y))}, 
                {(0.125*0.25*(x<0.5) + 0.125*2.25*(x > 1.5)*(x < 2)+0.125*6.25*(x > 2.5)*(x < 3)+0.125*12.25*(x > 3.5)*(x < 4) + 0.125*4 * (x > 2) * (x < 2.5) + 0.125 * (x < 1.5) * (x > 1) + 0.125 * 9 * (x > 3) * (x < 3.5)) * sin(deg(y))});
            \end{axis}
        \end{tikzpicture}
    \end{center}
    Im linken Bild haben wir uns für vier Zylinder der Breite 1 entschieden (den ganz linken Zylinder siehst du nicht,
    da sein Radius 0 ist) und im rechten Bild ist die Breite der Zylinder $\frac{1}{2}$. Damit der kleinste sichtbare
    Zylinder im linken Bild die Funktion berührt, 
\end{example}
\parpic[r]{
    \begin{tikzpicture}
        \begin{axis}[
        defgrid, y=0.9cm, x=0.9cm, ymin=-0.5, ymax=4.2, xmin=0, xmax=4, xtick={1,2,3,4}, ytick={1,...,4}
        ]
            \addplot[name path=poly, domain=0:4, violet] {0.2*x^2+1};
            \pgfplotsinvokeforeach {2,3}{
                \filldraw[draw=violet,fill=violet!20, opacity=0.5] (#1,0) -- (#1,0.2*#1*#1+1) -- (#1 + 1, 0.2*#1*#1+1) -- (#1 + 1, 0) -- cycle;
            }
            \draw[latex-latex,very thick] (1.9,0) -- node[left] {$f(2)$} (1.9,1.8);
        \end{axis}
    \end{tikzpicture}
}
Um das Integral $\displaystyle\int_a^bf(x)\diff x$ für eine Funktion $f$ auszurechnen, haben wir die gesuchte Fläche in
Rechtecke aufgeteilt. Mit der Summe der Rechtecke haben wir die tatsächliche Fläche angenähert. Die Höhe eines Rechtecks, 
das wir so erhalten, hängt vom Funktionswert an der Stelle ab, an der das Rechteck links beginnt. Ein Rechteck, das 
links bei $x=2$ beginnt, hat zum Beispiel die Höhe $f(2)$. Multiplizieren wir diesen Wert mit der Breite des Rechtecks,
erhalten wir die Rechteckfläche. Die Summe der Rechteckflächen entspricht ungefähr der gesuchten Fläche. Dieses
Vorgehen haben wir im Abschnitt über Riemann-Integrale verwendet, um Integrale zu definieren. 

\parpic[l]{
    \begin{tikzpicture}
        \pgfmathsetmacro{\u}{1}
        \begin{axis}[defgrid,
            set layers,
            x={(1*\u cm,0.0cm)}, 
            y={(0cm,\u cm)},
            z={(\u*0.3535cm,\u*0.3535cm)},  
            axis lines=center,
            hide z axis,
            xmin=0,xmax=4,ymin=-2,ymax=2,xtick={1,...,4},ytick={-2,-1.5,...,2},yticklabels={-4,...,4}
            ]
            \addplot3[opacity=0.05,surf,shader=flat,on layer=axis background,
                samples=10,
                domain=0:4,y domain=0:2*pi,
                z buffer=sort]
                (x,{(0.1*x^2+0.5)*cos(deg(y))}, {(0.1*x^2+0.5) * sin(deg(y))});
            \addplot3[opacity=0.2,surf,shader=flat,on layer=axis background,
            samples=10,
            domain=0:4,y domain=0:2*pi,
            z buffer=sort]
            (x,{(0.9 * (x > 2) * (x < 3) + 1.4 * (x > 3))*cos(deg(y))}, 
            {(0.9 * (x > 2) * (x < 3) + 1.4 * (x > 3)) * sin(deg(y))});
        \end{axis}
    \end{tikzpicture}
}\fi
Wir können dieses Vorgehen nun so anpassen, dass wir damit auch das Volumen eines Rotationskörpers berechnen können.
Dafür müssen wir statt Rechteckflächen die Volumina von kleinen Zylindern wie links dargestellt berechnen und addieren. 
Die Zylinder zeichnen wir genauso wie die Rechtecke so ein, dass sie am linken Rand den Funktionsgraphen berühren.
Zum Beispiel hat der linke Zylinder, der links durch $x=2$ begrenzt ist, einen Radius 
von $f(2)$. Wenn wir nun die Volumina dieser kleinen Zylinder addieren, gibt uns das eine Näherung für das gesuchte
Gesamtvolumen. Um das Volumen des linken Zylinders auszurechnen, müssen wir seine Grundfläche mit seiner Breite 
multiplizieren. Die Grundfläche ist ein Kreis mit dem Radius $f(2)$ und hat 
daher den Flächeninhalt $\pi r^2=\pi f(2)^2$ (denn der Radius 
entspricht ja dem Funktionswert, den die Funktion am linken Rand des Zylinders annimmt). 

Wenn unsere Rechtecke die
Höhe $\pi f(x)^2$ statt der Höhe $f(x)$ hätten, dann würde ihr \enquote{Flächeninhalt} genau dem Volumen der Zylinder 
entsprechen. Wenn wir das Integral für die Funktion $\pi f(x)^2$ in Rechtecke aufteilen, erhalten wir also Rechtecke
mit der gesuchten Höhe. Wir können also so tun als hätten wir es mit Zylindern statt Rechtecken zu tun, wenn wir das
Integral
\[\int_a^b\pi f(x)^2\diff x\]
ausrechnen und damit das Volumen des Rotationskörpers, der durch die Funktion $f(x)$
erzeugt wird, im Bereich zwischen $a$ und $b$ ausrechnen.

\begin{theorem}{Volumen eines Rotationskörpers}
    Der Rotationskörper mit den Grenzen $a$ und $b$ einer Funktion $f\colon[a,b]\rightarrow\R$ hat das Volumen
    \[\int_a^b\pi f(x)^2\diff x=\pi\cdot \int_a^bf(x)^2\diff x.\]
\end{theorem}
\begin{example}{}
    Anwendung der Formel
\end{example}

\end{document}