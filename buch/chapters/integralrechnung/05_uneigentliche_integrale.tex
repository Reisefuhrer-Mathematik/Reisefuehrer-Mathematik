\documentclass[../../main.tex]{subfiles}

\begin{document}
Mit unserem jetzigen Wissen können wir Integrale vieler (einfacher) Funktionen berechnen und sie verwenden, um Flächen 
zu berechnen, die Funktionen mit der $x$-Achse einschließen. Wir sind aber leider immer auf ein bestimmtes Intervall 
eingeschränkt, es ist uns also nicht möglich, Flächen auch in der Unendlichkeit zu betrachten. Es geht um die folgende 
Situation:
\begin{center}
    \begin{tikzpicture}
        \begin{axis}[
            defgrid, y=1cm, x=1cm, ymin=0, ymax=4, xmin=0, xmax=6, xtick={1,...,6}, ytick={1,...,4}
            ]
            \addplot[name path=poly, domain=0.1:6, violet] {4/x};
            \addplot[name path=line, domain=1:6, violet] {0};
            \addplot[fill opacity=0.5, fill=violet!20] fill between[ 
            of = poly and line,
            soft clip={domain=1:6},
            ];
        \end{axis}
    \end{tikzpicture}
\end{center}

Wir wollen die Fläche, die die Funktion $f(x)=\frac{4}{x}$ mit der $x$-Achse einschließt, berechnen. Und zwar die 
gesamte Fläche, beginnend bei $x=2$. Das dazugehörige Intervall lautet $[2,\infty[$. Als Integral sieht das 
folgendermaßen aus:
\[\int_2^\infty \frac{4}{x}\diff x=\Bigl[4\ln{x}\Bigr]_2^\infty=4\ln{\infty}-4\ln{2}\]
An dieser Stelle haben wir ein Problem: Wir müssen den Wert $\ln{\infty}$ berechnen. Dieses Problem tritt generell 
immer auf, wenn wir als eine Intervallgrenze einen Wert wählen, bei dem die Funktion, die wir betrachten, undefiniert 
ist. Häufig is das $\pm\infty$, aber es kann sich bei diesen Stellen auch um Definitionslücken handeln, zum Beispiel 
um 0 im Beispiel mit $f(x)=\frac{4}{x}$ (also oft Stellen, bei denen man durch 0 teilen würde). Für die beiden Fälle 
helfen wir uns mit einer Erweiterung unseres Integralbegriffs durch eine weitere Definition:
\begin{definition}{Uneigentliches Integral}
Für eine stetige Funktion $f:[a,\infty[\rightarrow\mathbb{R}$, die auf allen Intervallen $[a,b]$, $b\in\mathbb{R}$ 
integrierbar ist, setzt man
\[\int_a^\infty f(x)\diff x:=\lim\limits_{b\rightarrow\infty}\int_a^b f(x)\diff x\]
falls dieser Grenzwert existiert.\\

Für eine stetige Funktion $g:[a,b[\rightarrow\mathbb{R}$, die auf allen Intervallen $[a,b-\epsilon]$ für alle 
$\epsilon>0$ integrierbar ist, setzt man
\[\int_a^b g(x)\diff x:=\lim\limits_{\epsilon\rightarrow 0}\int_a^{b-\epsilon} g(x)\diff x\]
falls der Grenzwert existiert.
\end{definition}
\begin{example}{}
Im obigen Beispiel gilt \[\int_2^\infty \frac{4}{x}\diff x=\lim\limits_{b\rightarrow\infty}\int_2^b \frac{4}{x}\diff x=\lim\limits_{b\rightarrow\infty}\Bigl[4\ln{|x|}\Bigr]_2^b=\lim\limits_{b\rightarrow\infty}4\underbrace{\ln{b}}_{\rightarrow\infty}-4\ln{2}=\infty\]
\end{example}
\begin{example}{}
Für $f(x)=e^{-x}$ berechnet man das Integral $\displaystyle\int_1^\infty e^{-x}\diff x$ folgendermaßen: 
\[\int_1^\infty e^{-x}\diff x=\lim\limits_{b\rightarrow\infty}\int_1^b e^{-x}\diff x=\lim\limits_{b\rightarrow\infty}\Bigl[-e^{-x}\Bigr]_1^b=\lim\limits_{b\rightarrow\infty}\underbrace{-e^{-b}}_{\rightarrow 0}-(-e^{-1})=e^{-1}=\frac{1}{e}\]
\end{example}
\begin{example}{}
Das nachfolgende Integral kann nicht direkt berechnet werden, weil die Funktion für $x=0$ undefiniert ist.
\[\int_0^1\frac{1}{2\sqrt{x}}\diff x=\lim\limits_{\epsilon\rightarrow 0}\int_\epsilon^1 \frac{1}{2\sqrt{x}}\diff x=\lim\limits_{\epsilon\rightarrow 0}\Bigl[\sqrt{x}\Bigr]_\epsilon^1=\lim\limits_{\epsilon\rightarrow 0}\sqrt{1}-\underbrace{\sqrt{\epsilon}}_{\rightarrow 0}=1\]
\end{example}
\end{document}