\documentclass[../../main.tex]{subfiles}

\begin{document}

\subsection{Formale Definition des Riemann-Integrals}
\label{riemannintegral-richtig}
\[\sigma(f,\mathcal{Z})\defas \underbrace{\underbrace{(t_1-t_0)}_{\text{Breite}}\cdot \underbrace{f(t_0)}_{\text{Höhe}}}_{\text{1. Rechteck}}+\underbrace{\underbrace{(t_2-t_1)}_{\text{Breite}}\cdot \underbrace{f(t_1)}_{\text{Höhe}}}_{\text{2. Rechteck}}+\dots+\underbrace{(t_n-t_{n-1})}_{\text{Breite}}\cdot \underbrace{f(t_{n-1})}_{\text{Höhe}}\]    

\end{document}