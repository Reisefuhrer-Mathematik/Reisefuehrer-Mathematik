\documentclass[../../main.tex]{subfiles}

\begin{document}

\subsection*{Formale Definition des Riemann-Integrals}
\label{riemannintegral-richtig}
\[\sigma(f,\mathcal{Z})\defas\underbrace{\underbrace{(t_1-t_0)}_{\text{Breite}}\cdot \underbrace{f(t_0)}_{\text{Höhe}}}_{\text{1. Rechteck}}+\underbrace{\underbrace{(t_2-t_1)}_{\text{Breite}}\cdot \underbrace{f(t_1)}_{\text{Höhe}}}_{\text{2. Rechteck}}+\dots+\underbrace{(t_n-t_{n-1})}_{\text{Breite}}\cdot \underbrace{f(t_{n-1})}_{\text{Höhe}}\]    

\subsection*{Partielle Integration}
\label{partielle-integration}
Wie bereits angekündigt, ist es nicht nur möglich, die Potenzregel umzukehren, sondern wir können das gleiche auch
mit der Produkt- und der Kettenregel tun. In diesem Abschnitt fangen wir mit der Umkehrung der Produktregel an. Die
Beispiele aus dem Abschnitt zu Stammfunktionen und den Übungsaufgaben sind immer so gewählt, dass sich die Funktionen
mit einfachen Mitteln integrieren lassen. Die Stammfunktion war meistens sehr offensichtlich (etwa $\sin x$ als Stammfunktion 
von $\cos x$). Das ist leider nicht immer der Fall.

Wir schauen uns, bevor wir irgendwelche neuen Regeln aufstellen, erstmal ein Beispiel an, bei dem unsere bisherigen Verfahren 
versagen (wir können lediglich durch Raten auf das Ergebnis kommen):
\begin{example}{}
    Wir wollen das Integral $\displaystyle\int_0^\pi x\sin x\diff x$ berechnen. Zwar kennen wir uns mit Integralen 
    mittlerweile gut genug aus, um zu wissen, dass wir eigentlich nur eine Stammfunktion ermitteln müssen, aber die 
    Ermittlung eben dieser ist im Regelfall die eigentliche Herausforderung beim Berechnen von Integralen.

    Wir wissen, dass in der Stammfunktion ein Kosinus vorkommen muss, jedoch ist etwa $-x\cos x$ keine Lösung, da durch
    die Produktregel immer störende Terme hinzukommen. Es ist zum Beispiel
    \[(-x\cos x)'=x\sin x-\cos x.\]
    Eine Stammfunkion zu finden, die dieses Problem löst, ist mit unseren bisherigen Mitteln reines Ausprobieren.
\end{example}

Nun nehmen wir die Produktregel aus dem Kapitel über Differentialrechnung (Seite \pageref{produktregel}) zur Hilfe. Diese lautete
\[(u(x)v(x))'=u(x)v'(x)+u'(x)v(x),\]
wobei $u$ und $v$ zwei differenzierbare Funktionen sind. Um Platz zu sparen, schreiben wir dies kurz als $(uv)'=uv'+u'v$ 
(wir lassen also das Argument \enquote{($x$)} weg).
Durch Subtrahieren von $uv'$ können wir diese Regel 
umschreiben zu
\[u'v=(uv)'-uv'\]
und da wir uns nicht um Ableitungen, sondern um Integrale kümmern, integrieren wir die Gleichung, um
\[\int u'v\diff x=uv-\int uv'\diff x\]
zu erhalten. Nun können wir versuchen, damit weiterzukommen. Der Teil auf der linken Seite ist dabei der, den wir 
suchen: Ein Integral (also eine Stammfunktion) zur Funktion $uv'$. Nun haben wir aber nur eine Funktion $f(x)$ und keine 
Funktionen $u$ oder $v$. Um unsere Funktion in eine Form zu bringen, die wir in die Formel einsetzen können, müssen 
wir sie als ein Produkt zweier Funktionen, die wir $u(x)$ und $v'(x)$ nennen, schreiben.
\begin{example}{}
    Für die Funktion $f(x)=x\sin x$ aus dem letzten Beispiel wählen wir jetzt $v(x)=x$ und $u'(x)=\sin x$. 
    Warum das so herum sinnvoll ist, sehen wir gleich. Wir suchen also
    \[\int_0^\pi f(x)\diff x=\int_0^\pi\colorbrace{x}{v(x)}\colorbrace{\sin x}{u'(x)}\diff x=\int_0^\pi u'(x)v(x)\diff x.\]
\end{example}
Es lässt sich nun die folgende Regel herleiten, mit der wir Integrale von Funktionen ausrechnen können, die das Produkt
mehrerer Funktionen sind.
\begin{theorem}[thm:partielle-integration]{Partielle Integration}
    Es seien $f,g:[a,b]\rightarrow\mathbb{R}$ stetig differenzierbar. Dann gilt
    \[\int_a^b f(x)g'(x)\diff x=\Bigl[f(x)g(x)\Bigr]_a^b-\int_a^b f'(x)g(x)\diff x\]
\end{theorem}
\begin{proof}
    Wir setzen zunächst $F(x)=f(x)g(x)$. Dann gilt aufgrund der Produktregel \[F'(x)=f(x)g'(x)+f'(x)g(x).\] 
    Wir berechnen nun $\displaystyle\int_a^b f(x)g'(x)+f'(x)g(x)\diff x$. Da die Funktion, die wir hier integrieren, 
    genau $F'(x)$ entspricht, ist $F(x)$ eine Stammfunktion dieser Funktion und wir können schreiben:
    \begin{align*}
        &\int_a^b f(x)g'(x)\diff x+\int_a^b f'(x)g(x)\diff x\\
        =&\int_a^b f(x)g'(x)+f'(x)g(x)\diff x\\
        =&\int_a^b F'(x)\diff x\\
        =&\Bigl[F(x)\Bigr]_a^b=\Bigl[f(x)g(x)\Bigr]_a^b
    \end{align*}
    Subtrahieren wir von der obigen Gleichung nun $\displaystyle\int_a^b f'(x)g(x)\diff x$, dann erhalten wir die Formel,
    die wir herleiten wollten:
    \[\int_a^b f(x)g'(x)\diff x=\Bigl[f(x)g(x)\Bigr]_a^b-\int_a^b f'(x)g(x)\diff x\qedhere\]
\end{proof}

\begin{example}{}
    Wir haben die Funktion $f(x)=x\sin x$ aus den letzten Beispielen bereits als ein Produkt $u'(x)v(x)$ wie auf der linken
    Seite der Formel aus Satz \ref{thm:partielle-integration} geschrieben. Alles, was wir nun machen müssen, um das Integral
    \[\int_0^\pi f(x)\diff x=\int_0^\pi\colorbrace{x}{v(x)}\colorbrace{\sin x}{u'(x)}\diff x=\int_0^\pi u'(x)v(x)\diff x.\]
    auszurechnen, ist, unsere Funktionen in die Formel für die partielle Integration einzusetzen. In der Formel kommen 
    neben $v$ und $u'$ auch noch $v'$ und $u$ vor. Diese beiden Funktionen müssen wir also auch noch berechnen. Wir leiten 
    deswegen $v(x)=x$ ab und erhalten $v'(x)=1$. Die Funktion ist durch das Ableiten also einfacher geworden.
    
    Generell kannst du dir als Faustregel merken, dass die meisten Funktionen beim Ableiten einfacher und beim Integrieren 
    komplizierter werden. 
    
    Da $\sin x$ beim Ableiten nicht einfacher wird, haben wir $\sin x$ als $u'(x)$ gewählt, denn diese Funktion müssen wir
    in der Formel von oben nicht ableiten, die Funktion $v(x)=x$, die beim Ableiten einfacher wird, hingegen schon. 
    Nun müssen wir noch $u(x)$ berechnen. Es gilt für $u(x)=-\cos x$, dass $u'(x)=\sin x$. Somit haben wir 
    unsere Funktionen zusammen und können einsetzen:
    \[\int \colorbrace{x\sin x}{vu'=u'v}\diff x=\colorbrace{-\cos x\cdot x}{uv}-\int \colorbrace{-\cos x}{uv'}\diff x\]
    Durch die geschickte Wahl von $u$ und $v$ haben wir unser gesuchtes, kompliziertes Integral nun als eine Summe zweier 
    Teile geschrieben, die wir einfacher berechnen können. Zum einen haben wir den Teil $-x\cos x$, für den wir keine 
    Stammfunktion mehr berechnen müssen. Zum anderen haben wir das Integral \[\int -\cos x\diff x,\] 
    bei dem der Vorfaktor $x$ von $x\sin x$ durch das Ableiten weggefallen ist. Auf diese Weise können wir nun alles 
    ausrechnen und erhalten
    \[\int x\sin x\diff x=-\cos x\cdot x-(-\sin x)=-\cos x\cdot x+\sin x.\]
    Dadurch haben wir die Stammfunktion bestimmt -- das können wir auch noch durch Ableiten nachrechnen:
    \[(-x\cos x+\sin x)'=\colorbrace{-\cos x+x\sin x}{(-x\cos x)'}+\cos x=x\sin x\]
    Schließlich setzen wir die Grenzen $0$ und $\pi$ in das Integral ein:
    \[\int_0^\pi x\sin x\diff x=\Bigl[-x\cos x+\sin x\Bigr]_0^\pi=\pi-0=\pi\]
\end{example}
Wenn du also eine Funktion vor dir hast, die nach dem Produkt zweier Funktionen aussieht, die du aber nicht ohne
Probleme integrieren kannst, dann kannst du versuchen, dieses Problem mit der partiellen Integration zu lösen.

Leider ist diese Methode des Integrierens relativ aufwendig. In manchen Fällen erhältst du sogar erst eine ausreichende
Vereinfachung, wenn du einen der Faktoren mehrfach ableitest. In diesem Fall ist es auch möglich, die partielle Integration 
mehrfach anzuwenden. Man wendet sie also zunächst auf das ursprüngliche Integral und dann auf das Integral auf der 
rechten Seite der Formel an. Außerdem kommt es manchmal vor, dass man durch Produktintegration die Funktion selbst erneut 
auf der rechten Seite erhält, allerdings negativ. In diesem Fall müssen wir eine Gleichung auflösen, wie im folgenden 
Beispiel.

\begin{example}{}
    Wir wollen das Integral
    $\displaystyle\int \sin ^2 x\diff x$
    berechnen. Wir schreiben die Funktion als $u(x)v'(x)$ mit $u(x)=\sin x$ und $v'(x)=\sin x$. Dann gilt $u'(x)=\cos x$ 
    und \mbox{$v'(x)=-\cos x$}. In die Formel für die partielle Integration eingesetzt ergibt sich 
    \[\int \sin ^2 x\diff x=-\sin x\cos x-\int -\cos ^2 x\diff x=-\sin x\cos x+\int \cos ^2 x\diff x~(*)\]
    und wir können das letzte Integral wieder nicht sinnvoll berechnen. Hier nutzen wir die Regel $\sin ^2 x+\cos ^2 x=1$ 
    (siehe Seite \pageref{additionstheoreme}), um die Gleichung
    \[\int \cos ^2 x\diff x=\int (1-\sin ^2 x)\diff x=\int 1\diff x-\int \sin ^2 x\diff x=x-\int \sin ^2 x\diff x.\]
    zu erhalten. Das können wir in die Formel $(*)$ einsetzen:
    \[\int \sin ^2 x\diff x=-\sin x\cos x+x-\int \sin ^2 x\diff x\]
    Jetzt können wir einen Trick anwenden und auf beiden Seiten $\displaystyle\int \sin ^2 x\diff x$ addieren. Dann 
    verschwindet der Term auf der rechten Seite und wir erhalten
    \[2\int \sin ^2 x\diff x=-\sin x\cos x+x~\text{bzw.}~\int \sin ^2 x\diff x=\frac{1}{2}(-\sin x\cos x+x).\]
    Damit haben wir die Funktion $\sin^2 x$ erfolgreich integriert.
\end{example}

\subsection*{Integration durch Substitution}
\label{substitution}
Eine letzte Ableitungsregel, zu der wir eine passende Integrationsregel finden wollen, ist die Kettenregel. Der Effekt der
Kettenregel ist, dass wir nur die äußere Funktion ableiten, das Ergebnis aber mit der Ableitung der inneren Funktion multiplizieren:
\[\Bigl(f\bigl(g(x)\bigr)\Bigr)'=f'(g(x))\cdot g'(x)\]
Falls nun $g(x)$ eine komplizierte Funktion ist, die uns an der Berechnung des Integrals hindern kann, so können wir versuchen, 
sie durch eine einfachere Funktion zu ersetzen. Dafür leiten wir uns die folgende Integrationsregel aus der Kettenregel her.
\begin{theorem}{Substitutionsregel}
Sei $f:[a,b]\rightarrow\mathbb{R}$ stetig und $g:[c,d]\rightarrow[a,b]$ stetig differenzierbar. Dann gilt \[\int_a^b f(g(x))g'(x)\diff x=\int_{g(a)}^{g(b)} f(t)~dt\]
\end{theorem}
\begin{proof}
Es sei $F$ eine Stammfunktion von $f$. Wir können nun mit der Kettenregel die Stammfunktion von $F(g(x))$ bestimmen: 
\[F(g(x))'=F'(g(x))g'(x).\]
Da $F$ eine Stammfunktion von $f$ ist, gilt 
\[F'(g(x))g'(x)=f(g(x))g'(x).\] 
Die Funktion $f(g(x))g'(x)$ ist also die Ableitung von $F(g(x))$ und wir können $F(g(x))$ nun als Stammfunktion zur 
Berechnung des Integrals
$\displaystyle\int_a^b f(g(x))g'(x)\diff x$ verwenden:
\begin{align*}
    &\int_a^b f(g(x))g'(x)\diff x=\Bigl[F(g(x))\Bigr]_a^b=F(g(b))-F(g(a))\\
    =&\Bigl[F(t)\Bigr]_{g(a)}^{g(b)}=\int_{g(a)}^{g(b)} f(t)~dt\qedhere
\end{align*}
\end{proof}
Das Verfahren der Integration durch Substitution können wir generell dann anwenden, wenn in der zu integrierenden 
Funktion ein Term $g(x)$ und dessen Ableitung $g'(x)$ auftauchen. 
\begin{example}{}
    Wir möchten das Integral
    \[\int_a^b\sin x\cos x\diff x\]
    ausrechnen, in dem sowohl die Funktion $g(x)=\sin x$ als auch deren Ableitung $g'(x)=\cos x$ vorkommt. 
    Mit $g(x)=\sin x$ gilt also
    \[\int_a^b\sin x\cos x\diff x=\int_a^bg(x)g'(x)\diff x.\]
\end{example}
Die Substitutionsregel können wir nun anwenden, indem wir künstlich eine Situation produzieren, in der in unserem 
Integral die Funktion
\[f(g(x))\cdot g'(x)\]
steht. Dabei gibt es oft gar keine Verkettung, das heißt die Funktion $f(x)$ fehlt im Integral. Das ist aber kein Problem,
denn mit $f(x)=x$ können wir die Funktion $f(x)$ künstlich in unser Integral aufnehmen, ohne es dabei zu verändern.
\begin{example}{}
    In unserem obigen Beispiel ist $\sin x\cos x=f(\sin x)\cos x$, wenn wir $f(x)=x$ definieren. Dann gilt $g(x)=\sin x$ 
    und wir erhalten genau die Form der linken Seite der Substitutionsregel.
    \[\int_a^b \sin x\cos x\diff x=\int_a^b f(\sin x)\cos x\diff x=\int_{\sin a}^{\sin b} x\diff x=\Bigl[\frac{1}{2}x^2\Bigr]_{\sin a}^{\sin b}\]
    Wir können nun das Integral mit Mitteln, die wir bereits kennen, berechnen.
\end{example}

Eine weitere Anwendung der Substitutionsregel besteht darin, allgemein und unabhängig von Intervallgrenzen Stammfunktionen 
zu berechnen. Hierbei müssen wir aufpassen, weil wir nicht einfach die Integralgrenzen weglassen können (denn diese haben sich
bei der Substitution verändert). Dieses Problem zu lösen, ist aber nur ein einziger weiterer Schritt, den wir nach Anwendung der
Substitutionssregel nicht vergessen dürfen: Wir müssen in der Funktion, die wir als Ergebnis erhalten, eine sogenannte 
Rücksubstitution durchführen, also alle vorkommenden $x$ durch $g(x)$ ersetzen.
\begin{example}{}
    Im letzten Beispiel haben wir herausgefunden, dass wir das Integral
    \[\int_a^b\sin x\cos x\,dx\]
    ausrechnen können, indem wir
    \[\Bigl[\frac{1}{2}x^2\Bigr]_{\sin a}^{\sin b}\]
    berechnen. Hier können wir eine Rücksubstitution durchführen, um die alten Intervallgrenzen wiederherzustellen:
    \[\Bigl[\frac{1}{2}x^2\Bigr]_{\sin a}^{\sin b}=\frac{1}{2}\sin^2b-\frac{1}{2}\sin^2a=\Bigl[\frac{1}{2}\sin^2x\Bigr]_a^b\]
    Jetzt, da wir die Integralgrenzen beibehalten haben, können wir die erhaltene Funktion als Stammfunktion verwenden. 
    Das können wir auch nachrechnen, indem wir die Funktion mit der Produktregel ableiten: 
    \[\Bigl(\frac{1}{2}\sin^2 x\Bigr)'=\Bigl(\frac{1}{2}(\sin x)^2\Bigr)'=2\cdot\frac{1}{2}(\sin x)\cdot\sin'x=\sin x\cos x\]
\end{example}

\subsection*{Beweis zum Hauptsatz der Differential- und Integralrechnung}
\label{hdi-beweis}
Bevor wir beginnen können, den Hauptsatz der Differential- und Integralrechnung zu beweisen, schauen wir uns einmal
das folgende Integral an.
\begin{center}
    \begin{multicols}{2}
        \begin{tikzpicture}
            \begin{axis}[
                defgrid, y=1cm, x=1cm, ymin=0, ymax=4, xmin=0, xmax=6, xtick={1,...,6}, ytick={1,...,4}
                ]
                \addplot[name path=poly, domain=0:6, violet] {4/x};
                \addplot[name path=line, domain=1:5, violet] {0};
                \addplot[fill opacity=0.5, fill=violet!40] fill between[ 
                of = poly and line,
                soft clip={domain=1:5},
                ];
            \end{axis}
        \end{tikzpicture}

        \begin{tikzpicture}
            \begin{axis}[
                defgrid, y=1cm, x=1cm, ymin=0, ymax=4, xmin=0, xmax=6, xtick={1,...,6}, ytick={1,...,4}
                ]
                \addplot[name path=poly, domain=0:6, violet] {4/x};
                \fill[opacity=0.5,violet!40] (1,0) -- (1,1.609) -- (5,1.609) -- (5,0) -- cycle;
            \end{axis}
        \end{tikzpicture}
    \end{multicols}
\end{center}

Wir sehen im Bild links ein Integral, das zur Funktion $f(x)=\frac{4}{x}$ gehört. Das dargestellte Integral ist
\[\int_1^5\frac{4}{x}\diff x.\]
Bei diesem Integral handelt es sich um eine Fläche mit der Breite 4, deren Höhe von links nach rechts immer kleiner wird.
Wenn wir uns auf eine \enquote{mittlere} Höhe festlegen, können wir versuchen, eine gleichgroße Fläche durch ein Rechteck
der Breite 4 zu erhalten -- wie auf der rechten Seite. Dadurch ist die Fläche des Rechtecks links kleiner als die
des tatsächlichen Integrals, rechts dafür aber größer. 

Wir können eine Höhe für das Rechteck ausrechnen, sodass
links genauso viel vom Integral fehlt wie rechts zu viel ist (das wurde für das eingezeichnete Rechteck getan).
Tatsächlich existiert ein solches Rechteck bei jeder stetigen Funktion:

\begin{theorem}{Mittelwertsatz der Integralrechnung}
    Sei $f:[a,b]\rightarrow\mathbb{R}$ eine stetige Funktion. Dann existiert ein $x_0\in[a,b]$ mit \[\int_a^b f(x)\diff x=f(x_0)\cdot(b-a)\]
    \textit{(Das heißt, die Fläche unter einer Kurve im Bereich von $a$ bis $b$ lässt sich durch ein Rechteck mittlerer Höhe beschreiben.)}
\end{theorem}
\begin{proof}
    Da $f$ stetig ist, nimmt $f$ in $[a,b]$ sein Minimum $k$ und sein Maximum $K$ an. Es gilt in diesem Intervall also 
    $k\leq f(x)\leq K$. Deshalb können wir schreiben:
    \[(b-a)k=k\int_a^b 1\diff x=\int_a^b k\diff x\leq\int_a^b f(x)\diff x\leq\int_a^b K\diff x=K\int_a^b 1\diff x=(b-a)K\]
    Das Integral $\displaystyle\int_a^b f(x)\diff x$ liegt also zwischen $(b-a)k$ und $(b-a)K$. Es gibt folglich ein $l$ 
    mit $k\leq l\leq K$, sodass $\displaystyle\int_a^b f(x)\diff x=(b-a)l$. Da $f$ stetig ist, nimmt $f$ alle Werte zwischen 
    $k$ und $K$ im Intervall $[a,b]$ an und folglich auch den Wert $l$. Wir wählen nun ein $x_0\in[a,b]$ so, dass 
    $f(x_0)=l$. Dann gilt
    \[\int_a^b f(x)\diff x=(b-a)l=(b-a)f(x_0)\qedhere\]
\end{proof}
Diese Erkenntnis ist nun ausreichend, um den Satz zu beweisen, für den wir uns eigentlich interessiert haben.
\begin{proof}
    \emph{(vom Hauptsatz der Differential- und Integralrechnung)}
    Wir beweisen den Hauptsatz der Differential- und Integralrechnung in drei Schritten: Als erstes wählen wir eine
    Zahl $t\in\R$ und zeigen, dass 
    die Funktion $F_0$, die definiert ist durch
    \[F_0(x):=\int_t^x f(x)\diff x,\]
    eine Stammfunktion von $f$ ist. Anschließend zeigen wir, dass wir damit Integrale der Funktion $f$ durch 
    Einsetzen der Grenzen $a$ und $b$ in diese Funktion berechnen können. Schließlich müssen wir zeigen, dass wir das 
    auch mit jeder beliebigen anderen Stammfunktion machen können und das gleiche Ergebnis erhalten.
    
    \textbf{Schritt 1.} Um zu zeigen, dass $F_0$ eine Stammfunktion von $f$ ist, leiten wir $F_0$ ab und weisen nach, dass wir 
    tatsächlich $f$ erhalten:
    \begin{align*}
        F_0'(x)&=\lim_{h\rightarrow 0} \frac{F_0(x+h)-F_0(x)}{h}=\lim_{h\rightarrow 0}\frac{1}{h}\cdot\Biggl(\int_t^{x+h} f(x)\diff x-\int_t^x f(x)\diff x\Biggr)\\
        &=\lim_{h\rightarrow 0}\frac{1}{h}\cdot\int_x^{x+h} f(x)\diff x
    \end{align*}
    Der Mittelwertsatz der Integralrechnung besagt, dass es zwischen $x$ und $x+h$ ein $x_0$ geben muss, sodass 
    \[\int_x^{x+h} f(x)\diff x=(x+h-x)f(x_0)=hf(x_0)\]
    gilt. Wir können das Integral also insgesamt umschreiben zu
    \[\lim_{h\rightarrow 0}\frac{1}{h}\cdot\int_x^{x+h} f(x)\diff x=\lim_{h\rightarrow 0}\frac{1}{h}\cdot hf(x_0)=\lim_{h\rightarrow 0} f(x_0)=f(x)\]
    und haben damit gezeigt, dass die Integralfunktion eine Stammfunktion der Funktion $f$ ist.
    
    \textbf{Schritt 2.} Als zweites müssen wir zeigen, dass wir mithilfe dieser Stammfunktion nun das Integral \[\int_a^b f(x)\diff x\] 
    berechnen können. Dafür setzen wir einfach ein:
    \[\int_a^b f(x)\diff x=\int_t^b f(x)\diff x-\int_t^a f(x)\diff x=F_0(b)-F_0(a).\]
    \textbf{Schritt 3.} Schließlich wollen wir zeigen, dass das Integral mit jeder \textit{beliebigen} Stammfunktion von $f$ berechnet 
    werden kann. Dieser Schritt ist nun einfach, da alle Stammfunktionen von $f$ sich lediglich durch einen konstanten 
    Teil $c$ unterscheiden (alle nicht-konstanten Teile wirken sich auf die Ableitung aus). Damit sei $F$ eine beliebige 
    Stammfunktion von $f$ und es gelte $F(x)=F_0(x)+c$. Wie gerade erläutert, lässt sich jede Stammfunktion so schreiben. 
    Nun gilt aber
    \[F(b)-F(a)=F_0(b)+c-(F_0(a)+c)=F_0(b)-F_0(a)\]
    und da wir wissen, dass sich das Integral mit $F_0$ berechnen lässt, können wir es (da der gleiche Wert entsteht) 
    auch mit jeder anderen Stammfunktion $F$ von $f$ berechnen.
    \end{proof}
    
\subsection*{Beweis der Rechenregeln für Integrale (Satz \ref{thm:rechenregeln-integrale})}
\label{das-riemann-integral}
\begin{proof}
    \emph{(von Satz \ref{thm:rechenregeln-integrale})} Wir können diese Rechenregeln nachrechnen, indem wir unsere Definition von Integralen aus Abschnitt 
    \ref{das-riemann-integral} verwenden. Dann erhalten wir die folgenden Rechnungen.
    \begin{enumerate}
        \item Es ist
            \begin{align*}
                &\int_a^bf(x)\diff x\\
                =&\lim_{n\rightarrow\infty}\frac{b-a}{n}\cdot \biggl(f(a)+f\Bigl(a+\frac{b-a}{n}\Bigr)+\dots+f\Bigl(a+(n-1)\frac{b-a}{n}\Bigr)\biggr)\\
                =&\lim_{n\rightarrow\infty}\frac{-(a-b)}{n}\cdot \biggl(f(a)+f\Bigl(a+\frac{b-a}{n}\Bigr)+\dots+f\Bigl(a+(n-1)\frac{b-a}{n}\Bigr)\biggr)\\
                =&\lim_{n\rightarrow\infty}\frac{-(a-b)}{n}\cdot \biggl(f(a)+f\Bigl(\colorbrace{a+\frac{b-a}{n}}{=b-(n-1)\frac{b-a}{n}}\Bigr)
                +\dots+f\Bigl(\colorbrace{a+(n-1)\frac{b-a}{n}}{=b-\frac{b-a}{n}}\Bigr)\biggr)\\
                =&-\lim_{n\rightarrow\infty}\frac{a-b}{n}\cdot \biggl(f(a)+f\Bigl(b+(n-1)\frac{a-b}{n}\Bigr)
                +\dots+f\Bigl(b+\frac{a-b}{n}\Bigr)\biggr)\\
                =&-\colorbrace{\lim_{n\rightarrow\infty}\frac{a-b}{n}\cdot \biggl(f\Bigl(b+(n-1)\frac{a-b}{n}\Bigr)
                +\dots+f\Bigl(b+\frac{a-b}{n}\Bigr)+f(b)\biggr)}{=\int_b^af(x)\diff x}\\
                &-\colorbrace{\lim_{n\rightarrow\infty}\frac{a-b}{n}\cdot \colorbrace{\Bigl(f(a)-f(b)\Bigr)}{\text{konstant}}}{\text{=0}}
            \end{align*}
        \item Setzen wir $c\cdot f(x)$ in unsere Definition von Integralen ein, so erhalten wir
        \begin{align*}
            &\int_a^bc\cdot f(x)\diff x\\
            =&\lim_{n\rightarrow\infty}\frac{b-a}{n}\cdot \biggl(c\cdot f(a)+c\cdot f\Bigl(a+\frac{b-a}{n}\Bigr)+\dots+c\cdot f\Bigl(a+(n-1)\frac{b-a}{n}\Bigr)\biggr)\\
            =&\lim_{n\rightarrow\infty}c\cdot\frac{b-a}{n}\cdot \biggl(f(a)+f\Bigl(a+\frac{b-a}{n}\Bigr)+\dots+f\Bigl(a+(n-1)\frac{b-a}{n}\Bigr)\biggr)\\
            =&c\cdot \colorbrace{\lim_{n\rightarrow\infty}\frac{b-a}{n}\cdot \biggl(f(a)+f\Bigl(a+\frac{b-a}{n}\Bigr)+\dots+f\Bigl(a+(n-1)\frac{b-a}{n}\Bigr)\biggr)}{=\int_a^bf(x)\diff x}\\
            &=c\cdot \int_a^bf(x)\diff x.
        \end{align*}
        \item Wir setzen $f(x)+g(x)$ in unsere Definition für Integrale ein.
        \begin{align*}
            &\int_a^bf(x)+g(x)\diff x\\
            =&\lim_{n\rightarrow\infty}\frac{b-a}{n}\cdot \biggl(f(a)+g(a)+f\Bigl(a+\frac{b-a}{n}\Bigr)+g\Bigl(a+\frac{b-a}{n}\Bigr)+\dots\\
            &+f\Bigl(a+(n-1)\frac{b-a}{n}\Bigr)+g\Bigl(a+(n-1)\frac{b-a}{n}\Bigr)\biggr)\\
        \end{align*}
        Die Summe in der Klammer lässt sich nun auseinanderziehen zu
        \begin{align*}
            &\lim_{n\rightarrow\infty}\frac{b-a}{n}\cdot \biggl(f(a)+f\Bigl(a+\frac{b-a}{n}\Bigr)+\dots+f\Bigl(a+(n-1)\frac{b-a}{n}\Bigr)\biggr)\\
            &+\frac{b-a}{n}\cdot \biggl(g(a)+g\Bigl(a+\frac{b-a}{n}\Bigr)+\dots+g\Bigl(a+(n-1)\frac{b-a}{n}\Bigr)\biggr)\\
            =&\colorbrace{\lim_{n\rightarrow\infty}\frac{b-a}{n}\cdot \biggl(f(a)+f\Bigl(a+\frac{b-a}{n}\Bigr)+\dots+f\Bigl(a+(n-1)\frac{b-a}{n}\Bigr)\biggr)}{=\int_a^bf(x)\diff x}\\
            &+\colorbrace{\lim_{n\rightarrow\infty}\frac{b-a}{n}\cdot \biggl(g(a)+g\Bigl(a+\frac{b-a}{n}\Bigr)+\dots+g\Bigl(a+(n-1)\frac{b-a}{n}\Bigr)\biggr)}{\int_a^bg(x)\diff x}\\
            =&\int_a^bf(x)\diff x+\int_a^bg(x)\diff x\qedhere
        \end{align*}
    \end{enumerate}
\end{proof}

\end{document}