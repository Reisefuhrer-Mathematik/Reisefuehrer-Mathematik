\documentclass[../../main.tex]{subfiles}

\begin{document}

    \todo{Hier und in allen Abschnitten davor: Einheitliche Notation
    von Variablen}
    
    In den ersten beiden Abschnitten vom Zusatzwissen wollen wir Teile von 
    dem was du bereits in den vorherigen Abschnitten gesehen hast, präzisieren.
    Dir wird also vieles bereits bekannt vorkommen. Wir beschäftigen uns zunächst 
    damit, wie Aussagen formal definiert sind danach lernen wir eine 
    leicht abgewandelte Definition der Wahrheitswerte von Aussagen kennen. 
    Der letzte Abschnitt kann unabhängig
    von den vorherigen gelesen werden und handelt unter Anderem von Aussagen mit einer besonderen Struktur.
    

    Du hast bereits zu Anfang dieses Kapitels kennengelernt was Aussagen sind und wie man Aussagen zu neuen
    Aussagen verknüpfen kann. 
    
    \begin{example}{Aussagen}
        Sind $A$ und $B$ Abkürzungen für atomare Aussagen, dann wissen wir bereits,
        dass $A \land B$ eine Aussage ist aber $A \land $ keine Aussage ist.
        Aber was sind die konkreten Regel mit denen wir genau belegen
        können, dass der Ausdruck $A\land B$ eine Aussage ist,
        der Ausdruck $A \land $ aber nicht? 
        Diese Regeln lernen wir gleich kennen und sie nennen sich 
        die \textbf{Syntax der Aussagenlogik}.
    \end{example}

    Wir wollen präzise werden und einmal ganz
    genau festhalten, was in der Mathematik eine Aussage ist und was nicht.
    Dazu werden wir ein Regelwerk kennenlernen, das eindeutig
    definiert, was eine Aussage ist. Dieses Regelwerk nennt man auch die \textbf{Syntax der
    Aussagenlogik}.

    Die Syntax der Aussagenlogik ist mit Klemmbausteinen vergleichbar. Wir haben 
    eine Hand voll Grundbausteine und Muster nach denen wir die Grundbausteine
    kombinieren können. 

    \textbf{Erste Regel der Syntax der Aussagenlogik:}
    Unsere erste Regel bezieht sich auf die Grundbausteine der Aussagenlogik. Die Grundbausteine
    der Syntax der Aussagenlogik nennen sich \textbf{aussagenlogische Variablen}.
    Die erste Regel sagt aus, dass aussagenlogische Variablen, Aussagen sind.
    \\ \textit{Aussagenlogische Variablen sind konzeptionell genau das gleiche, wie atomare Aussagen,
    wie wir sie bereits kennengelernt haben. Sie haben jetzt nur einen anderen Namen.}

    %\todo{Wie sind AL-Form zu erklären? Problem: Wie Verwirrung zu dem Begriff atomare Aussagen
    %vorbeugen? Atomare Aussagen dürfen nicht verwenden werden, weil in der Def. einer
    %atomaren Aussage, bereits das Wort Aussage vorkommt.}
    %\todo{Idee: Auch Konstanten 0,1 einführen (sagen vorher weggelassen), dann 
    %wird Unterschied deutlich}

    \textbf{Zweite Regel der Syntax der Aussagenlogik:}
    Genauso wie man Klemmbausteine nach festen Mustern zusammenstecken 
    kann (Die Noppen eines Steins müssen in die Löcher des anderen gesetzt werden) 
    kann man auch Aussagen verknüpfen.
    
    \begin{example}{}
        Zwei Muster nach denen wir Aussagen zu komplexeren Aussagen verknüpfen
        können sind, dass $(X \land Y)$ sowie $(X \lor Y)$ wieder Aussagen sind,
        falls bereits $X,Y$ Aussagen sind.
        
        Nehmen wir mal an, dass $A,B$ zwei aussagenlogische Variablen sind. Da aussagenlogische Variablen,
        Aussagen sind, können wir diese also nach den beiden obigen
        Mustern verknüpfen.
        Wir wollen $A,B$ nun zu einer komplexeren Aussagen 
        nach diesen zwei Mustern kombinieren.

        \[\begin{array}{ccc s c}\toprule
            \textrm{Muster} & X & Y & \textrm{Verknüpfung} \\\midrule
            (X \lor Y) & A & B & (A \lor B)\\
            (X \land Y) & A & (A \lor B) & (A \land (A \lor B)) \\
            (X \lor Y) & (A \land (A \lor B)) & B & ((A \land (A \lor B)) \lor B)\\\bottomrule
        \end{array}\]

        Wir haben also nur nach zwei Mustern die aussagenlogische Variablen $A,B$
        zu der Aussage $((A \land (A \lor B)) \lor B)$ verknüpft.
    \end{example}

    Die zweite Regel der Syntax der Aussagenlogik sagt aus, dass wir Aussagen zu 
    neuen Aussagen nach fest vorgeschriebenen 
    Mustern verknüpfen können. Sind $A,B$ Aussagen, dann sind die 
    folgenden Ausdrücke auch wieder Aussagen:
    
    \begin{enumerate}
        \item $(\lnot A)$
        \item $(A \lor B)$
        \item $(A \land B)$
        \item $(A \implies B)$
        \item $(A \iff B)$
    \end{enumerate}

    Wir haben jetzt zwei Regeln kennengelernt, nämlich dass aussagenlogische Variablen,
    Aussagen sind und dass Verknüpfungen, wie $A \land B$ wieder Aussagen sind.
    Diese zwei Regeln
    bilden bereits die Syntax der Aussagenlogik.
    Wir können jetzt nur anhand dieser beiden Regeln genau begründen, ob
    ein Ausdruck eine Aussage ist.

    \begin{example}{}
        Wir wollen anhand der beiden Regeln der Syntax der Aussagenlogik schlussfolgern,
        dass $(A \land (B \lor C))$ eine Aussage ist, falls $A,B,C$ 
        aussagenlogische Variablen sind. Wir ziehen folgende Schlüsse:
        \begin{enumerate}
            \item $B,C$ sind aussagenlogische Variablen, also sind $B,C$ Aussagen
            \item Weil $B,C$ Aussagen sind, ist auch $(B \land C)$ eine Aussage
            \item $A$ ist eine aussagenlogische Variable, also ist $A$ eine Aussage
            \item Weil $A$ eine Aussage ist und $(B \land C)$ eine Aussage 
            ist, ist auch $(A \land (B \lor C))$ eine Aussage
        \end{enumerate}
    \end{example}

    Genauso gut können wir auch begründen, falls ein Ausdruck keine Aussage ist.

    \begin{example}{}
        Wir wollen diesmal anhand der Regeln der Syntax der Aussagenlogik zeigen,
        dass der Ausdruck $(A \land )$ aus dem Eingangsbeispiel 
        keine Aussage ist, auch 
        wenn $A$ eine aussagenlogische Variable ist. Wir beobachten folgendes:
        \begin{enumerate}
            \item $(A \land )$ ist keine aussagenlogische Variable
            \item $(A \land )$ ist keine Verknüpfung von 
            Aussagen nach den festgelegten Mustern (wie z.B. $(A \land B)$)
        \end{enumerate}
        Folglich ist $(A \land )$ keine Aussage, weil $(A \land )$ durch keine 
        der zwei Regeln der Syntax der Aussagenlogik begründet werden kann.
    \end{example}

    Wie am Anfang erwähnt, nennen sich die hier vorgestellten Regeln, die 
    Syntax der Aussagenlogik. Dies wird in der nächsten Definition festgehalten.

    \begin{definition}{Syntax der Aussagenlogik}
        \begin{enumerate}

            \item Aussagenlogische Variablen sind Aussagen 
            \item Sind $A,B$ Aussagen, dann sind folgende Ausdrücke auch wieder Aussagen:
                \begin{itemize}
                    \item $(\lnot A)$
                    \item $(A \lor B)$
                    \item $(A \land B)$
                    \item $(A \implies B)$
                    \item $(A \iff B)$
                \end{itemize}

        \end{enumerate}
    \end{definition}

    Du hast bereits den Begriff Variable in einem vorherigen Kapitel kennengelernt: Variablen sind 
    Platzhalter für irgendeine Art von Werten. Wie man dem Namen schon entnehmen 
    kann, handelt es sich bei aussagenlogischen Variablen auch um Variablen, wie 
    du sie bereits kennst. In diesem Falle sind die aussagenlogischen Variablen 
    Platzhalter für beliebige Aussagen. Es sind also Variablen mit einem anderen 
    Wertebereiche als Zahlen. Der Wertebereich besteht hier nämlich aus Aussagen.

    \newpage
    Wir haben uns bis eben mit dem Aufbau von Aussagen beschäftigt. 
    Jetzt wirst du eine neue, etwas andere Definition von Wahrheitswerten 
    kennenlernen, die uns unter Anderem ermöglichen wird Aussagen in zwei Kategorien zu unterteilen.


    Genau wie bei den atomaren Aussagen interessieren wir uns auch bei 
    aussagenlogischen Variablen 
    nicht dafür, welche 
    konkrete Aussage hinter einer aussagenlogischen Variable steckt, 
    sondern nur welchen Wahrheitswert die Aussage hinter der Variable besitzt. 
    Deswegen spricht man auch von dem Wahrheitswert einer 
    aussagenlogischen Variable.
    Sagt man beispielsweise, die Variable $x$ ist \wahr, bedeutet dies, 
    dass $x$ irgendeine wahre Aussage ist. \\
    In der Aussagenlogik würde man hierzu auch sagen, dass 
    $x$ mit dem Wahrheitswert \wahr\ \textbf{belegt} ist und wir notieren 
    dies durch $\mathfrak{I}(x) = \wahr$. 
    Statt zu sagen, dass eine Variable einen Wahrheitswert hat, spricht 
    man also in der Aussagenlogik davon, dass eine Variable mit einem Wahrheitswert 
    \textbf{belegt} ist. 

    \begin{definition}{Belegung}
        Nehmen die Variablen $x_1,\dots,x_n$ konkrete Wahrheitswerte an, dann
        nennt man diese Wahrheitswerte eine \textbf{Belegung} $\mathfrak{I}$ 
        dieser Variablen. Ist $x_i$ mit
        \wahr\ bzw. \falsch\  belegt, dann notieren 
        wir dies durch $\mathfrak{I}(x_i) = \wahr$ bzw. $\mathfrak{I}(x_i) = \falsch$.
    \end{definition}

    Genauso wie es ein Regelwerk zum Aufbau von Aussagen gibt, 
    nämlich die Syntax der Aussagenlogik, gibt es auch so ein Regelwerk mit der
    wir genau festlegen können, wie die Wahrheitswerte von Aussagen definiert sind.
    Dieses Regelwerk nennt sich die \textbf{Semantik der Aussagenlogik} und definiert 
    den Wahrheitswert von Verknüpfungen von Aussagen bezüglich einer 
    konkreten Belegung. Dieser Formalismus unterscheidet sich zu dem, was du im 
    vorherigen Abschnitt gesehen hast insofern, dass der Wahrheitswert einer Aussage 
    nun abhängig von einer konkreten Belegung ist und so gesehen keinen festen 
    Wahrheitswert hat. \\ 
    Zum Beispiel wirst du mit der gleich folgenden Definition der Semantik 
    sehen, dass die Aussage 
    \[A \land B\] 
    einerseits \wahr\ ist bezüglich der 
    Belegung $\mathfrak{I}_1$ mit $\mathfrak{I}_1(A) = \wahr, \mathfrak{I}_1(B) = \wahr$
    aber andereseits \falsch\ ist bezüglich der Belegung 
    $\mathfrak{I}_2$ mit $\mathfrak{I}_2(A) = \falsch, \mathfrak{I}_2(B) = \wahr$.

    \begin{definition}{Semantik der Aussagenlogik}
        Ist $X$ eine Aussage und $\mathfrak{I}$ eine Belegung, dann notieren 
        wir den Wahrheitswert von $X$ bezüglich $\mathfrak{I}$ durch $\mathfrak{I}(X)$.

        Sind $A,B$ zwei Aussagen und ist $\mathfrak{I}$ eine Belegung der in $A,B$ vorkommenden 
        Variablen, dann definiert folgende Tabelle den
        Wahrheitswert aller möglichen Verknüpfungen von $A,B$:

        \[\begin{array}{cc s ccccc}\toprule
            \mathfrak{I}(A) & \mathfrak{I}(B) & \mathfrak{I}(A \land B) & \mathfrak{I}(A\lor B) & \mathfrak{I}(A\implies B) & \mathfrak{I}(A\iff B) & \mathfrak{I}(\lnot A) \\\midrule
            \falsch & \falsch & \falsch & \falsch & \wahr & \wahr & \multirow{2}{*}{\wahr}\\
            \falsch & \wahr & \falsch & \wahr & \wahr & \falsch &  \\
             \wahr & \falsch & \falsch & \wahr & \falsch & \falsch & \multirow{2}{*}{\falsch}
            \\
            \wahr & \wahr & \wahr & \wahr & \wahr & \wahr & 
             \\\bottomrule
        \end{array}\]
    \end{definition}

    Wir können nun Aussagen in zwei Kategorien einteilen:
    \begin{example}{}
        Bezüglich der Belegung $\mathfrak{I}$ mit 
        $\mathfrak{I}(A) = \wahr, \mathfrak{I}(B) = \wahr$ ist die Aussage 
        $A \land B$ \wahr. Es gibt also mindestens eine Belegung für die 
        $A \land B$ \wahr  wird. Man spricht dann davon, 
        dass $A \land B$ \textbf{erfüllbar}
        ist.
    \end{example}
    In der ersten Kategorie befinden sich Aussagen zu welchen 
    es mindestens eine Belegung gibt, sodass die Aussage \wahr\ ist. Aussagen in 
    dieser Kategorie nennen sich \textbf{erfüllbar}.

    \begin{example}{}
        Die Aussage $A \land (\lnot A)$ ist bezüglich keiner Belegung 
        $\mathfrak{I}$ \wahr. In beiden möglichen Fällen, nämlich 
        $\mathfrak{I}(A) = \wahr$ oder $\mathfrak{I}(A) = \falsch$ ist die 
        Aussage $A \land (\lnot A)$ nämlich \falsch. Solche Aussagen nennt man \textbf{unerfüllbar}.
    \end{example}
    In der zweiten Kategorien befinden sich Aussagen, die nie \wahr\ werden, egal 
    welche Belegung man wählt. Solche Aussagen nennen wir \textbf{unerfüllbar}.

    \begin{definition}{Erfüllbarkeit}
        Wir nennen eine Aussage $A$ \textbf{erfüllbar}, falls es eine 
        Belegung $\mathfrak{I}$, der in $A$ vorkommenden Variablen gibt, 
        sodass $A$ \wahr\ ist, also $\mathfrak{I}(A) = \wahr$.
    \end{definition}

    \newpage

    \todo{Kein Erfüllbarkeits/Belegungsvokabular verwenden, damit die Abschnitte unabhängig voneinander gelesen werden können}

    \todo{@Tobias: Finde hierzu mal eine gute Überleitung oder Motivation :D}
    Manchmal ist es vorteilhaft sich nur mit Aussagen zu beschäftigen, 
    die eine ganz bestimmte Form besitzen.

    \begin{example}{}
        Wir betrachten einmal die Aussage $X$, welche wie folgt definiert ist:
        \[X = ((\lnot A) \land B \land C) \lor ((\lnot B) \land  C) 
        \lor (A \land (\lnot A))\]
        wobei $A,B,C$ aussagenlogische Variablen sind.
        Der Aufbau dieser Aussage entspricht einem bestimmten Schema:

        Grob betrachtet ist diese Aussage $X$ eine Veroderung von 
        mehreren (in diesem Fall 3) Unteraussagen.
        \[\underbrace{((\lnot A) \land B \land C)}_{D_1} \lor \underbrace{((\lnot B) \land 
        C)}_{D_2} \lor \underbrace{(A \land (\lnot A))}_{D_3}\]
        Diese 3 Unteraussagen $D_1,D_2,D_3$ haben selber auch wieder eine besondere Struktur.
        Um diese Struktur besser verstehen sie können brauchen wir erstmal einen neuen Begriff:
        Wir nennen alle Aussagen, die entweder eine aussagenlogische Variable 
        oder die Negation einer
        aussagenlogischen Variable sind, \textbf{Literale}.       
        In $D_1,D_2,D_3$ kommen Literale vor, die wir jetzt
        einmal alle markieren und bennenen:

        \[\begin{array}{cc}
            D_1 = & (\underbrace{(\lnot A)}_{L_1} \land \underbrace{B}_{L_2} \land 
            \underbrace{C}_{L_3})  \\
            \\
            D_2 = & (\underbrace{(\lnot B)}_{L_4} \land \underbrace{C}_{L_5}) \\ \\
            D_3 = & (\underbrace{A}_{L_6} \land \underbrace{(\lnot A)}_{L_7}) \\
        \end{array}\]
        Es stellt sich also heraus, dass 
        $D_1,D_2,D_3$ Verundungen von Literalen sind.

        Zusammengefasst ist also unsere ursprüngliche Aussage $X$
         eine Veroderung von Verundungen von
        Literalen. Wir können das wie folgt notieren:
        \[X = D_1 \lor D_2 \lor D_3\]
        \[\textrm{wobei: }D_1 = (L_1 \land L_2 \land L_3),\  D_2 = (L_4 
        \land L_5),\  D_3 = (L_6 \land L_7)\]
        Aussagen, die in dieses Schema passen, nennen wir 
        Aussagen in \textbf{disjunktiver Normalform}.
 
        Was ist nun der Vorteil daran, dass sich $X$ in
        \textbf{disjunktiver Normalform}
        befindet? Der Vorteil ist, dass wir sofort eine oder mehrere Kombinationen der 
        Wahrheitswerte der
        aussagenlogischen Variablen $A,B,C$ ablesen können, für 
        die $X$ \wahr\ wird. 

        Wir gehen wie folgt vor:
        \begin{enumerate}
            \item Wähle beliebig zwischen $D_1,D_2$:

                Wir entscheiden uns hier willkürlich für $D_1$. Wir wollen nun 
                Wahrheitswerte finden, sodass $D_1$ \wahr\  wird. Deswegen stand 
                hier auch nicht $D_3$ zur Auswahl, da $D_3$ ohnehin nie \wahr\  
                werden könnte, da 
                $A$ negiert als auch unnegiert in $D_3$ vorkommt.
            \item Wähle Wahrheitswerte der aussagenlogischen Variablen $A,B,C$, 
            so dass $D_1$ \wahr\ wird:
                \[ D_1 = ((\lnot \underbrace{A}_{\falsch}) \land \underbrace{B}_{\wahr} \land \underbrace{C}_{\wahr})\] 
            Mit $A$ \falsch\  sowie $B$ und $C$ \wahr, ist $D_1$ \wahr. $A$ hat den Wahrheitswert
            \falsch\ bekommen, weil $(\lnot A)$ in $D_1$ vorkommt. $B$ und $D$ kommen nicht negiert 
            vor, deshalb bekommen sie den Wahrheitswert \wahr.
        \end{enumerate}
        Wir haben also herausgefunden, dass durch $A$ \falsch\  sowie $B$ 
        und $C$ \wahr\ 
        $D_1$ \wahr\ ist und damit auch $X$. 
        Wir könnten auch noch andere Wahrheitswerte für $A,B,C$ finden für die $X$ \wahr\ 
        wird, indem wir im ersten Schritt $D_2$ gewählt hätten.
    \end{example}

    Eine häufig zu sehende Form, in der Aussagen sein können ist die 
    \textbf{disjunktive Normalform}. Um diese Form zu verstehen wird ein neuer 
    Aussagentyp benötigt: das \textbf{Literal}.

    \begin{definition}{Literal}
        Wir nennen aussagenlogische Variablen oder Negationen von aussagenlogischen Variablen, Literale.
    \end{definition}

    Aussagen in disjunktiver Normalform sind Aussagen, die Verorderungen von Verundungen von
    Literalen sind.

    \begin{definition}{Disjunktive Normalform}
        Eine Aussage $A$ ist in disjunktiver Normalform, falls $A$ folgende Form hat:
        \[A = D_1 \lor \dots \lor D_n\]
        wobei jedes $D_i$ folgende Form hat:
        \[D_i = L_1 \land \dots \land L_m \]
        wobei jedes $L_j$ ein Literal ist.
    \end{definition}

    Ein Grund, weshalb es von Vorteil ist Aussagen in disjunktiver Normalform
    vorliegen zu haben, ist es, dass man sofort, falls vorhanden, 
    eine Kombination der Wahrheitswerte der aussagenlogischen
    Variablen ablesen kann, für die die Aussage \wahr\ wird. 
    Dazu geht man wie folgt vor, falls man 
    die Aussage
    \[X= D_1 \lor \dots \lor D_n\]
    in disjunktiver Normalform vorliegen hat.
    \begin{enumerate}
        \item Wähle beliebig ein $D_1,\dots,D_n$, in welchem keine 
        aussagenlogische Variable sowohl negiert als auch nicht negiert vorkommt. 
        Ist dies nicht möglich, wissen wir bereits, dass $X$ nie \wahr\ werden kann 
        und wir brechen ab.
        
        Gehen wir davon aus, wir haben unsere Wahl getroffen und nennen sie $D_i$.
        \item Wähle Wahrheitswerte für die aussagenlogischen 
        Variablen in $D_i$ so, dass $D_i$ \wahr\  wird.
        
        Dies funktioniert wie folgt: Ist $A$ eine aussagenlogische Variable, die in $D_i$ vorkommt, dann:
            \begin{itemize}
                \item setzen wir $A$ \falsch, falls auch $(\lnot A)$ in $D_i$ vorkommt
                \item setzen wir $A$ \wahr, falls $(\lnot A$) nicht in $D_i$ vorkommt
            \end{itemize}
        
    \end{enumerate}
   
    Für die gewählten Wahrheitswerte der aussagenlogischen Variablen durch die zwei 
    Schritte ist das gewählte $D_i$ \wahr\ und damit dann auch $X$. Welche 
    Wahrheitswerte wir für die aussagenlogischen Variablen finden, hängt davon ab
    welches $D_i$ wir im ersten Schritt wählen. Andere Wahlen können zu anderen 
    Wahrheitswertkombinationen führen.

    Man kann dieses Vorgehen auch umdrehen. Wir haben gerade gesehen, 
    wie wir Wahrheitswerte für aussagenlogische Variablen
    aus einer disjunktiven Normalform ablesen können, sodass die Aussage \wahr\ wird. Andererseits können wir auch 
    aus einer gegebenen Wahrheitstabelle eine Aussage in disjunktiver Normalform ablesen,
    die genau diese Wahrheitstabelle besitzt.

    \begin{example}{}
        Stell dir vor wir haben diese Wahrheitstabelle mit den
        3 aussagenlogischen Variablen $a,b,c$ gegeben.

        \[\begin{array}{cccc s c}\toprule
            & a & b & c & \\\midrule
            1& \falsch & \falsch & \falsch & \falsch \\ 
            2& \falsch & \falsch & \wahr & \wahr \\ 
            3& \falsch & \wahr & \falsch & \falsch \\ 
            4& \falsch & \wahr & \wahr & \falsch \\ 
            5& \wahr & \falsch & \falsch & \falsch \\ 
            6& \wahr & \falsch & \wahr & \falsch \\ 
            7& \wahr & \wahr & \falsch & \wahr \\ 
            8& \wahr & \wahr & \wahr &\falsch \\ \bottomrule
        \end{array}\]

        Unser Ziel ist es eine Aussage $X$ zu finden, sodass $X$ genau diese Wahrheitstabelle 
        besitzt. Unsere Aussage $X$ muss nur in zwei Fällen \wahr\ werden, nämlich 
        in den Fällen:

        \[\begin{array}{cccc s c}\toprule
            \textrm{Zeile} & a & b & c\\\midrule

            2& \falsch & \falsch & \wahr & \wahr \\ 

            7&\wahr & \wahr & \falsch & \wahr \\ 
            \bottomrule
        \end{array}\]
        \[\]
        $X$ soll also nur \wahr\  sein, wenn der Fall in Zeile zwei oder 
        der Fall in Zeile sieben eintritt. 
        Dass der Fall in Zeile zwei eintritt nennen wir $D_1$. $D_1$
        soll also aussagen: $a$ ist \falsch\ und $b$ ist \falsch\  und $c$ ist \wahr. 
        Als Aussage formalisiert lautet dies:
        \[ D_1 = ((\lnot a) \land (\lnot b) \land c) \]
        Dass der Fall in Zeile sieben eintritt nennen wir $D_2$.$D_2$ sagt also 
        aus: $a$ ist \wahr\ und $b$ ist \wahr\ und $c$ ist \falsch. Formalisiert:
        \[ D_2 = (a \land b \land (\lnot c)) \]
        
        Dass mindestens einer dieser beiden Fälle eintritt ist genau die gesuchte 
        Aussage $X$:
        \[X = D_1 \lor D_2\]

        $X$ besitzt genau die geforderte Wahrheitstabelle und $X$ ist 
        eine Aussage in disjunktiver Normalform.

    \end{example}


    %Möchte man aus einer Wahrheitstabelle mit den aussagenlogischen
    %Variablen $a_1, \dots, a_n$ eine Aussage $X$ in disjunktiver Normalform ablesen, dann
    %betrachtet man dafür die Zeilen der Wahrheitstabelle in der die Aussage \wahr\ werden soll.
    %Nehmen wir an die Aussage ist in der Zeile $i$ \wahr. Die Zeile $i$ gibt für jede 
    %der $n$ aussagenlogischen Variablen einen Wahrheitswert an.
    %Für jede Zeile $i$ in der die Aussage \wahr\ wird, erstellen wir dann die Aussage 
    %\[D_i = L_1 \land 
    %\dots \land L_n\]
    %Das Literal $L_j$ ist die dabei aussagenlogische Variable 
    %$a_j$, falls die Variable $a_j$ in der Zeile \wahr\ ist. Ist die Variable $a_j$ in der Zeile 
    %\falsch\ , dann ist $L_j = (\lnot a_j)$.
    %Um die gesuchte Aussage in $X$ in disjunktiver Normalform zu erhalten, verodern wir alle erstellen
    %$D_i$.
    %\[X = D_1 \lor \dots \lor D_m\]


    Möchte man aus einer Wahrheitstabelle eine Aussage $X$ in disjunktiver Normalform 
    ablesen, die dieselbe Wahrheitstabelle besitzt,
    betrachtet man dafür die Zeilen der Wahrheitstabelle in der die 
    Aussage $X$ \wahr\ werden soll.

    Für jede Zeile $i$ in der die Aussage \wahr\ werden soll 
    erstellen wir eine Aussage
    $D_i$, die genau dann \wahr\  wird wenn der Fall  
    aus Zeile $i$ eintritt. 
    
    Ist Beispielsweise Zeile $i$ wie folgt:

    \[\begin{array}{ccccc}\toprule
         Zeile & a & b & c & d\\\midrule
         i &\wahr & \falsch & \wahr & \falsch \\ 
        \bottomrule
    \end{array}\]

    Dann wählen wir:
    \[D_i = a \land (\lnot b) \land c \land (\lnot d)\]

    Das Muster dahinter ist das Folgende:
    Wenn $x$ eine beliebige aussagenlogische Variable ist, 
    die in der Wahrheitstabelle auftaucht, dann
    \begin{enumerate}
        \item kommt $x$ ohne Veränderung in $D_i$ vor, falls in der Zeile $i$, $x$ den Wahrheitswert \wahr\ hat 
        \item kommt die Negation $(\lnot x)$ in $D_i$ vor, falls in der Zeile $i$, $x$ den Wahrheitswert \falsch\ hat
    \end{enumerate}

    Die gesuchte Aussage $X$ erhalten wir dann indem wir alle $D_i$ verodern, also
    ausdrücken, dass ein Fall eintritt indem die entsprechende Zeile in der 
    Wahrheitstabelle \wahr\  anzeigt. Also ergibt sich:
    \[X = D_1 \lor \dots \lor D_n\]
    

    Mit dem Wissen wie wir zu einer Wahrheitstabelle eine Aussage in disjunktiver Normalform
    aufstellen können, können wir jetzt sogar zu jeder Aussage eine semantisch äquivalente
    Aussage in disjunktiver Normalform bestimmen.

    \begin{example}{}
        Stell dir vor du hast die Aussage $A$
        \[A = (x \lor (\lnot y)) \implies (x \land (\lnot y)) \]
        vorliegen, wobei $x$ und $y$ aussagenlogische Variablen sind. Wir wollen 
        nun systematisch eine semantisch äquivalente Aussage in disjunktiver Normalform
        bestimmen. Wir gehen dabei in zwei Schritten vor.
        \begin{enumerate}
            \item Bestimmte Wahrheitstabelle von $A$
            \[\begin{array}{ccc s c}\toprule
                & x & y & A \\\midrule
                1 & \falsch & \falsch  & \falsch \\ 
                2  & \falsch & \wahr  & \wahr \\ 
                3 & \wahr & \falsch  & \wahr \\ 
                4 & \wahr & \wahr  & \falsch \\  \bottomrule
            \end{array}\]
            \item Lese eine Aussage $X$ in disjunktiver Normalform aus der Wahrheitstabelle ab
            
            Mit dem Vorgehen, was wir bereits kennengelernt haben, können wir nun aus dieser 
            Wahrheitstabelle eine 
            Aussage $X$ in disjunktiver Normalform ablesen. Wir betrachten wieder dazu 
            die Zeilen mit \wahr. Das sind die Zeilen 2,3, die für die Fälle ($x$ \falsch\ und $y$ \wahr) sowie
            ($x$ \falsch\ und $y$ \wahr) stehen. $X$ ist also:
            \[X = ((\lnot x) \land y) \lor (x \land (\lnot y))\]
        \end{enumerate}

        Weil $X$ und $A$ dieselbe Wahrheitstabelle besitzen, haben wir nun 
        eine Aussage $X$ in disjunktiver Normalform bestimmt, die semantisch äquivalent 
        zu unser Ausgangsaussage $A$ ist.
    
    \end{example}

    Möchten wir zu einer Aussage $A$ eine semantisch äquivalente Aussage $X$ in disjunktiver Normalform
    finden, dann gehen wir dabei wie folgt vor:
    \begin{enumerate}
        \item Stelle Wahrheitstabelle zu $A$ auf
        \item Lese aus Wahrheitstabelle von $A$ die Aussage $X$ in disjunktiver Normalform ab
    \end{enumerate}

    Neben der disjunktiven Normalform gibt es noch eine 
    weitere sehr ähnlich Normalform, die sogenannte \textbf{konjunktive Normalform}.
    Eine konjunkive Normalform ist die Veroderung von Verundungen von Literalen. 
    Die konjunktive Normalform ist also sehr ähnlich zur disjunktiven Normalform. Es 
    wurden quasi nur die Rollen der Verundungen und Veroderungen getauscht.
    
    \begin{definition}{Konjunktive Normalform}
        Eine Aussage $A$ ist in konjunktiver Normalform, falls $A$ folgende Form hat:
        \[A = K_1 \land \dots \land K_n\]
        wobei jedes $K_i$ folgende Form hat:
        \[K_i = L_1 \lor \dots \lor L_m \]
        wobei jedes $L_j$ ein Literal ist.
    \end{definition}

    Wir wissen, dass wir zu jeder Aussage eine semantisch äquivalente
    Aussage in 
    disjunktiver Normalform bestimmen können. Und wir stellen dabei fest,
    dass in einer Aussage in disjunktiver Normalform die Konnektoren 
    $\implies,\iff$ gar nicht vorkommen.
    Wir können also jede Aussage semantisch äquivalent nur mit den Konnektoren 
    $\lnot,\lor,\land$ schreiben.

    \begin{example}{}
        Wir können also vorallem die Konnektoren $\implies$ und $\iff$ nur durch
        $\lnot, \lor, \land$ semantisch äquivalent ausdrücken:
        \begin{enumerate}
            \item $\implies$:
            
            Wir kennen bereits die Äquivalenzumformung \enquote{Auflösen der Implikation} 
            aus dem Abschnitt zu Wahrheitstabellen. Wir wissen daher, dass 
            $a \implies b \equiv \lnot (a \land (\lnot b))$ gilt. Wir können 
            die Implikation also nur mit den Konnektoren $\lnot,\lor,\land$ schreiben.
            \item $\iff$:
            
            Wir stellen zunächst die Wahrheitstabelle zu $a \iff b$ auf:
            \[\begin{array}{cc s c}\toprule
                a & b & a \iff b \\\midrule
                 \falsch & \falsch  & \wahr \\ 
                 \falsch & \wahr  & \falsch \\ 
                \wahr & \falsch  & \falsch \\ 
                 \wahr & \wahr  & \wahr \\  \bottomrule
            \end{array}\]
            Wir können nun hieraus eine Aussage $X$ in disjunktiver Normalform ablesen, nämlich:
            \[X = ( (\lnot a) \land (\lnot b)) \lor (a \land b)\]
            $X$ ist semantisch äquivalent zu $a \iff b$ und enthält nur die Konnektoren $\lnot,\lor,\land$

        \end{enumerate}
    \end{example}

    Dadurch, dass wir semantisch äquivalent jede Aussage
    nur mit Hilfe von $\lnot,\lor,\land$
    ausdrücken können, nennen sich diese Konnektoren eine \textbf{vollständige Basis}.

    \begin{definition}{Vollständige Basis}
        Kombinationen von Konnektoren für die es für jede Aussage eine 
        semantisch äquivalente Aussage 
        gibt, in der nur diese Konnektoren vorkommen, nennen wir 
        eine \textbf{vollständige Basis}.
    \end{definition}

    Wir haben gesehen, dass $\implies, \iff$ gar nicht benötigt werden, um 
    alle Aussagen semantisch äquivalent auszudrücken. Wir brauchen nur 
    $\lnot,\lor,\land$. Aber geht das noch besser? Können wir noch einen Konnektor 
    loswerden? Erstaunlicherweise ja: Schon $\lnot,\land$ ist eine vollständige
    Basis, denn wir können nur mit diesen beiden Konnektoren bereits $\lor$ ausdrücken.

    \begin{example}{}
        Wir wollen zunächst $A \lor B$ nur mit $\lnot,\land$ ausdrücken.
        Dazu formen wir jetzt die Aussage $A \lor B$ sukkzessiv mit semantischen Äquivalenzumformungen
        um, bis die Aussage nur noch aus $\lnot$ sowie $\land$ besteht:
        \[\begin{array}{cc s c}\toprule
            \textrm{Schritt} & \textrm{Umformung} & \textrm{Resultat}\\\midrule
            \textrm{Start}   &   & A \lor B  \\
            1   & \textrm{Involution} & \lnot \lnot (A \lor B)\\
            2 & \textrm{1. De Morgansches Gesetz}   & 
            \lnot ((\lnot A) \land (\lnot B))\\
            \bottomrule
        \end{array}\]

        Wir haben also gezeigt, dass $A \lor B$ semantisch äquivalent zu  $\lnot ((\lnot A) \land (\lnot B))$
        ist.
    \end{example}

    Wir bemerken 
    zunächst, dass $A \lor B \equiv \lnot ((\lnot A) \land (\lnot B))$ ist. Dabei 
    sind $A,B$ beliebige Aussagen. Veroderungen 
    können wir also bereits nur durch die Konnektoren $\lnot,\land$ ausdrücken.

    Daraus können wir bereits folgern, dass sich jede Aussage semantisch 
    äquivalent durch $\lnot,\land$ schreiben lässt.

    \begin{example}{}
        Nehmen wir uns als Ziel mit System eine semantisch 
        äquivalente Aussage zu der Aussage
        \[a \iff b\]
        aus letztem Beispiel zu finden, die nur die Konnektoren $\lnot,\land$
        enthält.
        Wir gehen dabei in zwei Schritten vor.
        \begin{enumerate}
            \item Wir bestimmen zunächst die disjunktive Normalform:
            
                Wir wissen aus letzten Beispiel, dass $a \iff b \equiv ( (\lnot a) \land (\lnot b)) \lor (a \land b)$

                Die disjunktive Normalform enthält die Konnektoren $\lnot,\lor,\land$. Wir 
                müssen jetzt nur noch das $\lor$ loswerden.
            
            \item Wir können das $\lor$ entfernen, indem wir die Äquivalenz 
             $A \lor B \equiv \lnot ((\lnot A) \land (\lnot B))$ benutzen.
             Die folgende Markierung zeigt, was in unserem Fall 
             $A$ und was $B$ ist.
             \[\underbrace{((\lnot a) \land (\lnot b))}_{A} 
             \lor \underbrace{(a \land b)}_{B} \]
            Also ist $( (\lnot a) \land (\lnot b)) \lor (a \land b)$ semantisch 
            äquivalent zu 
            \[\lnot (   \underbrace{(\lnot ( (\lnot a) \land (\lnot b)))}_{\lnot A} 
                \land \underbrace{(\lnot (a \land b))}_{\lnot B}
            )\] 
            
            Wir haben also alle $\lor$ (hier nur ein einziges) entfernt, indem wir 
            Unteraussagen der Form $A \lor B$ durch $\lnot((\lnot A) \land (\lnot B))$
            ersetzt haben.

        \end{enumerate}

    Wir haben es also in zwei Schritten geschafft $a \iff b$ semantisch äquivalent 
    nur durch $\lnot,\land$ zu schreiben:
    \[ a \iff b \equiv \lnot ((\lnot ( (\lnot a) \land (\lnot b))) \land (\lnot (a \land b)))\]
    \end{example}

    Um zu einer beliebigen Aussage eine semantisch äquivalente Aussage 
    zu finden, die nur die Konnektoren $\lnot,\land$ enthält, geht man wie 
    folgt vor:

    \begin{enumerate}
        \item Bestimme disjunktive Normalform
        
            Nach diesem Schritt enthält unsere Aussage nur noch die Konnektoren 
            $\lnot,\lor,\land$. Der nächste Schritt hat das Ziel noch die $\lor$ loszuwerden.
        \item Ersetze Unteraussagen der Form $A \lor B$ durch 
                $\lnot ((\lnot A) \land (\lnot B))$ bis die Aussage kein $\lor$ mehr enthält
    \end{enumerate}
    
    Nach den Schritten 1 und 2 haben wir eine Aussage konstruiert, die nur auch 
    den Konnektoren $\lnot,\land$ besteht und semantisch äquivalent zu der 
    Ausgangsaussage ist.
    
    Wir haben damit also gezeigt, dass $\lnot,\land$ eine vollständige Basis ist.

    \begin{theorem}{}
       $\lnot,\land$ ist eine vollständige Basis.
    \end{theorem}

    Es gibt noch viele weitere vollständige Basen. Zum Beispiel ist auch 
    $\lnot,\lor$ eine vollständige Basis. Findest du 
    weitere Beispiele für vollständige Basen?

    \todo{Titel der Nutshellbox anders nennen?}
    \begin{nutshell}{Weiterführendes Wissen}
       Aussagen können wir präsize Anhand von einem Regelwerk definieren, welches 
       man \textbf{Syntax der Aussagenlogik} nennt und aus zwei Regeln besteht:
       \begin{enumerate}

        \item Aussagenlogische Variablen sind Aussagen 
        \item Sind $A,B$ Aussagen, dann sind folgende Ausdrücke auch wieder Aussagen:
            \begin{itemize}
                \item $(\lnot A)$
                \item $(A \lor B)$
                \item $(A \land B)$
                \item $(A \implies B)$
                \item $(A \iff B)$
            \end{itemize}

    \end{enumerate}
    
    Hat eine Variable $x$ einen konkreten Wahrheitswert angenommen, 
    nennen wir diesen auch \textbf{Belegung} von $x$ und notieren ihn 
    durch $\mathfrak{I}(x)$.

    Mithilfe des Begriffs der Belegung können wir nun auch ein Regelwerk zu 
    den Wahrheitswerten von Aussagen definieren, welches sich 
    \textbf{Semantik der Aussagenlogik} nennt. Dabei ist der Wahrheitswert 
    einer Aussage abhängig von einer Belegung $\mathfrak{I}$ und wir notieren 
    den Wahrheitswert einer Aussage $X$ bezüglich $\mathfrak{I}$ 
    durch $\mathfrak{I}(X)$.
    
    Sind $A,B$ zwei Aussagen und ist $\mathfrak{I}$ eine Belegung der in $A,B$ 
    vorkommenden 
    Variablen, dann definiert folgende Tabelle den
    Wahrheitswert aller möglichen Verknüpfungen von $A,B$:
    \[\begin{array}{cc s ccccc}\toprule
        \mathfrak{I}(A) & \mathfrak{I}(B) & \mathfrak{I}(A \land B) & \mathfrak{I}(A\lor B) & \mathfrak{I}(A\implies B) & \mathfrak{I}(A\iff B) & \mathfrak{I}(\lnot A) \\\midrule
        \falsch & \falsch & \falsch & \falsch & \wahr & \wahr & \multirow{2}{*}{\wahr}\\
        \falsch & \wahr & \falsch & \wahr & \wahr & \falsch &  \\
         \wahr & \falsch & \falsch & \wahr & \falsch & \falsch & \multirow{2}{*}{\falsch}
        \\
        \wahr & \wahr & \wahr & \wahr & \wahr & \wahr & 
         \\\bottomrule
    \end{array}\]

    Wir nennen Aussagen, die entweder eine Variable oder die Negation 
    einer Variable sind,
    \textbf{Literale}. Aussagen der Form 
    \[D_1 \lor \dots \lor D_n\]
    wobei alle $D_i$ Verundungen von Literalen sind, befinden sich in \textbf{disjunktiver Normalform}.
    Andersherum nennen wir Aussagen der Form 
    \[K_1 \land \dots \land K_n\]
    wobei alle $K_i$ Veroderungen von Literalen sind, in \textbf{konjunktiver Normalform}.
    \\ \\
    Wir können zu jeder Aussage eine semantisch äquivalente Aussage in disjunktiver
    Normalform finden und in einer disjunktiven Normalform kommen 
    nur die Konnektoren 
    $\lnot,\lor,\land$ vor. Diese drei Konnektoren bilden also eine 
    \textbf{vollständige Basis}, was eine Kombination von Konnektoren ist mit denen 
    man jede Aussage semantisch äquivalent formulieren kann. 
    Auch $\lnot,\land$ ist eine vollständige
    Basis.

    \end{nutshell}

\end{document}