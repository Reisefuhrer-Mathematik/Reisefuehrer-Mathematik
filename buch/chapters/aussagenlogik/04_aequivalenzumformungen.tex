\documentclass[../../main.tex]{subfiles}

\begin{document}

    Im Alltag begegnet man immer wieder Aussagen, die zwar verschieden sind, im Kern 
    aber genau das gleiche aussagen. Zum Beispiel sind die Aussagen \statement{Der Zauberer ist alt} und \statement{Der Zauberer ist nicht nicht alt} zwar verschieden, sagen aber das Gleiche aus.
    Jetzt möchten wir einmal ganz genau unter die Lupe nehmen, was es 
    für Aussagen heißt die gleiche Bedeutung zu haben.

    \begin{example}{}

    Betrachten wir dazu nochmal die atomare Aussage \statement{Der Zauberer ist alt} und kürzen diese 
    mit $Z$ ab. Die Aussage \statement{Der Zauberer ist nicht nicht alt} lässt sich dann 
    durch $\lnot \lnot Z$ notieren. Wir haben bereits im Gefühl,
    dass diese Aussagen im Kern genau das gleiche aussagen. Wir schauen uns als nächstes einmal die Wahrheitstabellen der 
    beiden Aussagen an:
    \[\begin{array}{c s ccc}\toprule
        Z & Z & \lnot Z & \lnot \lnot Z\\\midrule
        \falsch & \falsch & \wahr & \falsch\\
        \wahr & \wahr &\falsch & \wahr
         \\\bottomrule
    \end{array}\]
    
    Es stellt sich also heraus, egal welchen Wahrheitswert die einzige atomare Unteraussage $Z$, der 
    Aussagen $Z$ und $\lnot\lnot Z$ annimmt, die beiden Aussagen, haben immer den gleichen Wahrheitswert. 
    Diese Eigenschaft nennen wir \textbf{semantische Äquivalenz}. 
    \end{example}
    
    Wir nennen zwei Aussagen \textbf{semantisch äquivalent}, wenn sie immer denselben Wahrheitswert haben. Das 
    entspricht intuitiv dem Konzept, dass die Aussagen die gleiche Bedeutung haben. Diese Intuition findet sich auch
    im Namen dieses Begriffs wieder, denn \enquote{semantisch} ist ein Synonym für \enquote{von der Bedeutung her}
    und \enquote{äquivalent} können wir mit \enquote{gleichwertig} übersetzen.
    
    \begin{definition}{Semantische Äquivalenz}
        Zwei Aussagen $A,B$ heißen \textbf{semantisch äquivalent}, wenn diese für alle möglichen Wahrheitswerte ihrer
        atomaren Unteraussagen den gleichen Wahrheitswert haben. Wir notieren dies durch $A \equiv B$.
    \end{definition}
    
    Wir schauen uns nun ein weiteres Beispiel für zwei semantisch äquivalente Aussagen an.
    
    \begin{example}{Semantische Äquivalenz} 

        Die Aussagen \statement{Der Zauberer ist gut gelaunt und es regnet} und 
        \statement{Es regnet und der Zauberer ist gut gelaunt} sind zwar
        verschieden, haben aber die gleiche Bedeutung, weil es bei einer Verknüpfung von 
        zwei Aussagen durch \statement{und} egal ist, in welcher Reihenfolge wir die Aussagen 
        verbinden. Diese beiden Aussagen
        sind also semantisch äquivalent. 
        \\\\
        Wir wollen jetzt einmal ganz genau mit Wahrheitstabellen
        zeigen, dass diese beiden Aussagen semantisch äquivalent sind. Das heißt, sie haben denselben Wahrheitswert,
        egal welche Wahrheitswerte die atomaren Unteraussagen haben.
        Die atomaren Unteraussagen sind \statement{Es regnet} (Abkürzung: $R$) und
         \statement{Der Zauberer ist gut gelaunt} (Abkürzung: $G$). Formalisiert lauten unsere
         Aussagen dann
         \[ G \land R \textrm{ sowie } R \land G.\]
        Wir erhalten für diese beiden Aussagen die folgenden Wahrheitstabellen.
 
         \[\begin{array}{cc s cc}\toprule
            G & R & G \land R &  R \land G\\\midrule
            \falsch   & \falsch   & \falsch & \falsch  \\
            \falsch   & \wahr & \falsch &\falsch\\
            \wahr & \falsch   & \falsch & \falsch\\
            \wahr & \wahr & \wahr & \wahr\\\bottomrule
        \end{array}\]

        Die Wahrheitstabellen zeigen, dass die beiden Aussagen für jede der vier Kombinationen der 
        Wahrheitswerte für $R$ und $G$ die beiden Aussagen
        denselben Wahrheitswert annehmen. Die Aussagen sind also semantisch äquivalent. Das notieren wir
        wie folgt:
        \[G \land R \equiv R \land G\]

    \end{example}
    
    Um herauszufinden, ob zwei Aussagen semantisch äquivalent sind, können wir einfach die Wahrheitstabellen der Aussagen aufstellen und überprüfen, ob diese übereinstimmen. Das ist eine Methode, die immer zum Ziel führt. Das kann jedoch sehr aufwendig werden. 
    Für eine Aussage mit nur fünf atomaren Aussagen hat die Wahrheitstabelle beispielsweise bereits 32 Zeilen, bei 10 atomaren Aussagen, sind es sogar 1024 Zeilen. 
    
    Bekanntlich führen viele Wege nach Rom. So auch für den Fall, dass man herauszufinden möchte, 
    ob zwei Aussagen semantisch äquivalent sind. Wir lernen gleich eine Methode kennen, mit der man ohne 
    Wahrheitstabellen zeigen kann, dass Aussagen semantisch äquivalent sind. Um das Prinzip dieser 
    Methode verstehen zu können, schauen wir an dieser Stelle einmal kurz, wie man bei Termen herausfinden 
    kann, ob diese gleichwertig sind. Für Aussagen funktioniert das dann fast genauso.
    
    \begin{example}{Termumformungen}
        Stell dir vor, du sollst zeigen, dass die Terme $-(x \cdot (x - y))$ und $-x^2 + x\cdot y$ 
        gleichwertig sind. Eine Möglichkeit ist es, zu zeigen, dass man mittels Terumformungen von 
        einem Term zum anderen gelangen kann. Die nächste Tabelle zeigt Schritt für Schritt eine 
        mögliche Abfolge von Termumformungen, um zu zeigen, dass die Terme $-(x \cdot (x - y))$ und $-x^2 + x\cdot y$
        gleichwertig sind.
        \[\begin{array}{cc s c}\toprule
        \textrm{Schritt} & \textrm{Umformung} & \textrm{Resultat}\\\midrule
        \textrm{Start}   &   & -(x\cdot (x-y))  \\
        1   & \textrm{Distributivgesetz} & -(x^2 -x\cdot y)\\
        2 & \textrm{Distributivgesetz}   & -x^2 - (-x\cdot y)\\
        3 & \textrm{Vorzeichenregel} &  -x^2 + x\cdot y\\\bottomrule
        \end{array}\]

    \end{example}
    
    Um zu zeigen, dass zwei Terme gleichwertig sind, kann man also zeigen, dass man vom ersten 
    zum zweiten Term mittels Termumformungen kommt. Und dieses Prinzip übertragen wir jetzt auf die 
    Aussagenlogik: Um zu zeigen, dass zwei Aussagen semantisch äquivalent sind, zeigt man, 
    dass man von der ersten Aussage durch \textbf{(semantische) Äquivalenzumformungen} zur zweiten
    gelangt. Diese Äquivalenzumformungen quasi die Terumformungen der Aussagenlogik.
    Termumformungen produzieren gleichwertige Terme, Äquivalenzumformungen produzieren semantisch 
    äquivalente Aussagen.
    
    Im Folgenden lernen wir solche Äquivalenzumformungen kennen, um am Ende einen Werkzeugkasten zu haben, mit dem man Aussagen (semantisch äquivalent) umformen kann.
    
    
    \parpic[r]{
        \begin{array}{c s ccc}\toprule
            Z & Z & \lnot Z & \lnot \lnot Z\\\midrule
            \falsch & \falsch & \wahr & \falsch\\
            \wahr & \wahr &\falsch & \wahr
             \\\bottomrule
        \end{array}
    }

    Eine erste Äquivalenzumformung haben wir bereits im Eingangsbeispiel gesehen. 
    Negieren wir eine 
    Aussage $Z$ doppelt, dann ist die Aussage $\lnot \lnot Z$ semantisch äquivalent zur ursprünglichen Aussage. Das Hinzufügen oder
     Entfernen von einer doppelten Negation, nennt sich \textbf{Involution} und ist eine Äquivalenzumformung.

    \begin{theorem}{Involution}
    Ist $A$ eine Aussage, dann gilt:
        \[\lnot \lnot A \equiv A\]
    \end{theorem}
     
    Im nächsten Beispiel lernen wir nun eine neue semantische Äquivalenz kennen. Diese Äquivalenz
    ist etwas komplexer, als die vorherige. Es ist aber auch durchaus möglich, dass du diese
    Äquivalenz bereits unbewusst im Alltag verwendet hast.
    
    \begin{example}{De Morgansche Gesetze}
            Bisher waren wir oft damit konfrontiert, 
            dass uns ein Zauberer einen Zaubertrank anbietet, der entweder giftig ist 
            oder Superkräfte verleiht. Wir sind der Meinung, 
            dass ein netter Zauberer uns niemals einen giftigen Zaubertrank andrehen würde. 
            Deshalb kann der Zauberer nicht gleichzeitig nett und der Trank giftig sein. 

            Wenn man ein bisschen darüber nachdenkt, kann man merken, 
            dass dies gleichbedeutend damit ist, dass der Zauberer nicht nett oder 
            der Trank nicht giftig ist.  
             
             Wir verwenden $N$ als die Abkürzung für \statement{Der Zauberer ist nett} 
             und $G$ als die Abkürzung für \statement{Der Zaubertrank ist giftig}. 
             Die beiden Aussagen sind dann:
             \[\lnot (N \land G) \textrm{ sowie } (\lnot N) \lor (\lnot G)\]
             Wir überprüfen jetzt unsere Vermutung, dass diese beiden Aussage im Grunde 
             dasselbe bedeuten. Das heißt:
             \[\lnot (N \land G) \equiv (\lnot N) \lor (\lnot G)\]
             Dazu stellen wir eine Wahrheitstabelle auf:
                 \[\begin{array}{cc s cccccc}\toprule
                    N & G & \lnot N & \lnot G &N \land G &\lnot (N \land G) & (\lnot N) \lor (\lnot G)\\\midrule
                    \falsch   & \falsch &  \wahr & \wahr &\falsch   & \wahr & \wahr  \\
                    \falsch   & \wahr &  \wahr & \falsch &\falsch &\wahr &\wahr\\
                    \wahr & \falsch   & \falsch & \wahr &\falsch & \wahr & \wahr\\
                    \wahr & \wahr & \falsch & \falsch &\wahr &\falsch & \falsch\\\bottomrule
              \end{array}\]
              Und tatsächlich bestätigen uns die Wahrheitstabellen: Die beiden Aussagen sind semantisch äquivalent.
    \end{example}
     
     Das Beispiel lässt sich auf beliebige Aussagen verallgemeinern. 
     Haben wir zwei Aussagen $A,B$ gegeben, dann gilt immer
     
     \[\lnot( A \land B) \equiv (\lnot A) \lor (\lnot B).\]
     
     Völlig gleich, funktioniert dies auch, wenn wir 
     statt dem $\land$ ein $\lor$ haben. Es gilt also auch:
     \[\lnot( A \lor B) \equiv (\lnot A) \land (\lnot B).\]
     Diese Regeln nennen sich die \textbf{De Morganschen Gesetze}.
     
    \begin{theorem}{De Morgansche Gesetze}

        Sind $A,B$ zwei Aussagen, dann gilt:
        \begin{enumerate}
            \item $\lnot( A \land B) \equiv (\lnot A) \lor (\lnot B)$
            \item $\lnot( A \lor B) \equiv (\lnot A) \land (\lnot B)$
        \end{enumerate}
    \end{theorem}
    
    Die nächsten beiden Äquivalenzumformungen beschäftigten sich speziell damit, wie 
    man eine Implikation umformen kann. Das wird uns ermöglichen,
    die Implikation besser zu verstehen.
    
    \begin{example}{Kontraposition}
        Da uns Gift im Allgemeinen nicht so gut bekommt, stellen wir folgende Aussage auf:
        \statement{Wenn der Trank giftig ist, dann stirbt man beim Trinken}
        \\ \\
        Gibt es eine alternative, gleichbedeutende Formulierung für diese Aussage? Ja:
        \statement{Stirbt man nicht beim Trinken, dann ist der Trank nicht giftig}
        \\ \\
        Wir vermuten, dass diese Aussagen semantisch äquivalent sind.

        Wir wollen unsere Vermutung mit Wahrheitstabellen überprüfen. Zunächst formalisieren wir die beiden 
        Aussagen. Wir verwenden dafür die Abkürzungen $S$ (\statement{Man stirbt beim Trinken}) und $G$ (\statement{Der Trank ist giftig}). Die Formalisierungen lauten dann:

        \[G \implies S \textrm{ und } \lnot S \implies \lnot G\]

        \[\begin{array}{cc s cccc}\toprule
            G & S & G \implies S & \lnot S & \lnot G &\lnot S \implies \lnot G\\\midrule
            \falsch   & \falsch   & \wahr & \wahr & \wahr &\wahr  \\
            \falsch   & \wahr & \wahr & \falsch & \wahr &\wahr\\
            \wahr & \falsch   & \falsch & \wahr & \falsch &\falsch\\
            \wahr & \wahr & \wahr & \falsch & \falsch&\wahr\\\bottomrule
      \end{array}\]

      Die Wahrheitstabellen bestätigen, dass die beiden Aussagen semantisch äquivalent sind.
      Es gilt also
      \[G \implies S \equiv \lnot S \implies \lnot G.\]
        
    \end{example}
    
    Haben wir zwei Aussagen $A,B$, dann 
    gilt immer \[A \implies B \equiv (\lnot B) \implies (\lnot A).\] Klar muss 
    diese Äquivalenz gelten:  Wir erinnern uns, dass man eine Implikation als ein 
    Versprechen auffassen kann, dass immer, wenn $A$ \wahr\ ist, auch $B$ \wahr\ ist. Und nur wenn dieses Versprechen explizit gebrochen wird,
     dann wird die Implikation \falsch. Das ist der Fall, wenn $A$ \wahr\ ist, $B$ 
     aber $\falsch$ ist. Und um auszudrücken, dass dieser Fall nicht eintritt, können wir 
     auch genauso gut sagen, dass immer, wenn $B$ \falsch\  ist, auch $A$ \falsch\  ist. Wir 
     stellen also sicher, dass das Versprechen nicht gebrochen wird. Formalisieren wir dies, 
     wird daraus $(\lnot B) \implies (\lnot A)$.

    Diese Äquivalenzumformung nennt sich \textbf{Kontraposition} (lateinisch: \enquote{contra} = \enquote{gegen}, \enquote{positio} = \enquote{stellung})
    und wird im nächsten Satz noch einmal präzisiert.
    
    \begin{theorem}{Kontraposition}

    Sind $A,B$ zwei Aussagen, dann gilt:
        \[A \implies B  \equiv (\lnot B) \implies (\lnot A)\]
    \end{theorem}

    Es ist bei der Kontraposition äußerst wichtig, dass wir die Negationen bei der Umkehrung
    nicht einfach weglassen können. 

    \begin{example}{Implikationen können nicht 'umgedreht' werden}
        Die Aussage \statement{Wenn der Trank giftig ist, 
        dann stirbt man beim Trinken} (formalisiert: $G \implies S$) aus letztem Beispiel ist 
        keinesfalls semantisch äquivalent zu der 
        Aussage \statement{Stirbt man beim Trinken, dann ist der Trank giftig} (formalisiert: $S \implies G$).
        Stirbt man beim Trinken, könnte das ja auch andere Ursachen gehabt haben (z.B. weil man sich verschluckt hat).
        \\ \\
        Für dieses Beispiel ist also $G \implies S$ \textbf{nicht} semantisch äquivalent zu $S \implies G$.

    \end{example}

    Im Allgemeinen ist $A \implies B$ also \textbf{nicht} semantisch äquivalent zu $B \implies A$.

    Wir haben gerade gesehen, dass sich jede Implikation auch anders darstellen lässt, 
    nämlich mit der Kontraposition. Es gibt noch eine weitere, nützliche Äquivalenzumformung 
    der Implikation, die uns auch dabei weiter hilft die Implikation besser zu verstehen.
    Diese schauen wir uns nun an.
    
    \begin{example}{Auflösen der Implikation}
        Wir überlegen uns eine weitere alternative Möglichkeit dafür, wie wir die Aussage
        \statement{Wenn der Trank giftig ist, 
        dann stirbt man beim Trinken} semantisch äquivalent formulieren können.
        \\ \\
        Wir wissen, dass Implikationen Versprechen sind und eine Implikation nur \falsch\ wird,
        wenn dieses Versprechen explizit gebrochen wird. Das Versprechen wird hier nur
        gebrochen, wenn der Trank zwar giftig ist, aber man trotzdem beim Trinken nicht stirbt.

        Dass das Versprechen nicht gebrochen wird können wir mit der Aussage
        
        \statement{Es stimmt nicht, dass der Trank giftig ist und man beim Trinken nicht stirbt}
        ausdrücken.
        \\ \\
        Diese Aussage formalisieren wir nun, um unsere 
        Vermutung, dass diese Aussage zur ursprünglichen Aussage semantisch
         äquivalent ist formal zu überprüfen.
        Wir verwenden wieder die Abkürzungen $G$ (\statement{Der Trank ist giftig}) und
        $T$ (\statement{Beim Trinken stirbt man}). Die beiden Aussagen, die wir auf semantische 
        Äquivalenz überprüfen wollen, lauten dann:
        \[G \implies T \textrm{ sowie } \lnot (G \land (\lnot T))\]

        \[\begin{array}{cc s cccc}\toprule
            G & T & G \implies T & \lnot T& G \land (\lnot T) &\lnot (G \land (\lnot T))\\\midrule
            \falsch   & \falsch   & \wahr & \wahr &\falsch &\wahr  \\
            \falsch   & \wahr & \wahr & \falsch &\falsch &\wahr\\
            \wahr & \falsch   & \falsch & \wahr &\wahr &\falsch\\
            \wahr & \wahr & \wahr & \falsch &\falsch &\wahr\\\bottomrule
      \end{array}\]

      Die Wahrheitstabellen bestätigen, dass unsere Aussagen semantisch äquivalent sind. 
      Es gilt also
      \[G \implies T \equiv \lnot (G \land (\lnot T)).\]


    \end{example}
    
    Egal welche zwei Aussagen $A,B$ wir 
    gegeben haben, es gilt immer:
    \[A \implies B \equiv \lnot (A \land (\lnot B))\] 
    Man fasst die Implikation wieder als 
    ein Versprechen auf und nur wenn dieses Versprechen explizit gebrochen wird,
     wird die Implikation \falsch. Und dieses Versprechen wird nur in dem Fall 
     gebrochen, dass $A$ \wahr\ ist, aber $B$, \falsch\  ist. Um auszudrücken, 
     dass dieser Fall nicht eintritt sagen wir es dürfen \textbf{nicht} gleichzeitig 
     $A$ \wahr\ und $B$ \falsch\ sein. Formalisieren wir dies, wird daraus 
     $\lnot (A \land (\lnot B))$.
    
    Diese Äquivalenzumformung nennt sich auch das \textbf{Auflösen der Implikation}, da wir hier die Implikation darstellen durch eine Aussage ohne Implikationszeichen. Wir haben sozusagen die Implikation \enquote{aufgelöst}. Der nächste Satz formalisiert dies nun.
    
    \begin{theorem}{Auflösen der Implikation}
        Sind $A,B$ zwei Aussagen, dann gilt:
        \[A \implies B \equiv \lnot (A \land (\lnot B))\]
    \end{theorem}
    
    Wenn wir nicht nur die letzte Äquivalenzumformung betrachten, sondern auch die davor, 
    dann stellt sich heraus, 
    dass wir jetzt zwei verschiedene Möglichkeiten kennen eine Implikation alternativ 
    darzustellen bzw. umzuformen. Zusammengefasst wissen wir jetzt also, dass:
     \[A \implies B \equiv \lnot B \implies \lnot A \equiv \lnot (A \land (\lnot B))\]
    
    Ähnlich dazu, wie man eine Implikation auflösen kann, 
    kann man auch eine Äquivalenz ($\iff$) auflösen.

    \todo{In der Wahrheitstabelle noch 2 Spalten für die Zwischenschritte anlegen. Dafür 
    müsste aber die gesamte Wahrheitstabelle runterskaliert werden, aber ich hab tbh keinen Plan,
    wie man das macht.}
    \begin{example}{Auflösen der Äquivalenz}
        Der Zaubertrank vom Zauberer kann nicht gleichzeitig giftig 
        sein (Abkürzung: $G$) 
        und Superkräfte verleihen (Abkürzung: $S$). Der Trank ist also genau dann, 
        giftig, wenn
        der Trank keine Superkräfte verleiht.
        Dieser Sachverhalt als Aussage formuliert, lautet:

        \statement{Der Zaubertrank ist giftig genau dann, wenn er keine Superkräfte 
        verleiht}

        Formalisieren wir diese Aussage, erhalten wir:
        \[G \iff (\lnot S)\]
        Wenn man genau hinschaut, sieht das Äquivalenzzeichen so aus wie zwei sich überlappende
        Implikationszeichen ($\iff$$=$\  $\Longleftarrow$\  $+$$\implies$). Das ist kein Zufall,
        tatsächlich wurde das Zeichen mit Absicht so gewählt. Jede Äquivalenz kann auch durch zwei 
        Implikationen ausgedrückt werden.
        \\ \\
        Wir probieren unsere Aussage jetzt semantisch äquivalent durch zwei 
        Implikationen auszudrücken:

        \statement{Wenn der Trank giftig ist, verleiht er keine Superkräfte und wenn der Trank keine Superkräfte
        verleiht, ist er giftig}

        Wenn wir diese Aussage formalisieren, wird deutlich, wie wir das Äquivalenzzeichen in zwei 
        Implikationszeichen \enquote{aufgeteilt} haben: 

        \[(G \implies (\lnot S)) \land ((\lnot S) \implies G)\]

        Unsere Vermutung, dass diese und unsere ursprüngliche Aussage wirklich semantisch äquivalent sind, überprüfen wir
        jetzt mit Wahrheitstabellen:
        \[\begin{array}{cc s cc}\toprule
            G & S &  G \iff (\lnot S) & (G \implies (\lnot S)) \land ((\lnot S) \implies G)\\\midrule
            \falsch   & \falsch    & \falsch & \falsch  \\
            \falsch   & \wahr  & \wahr &\wahr\\
            \wahr & \falsch    &\wahr & \wahr\\
            \wahr & \wahr & \falsch & \falsch\\\bottomrule
      \end{array}\]
      Unsere Aussagen sind also semantisch äquivalent.
      \[G \iff (\lnot S) \equiv (G \implies (\lnot S)) \land ((\lnot S) \implies G)\]
    \end{example}
    
    Wir verallgemeinern jetzt das, was wir im letzten Beispiel gesehen haben.
     Haben wir zwei beliebige Aussagen $A,B$ gegeben, können wir immer die Äquivalenz
     durch zwei Implikationen ausdrücken:
    \[A \iff B \equiv (A \implies B) \land (B \implies A)\]
    Die Regel können wir uns leicht merken, indem wir uns einfach das Äquivalenzzeichen
    ($\iff$) vorstellen, als zwei sich überlappende Implikationszeichen ($\iff = \Longleftarrow + \implies$).

    Dass diese Äquivalenzumformung korrekt ist, können wir auch mithilfe der anderen, hier vorgestellten Äquivalenzumformungen 
    zeigen: Wir erinnern uns, dass zwei äquivalente Aussagen 
    immer denselben Wahrheitswert haben. Anders ausgedrückt heißt das:
    \begin{enumerate}
        \item Ist $A$ \wahr, dann ist $B$ auch \wahr\  (Formal: $A \implies B$)
        \item Ist $A$ \falsch, dann ist $B$ auch \falsch\  (Formal: $(\lnot A) \implies (\lnot B)$)
    \end{enumerate}
    Durch Kontraposition wissen wir, dass die 2. Aussage semantisch äquivalent dazu ist,
    dass aus $B$, $A$ folgt:
    \[(\lnot A) \implies (\lnot B) \equiv B \implies A\]
    Verknüpfung wir jetzt diese Aussagen mit einem \statement{und}, dann erhalten wir:
    \[(A \implies B) \land (B \implies A)\]
    
    Im Grunde ist jede Äquivalenz also nichts anderes als zwei Implikationen und kann immer als solche geschrieben werden. 
    Diese Aufteilung einer Äquivalenz in zwei Implikationen nennt sich auch \textbf{Auflösen der Äquivalenz}.
    
    \begin{theorem}{Auflösen der Äquivalenz}
    Sind $A,B$ zwei Aussagen, dann gilt:
        \[ A \iff B \equiv (A \implies B) \land (B \implies A)\]
    \end{theorem}

    Die Äquivalenzumformungen, die wir hier vorgestellt haben, sind bei weitem nicht alle, die es gibt. Es gibt
    noch eine riesige Menge weiterer Umformungen, die wir hier aus Zeit/Platzgründen nicht vorstellen können.
    
    Der Hauptvorteil von semantischen Äquivalenzumformungen ist es, dass wir ohne
    Wahrheitstabellen überprüfen können, ob zwei Aussagen semantisch äquivalent sind. 
    Im nächsten Beispiel siehst du, wie wir das machen können.

    \begin{example}{Anwendung von Äquivalenzumformungen}{
            Wir wollen mittels semantischer Äquivalenzumformungen zeigen, dass 
            folgende semantische Äquivalenz gilt:
            \[A \implies B \equiv (\lnot A) \lor B\]
            Die nächste Tabelle zeigt Schritt für Schritt welche Umformungen wir angewendet haben:
            \[\begin{array}{cc s c}\toprule
                \textrm{Schritt} & \textrm{Umformung} & \textrm{Resultat}\\\midrule
                \textrm{Start}   &   & A \implies B  \\
                1   & \textrm{Auflösen der Implikation} & \lnot (A \land (\lnot B))\\
                2 & \textrm{1. De Morgansches Gesetz}   & (\lnot A) \lor (\lnot \lnot B)\\
                3 & \textrm{Involution} &  (\lnot A) \lor B\\\bottomrule
            \end{array}\]
            Durch die Äquivalenzumformungen konnten wir also zeigen, dass die Implikation $A \implies B$
            semantisch äquivalent zu der Aussage $(\lnot A) \lor B$ ist. Wir haben
            jetzt also noch eine neue Darstellung für die Implikation kennengelernt. 

    } \end{example}

    Äquivalenzumformungen können aber auch einen anderen Zweck erfüllen: So wie man Termumformungen
    nutzt, um Terme zu vereinfachen, kann man auch Äquivalenzumformungen benutzen, um Aussagen
    zu vereinfachen.

    \todo{Wegen Farbblindheit, die rot markierten Stellen textmarkern.}
    \begin{example}{Eine Vereinfachung mit Äquivalenzumformungen}

        Wir wollen einmal mit Äquivalenzumformungen die Aussage 

        \[ (\lnot ( (\lnot \lnot A)\land (\lnot B) ) \land (B \implies A))\]

        vereinfachen. Die folgende Tabelle zeigt schrittweise, welche Äquivalenzumformungen
        wir anwenden. Rot markiert ist jeweils die Unteraussage, auf die eine Umformung angewendet wurde.

        \[\begin{array}{cc s c}\toprule
            \textrm{Schritt} & \textrm{Umformung} & \textrm{Resultat}\\\midrule
            \textrm{Start}   &   & (\lnot ( \color{red}(\lnot \lnot A) \color{black}\land (\lnot B) ) \land (B \implies A))  \\
            1   & \textrm{Involution} & \color{red}(\lnot ( A\land (\lnot B) ) \color{black} \land (B \implies A))\\
            2 & \textrm{Auflösen der Implikation}   & \color{red}(A \implies B) \land (B \implies A))  \color{black}\\
            3 & \textrm{Auflösen der Äquivalenz} &  A \iff B\\
        \end{array}\]

        Durch Äquivalenzumformungen haben wir also Folgendes gezeigt:
        \[(\lnot ( (\lnot \lnot A)\land (\lnot B) ) \land (B \implies A)) \equiv A \iff B\]
        Wir haben die Aussage also deutlich vereinfacht.

    \end{example}

    Äquivalenzumformungen können auch im Alltag helfen. Versteht man eine Aussage nicht,
    kann es helfen, diese Aussage semantisch äquivalent umzuformen, um einen anderen
    Blickwinkel auf die Aussage zu bekommen, um diese besser zu verstehen.

    \begin{example}{Vereinfachung einer Alltagsaussage}
        Wir tun uns schwer damit, viele Negationen im Kopf schnell zu verarbeiten.
        Hören wir eine Implikation, deren atomare Aussagen negiert sind, dann bietet es sich
        an, Kontraposition anzuwenden. Die Aussage
        \statement{Wenn die Straße nicht nass ist, dann regnet es nicht}
        ist schwer zu verstehen. Dank Kontraposition wissen, dass sie
        semantisch äquivalent zu der Aussage
        \statement{Wenn es regnet, dann ist die Straße nass}
        ist. Diese ist wesentlich einfacher zu verstehen.
        
    \end{example}

    \begin{nutshell}{Äquivalenzumformungen}
        Zwei Aussagen $A,B$ heißen \textbf{semantisch äquivalent}, wenn diese für alle möglichen Wahrheitswerte ihrer
        atomaren Unteraussagen den gleichen Wahrheitswert haben. Wir notieren dies durch $A \equiv B$.
        \\ \\
        Eine Umformung einer Aussage $A$ zu einer Aussage $A^{\prime}$, so dass $A \equiv A^{\prime}$ ist, nennt sich \textbf{(semantische) Äquivalenzumformung}.
        \\ \\
        Die folgende Tabelle zeigt einige (semantische) Äquivalenzumformungen. Sind $A,B$ Aussagen, dann gilt:
        \[\begin{array}{c s ccc}\toprule
        \textrm{Name} & & \textrm{Äquivalenz}\\\midrule
        \textrm{Involution} & \lnot \lnot A &\equiv& A\\
        \textrm{1. De Morgansches Gesetz} & \lnot (A \land B) &\equiv& (\lnot A) \lor (\lnot B)\\
        \textrm{2. De Morgansches Gesetz} & \lnot (A \lor B) &\equiv& (\lnot A) \land (\lnot B) \\   
        \textrm{Kontraposition} & A \implies B &\equiv& (\lnot B) \implies (\lnot A) \\
        \textrm{Auflösen der Implikation} & A \implies B & \equiv & \lnot (A \land (\lnot B))\\
        \textrm{Auflösen der Äquivalenz} & A \iff B & \equiv & (A \implies B) \land (B \implies A) \\
        \bottomrule
        \end{array}\]
    \end{nutshell}
\end{document}