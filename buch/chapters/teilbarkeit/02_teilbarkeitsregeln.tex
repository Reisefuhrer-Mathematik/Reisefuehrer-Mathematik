\documentclass[../../main.tex]{subfiles}

\begin{document}

\begin{theorem}{}
    Für alle $n$ gilt $1\divides n$ und $n\divides 0$
\end{theorem}

\begin{theorem}{}
    $a\divides b,b\neq 0\implies |a|\leq|b|$
\end{theorem}

\begin{theorem}{}
    $a\divides b$ und $b\divides c\implies a\divides c$
\end{theorem}

\begin{theorem}[thm:teilbarkeit-summen]{}
    Es gilt $a\divides n$ und $b\divides n\iff (a+b)\divides n\iff (a-b)\divides n$
\end{theorem}

Herleitung Endstellenregel
\begin{theorem}{Endstellenregel}
    Eine Zahl ist genau dann\dots
    \begin{itemize}
        \item durch $2$ teilbar, wenn ihre letzte Ziffer durch $2$ teilbar ist.
        \item durch $5$ teilbar, wenn ihre letzte Ziffer durch $5$ teilbar ist.
        \item durch $10$ teilbar, wenn ihre letzte Ziffer $0$ teilbar ist.
    \end{itemize}
\end{theorem}

Quersummenregel ohne Herleitung
\begin{theorem}{Quersummenregel}
    Eine Zahl ist genau dann\dots
    \begin{itemize}
        \item durch $3$ teilbar, wenn ihre Quersumme durch $3$ teilbar ist.
        \item durch $9$ teilbar, wenn ihre Quersumme durch $9$ teilbar ist.
    \end{itemize}
\end{theorem}

\begin{theorem}{}
    $n\divides p,n\divides q \implies n\divides pq$ (Umkehrung gilt nur für teilerfremde Zahlen).
\end{theorem}

Zusatzwissen: Satz 1.10 im Schütt Skript.

ggt Kapitel: Satz 1.5 bis 1.9
\end{document}