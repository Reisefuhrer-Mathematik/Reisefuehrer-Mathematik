\documentclass[../../main.tex]{subfiles}

\usepackage{tabularray}

\begin{document}

Im letzten Kapitel haben wir gesehen, dass man nicht jede Zahl \enquote{sauber} Teilen kann.
\begin{example}{}
	Wenn Alice, Bob und Charly fünf Pizzen unter einander aufteilen möchten, dann haben wir gesehen, dass jeder eine
	Pizza bekommt und zwei übrig bleiben. Sprich: \enquote{5 Pizzen geteilt durch 3 ist 1 Rest 2}. Wir sehen also, dass
	5 nicht durch 3 teilbar ist.
\end{example}

\begin{remark}{}
	Auch wenn manche Leser bereits mit Brüchen vertraut sind und somit wissen, dass in diesem Beispiel Alice, Bob und
	Charly die letzte Pizza dritteln können, um sie aufzuteilen, sei an dieser Stelle erwähnt, dass die Division mit
	Rest, wie sie in diesem Kapitel behandelt wird, ein sehr wichtiger Teil der Mathematik ist. Insbesondere ist
	Division mit Rest also eine Krücke, die man nur lernt, um sich erst später mit Bruchrechnung beschäftigen zu müssen.
	Ohne weiter darauf einzugehen, sei versichert, dass das Thema Teilbarkeit -- trotz seiner Einfachheit -- heutzutage
	für Bereiche wie zum Beispiel die Kryptographie unabdingbar ist.
\end{remark}

Wir sehen also, dass eine Zahl $a$ teilbar ist durch eine andere, nennen wir sie $b$, wenn die Division keinen Rest
ergibt. Oder, alternativ, wenn wir $a$ mit einer anderen Zahl, multiplizieren können, sodass $b$ rauskommt.

\begin{definition}{Teilbarkeit}
	$b$ ist teilbar durch $a$ genau dann, wenn wir $b$ als Produkt $a\cdot \Box = b$ schreiben können.
\end{definition}

Mit den folgenden Beispielen wird dir bestimmt schnell auffallen, dass die Definition für Teilbarkeit also nichts neues
ist, sondern lediglich eine mathematische Beschreibung von dem, was wir intuitiv schon wissen.

\begin{example}{}
	Wir haben oben schon gesehen, dass 5 nicht durch 3 teilbar ist, weil die Division $5/3 = 1 \text{R: }2$ ergibt. Wir
	können auch argumentieren, dass wir keine Zahl finden können, sodass $3\cdot \Box = 5$. Wenn wir in das Kästchen
	eine 1 schreiben, dann steht da $3\cdot 1 = 3$ und somit ist die Zahl im Kästchen zu klein. Wenn wir aber die 2
	eintragen, kommt $3\cdot 2 = 6$ raus und das Ergebnis ist zu groß.
\end{example}

\begin{example}{}
	
\end{example}

\[\begin{tblr}{
		colspec={c*{11}{c}},
		row{1}={bg=maincolor},
		column{1}={bg=maincolor},
		row{1-Z}={ht=1.5em, rowsep=2pt},
		column{1-Z}={wd=1.5em, colsep=2pt},
		cell{ 3}{2- 3}={bg=maincolor!50!white},
		cell{ 5}{2- 5}={bg=maincolor!50!white},
		cell{ 7}{2- 7}={bg=maincolor!50!white},
		cell{ 9}{2- 9}={bg=maincolor!50!white},
		cell{11}{2-11}={bg=maincolor!50!white},
		cell{2- 3}{ 3}={bg=maincolor!50!white},
		cell{2- 5}{ 5}={bg=maincolor!50!white},
		cell{2- 7}{ 7}={bg=maincolor!50!white},
		cell{2- 9}{ 9}={bg=maincolor!50!white},
		cell{2-11}{11}={bg=maincolor!50!white},
		hline{1,Z}={},
		vline{1,Z}={},
	}
	\cdot &  0 &  1 &  2 &  3 &  4 &  5 &  6 &  7 &  8 &  9 & 10 \\
	    0 &  0 &  0 &  0 &  0 &  0 &  0 &  0 &  0 &  0 &  0 &  0 \\
	    1 &  0 &  1 &  2 &  3 &  4 &  5 &  6 &  7 &  8 &  9 & 10 \\
	    2 &  0 &  2 &  4 &  6 &  8 & 10 & 12 & 14 & 16 & 18 & 20 \\
	    3 &  0 &  3 &  6 &  9 & 12 & 15 & 18 & 21 & 24 & 27 & 30 \\
	    4 &  0 &  4 &  8 & 12 & 16 & 20 & 24 & 28 & 32 & 36 & 40 \\
	    5 &  0 &  5 & 10 & 15 & 20 & 25 & 30 & 35 & 40 & 45 & 50 \\
	    6 &  0 &  6 & 12 & 18 & 24 & 30 & 36 & 42 & 48 & 54 & 60 \\
	    7 &  0 &  7 & 14 & 21 & 28 & 35 & 42 & 49 & 56 & 63 & 70 \\
	    8 &  0 &  8 & 16 & 24 & 32 & 40 & 48 & 56 & 64 & 72 & 80 \\
	    9 &  0 &  9 & 18 & 27 & 36 & 45 & 54 & 63 & 72 & 81 & 90 \\
	   10 &  0 & 10 & 20 & 30 & 40 & 50 & 60 & 70 & 80 & 90 & 100\\
\end{tblr}\]


%\begin{definition}{Darstellung mit Rest}
%\end{definition}

\end{document}