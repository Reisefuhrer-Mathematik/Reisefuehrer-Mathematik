\documentclass[../../main.tex]{subfiles}

\usetikzlibrary{angles,quotes,babel,calc}

\tikzset{
	angle/.style={draw=#1, fill=#1!20},
	angle/.default={maincolor},
}

\begin{document}
\begin{tikzpicture}[very thick]
	\coordinate (A) at (0,0);
	\coordinate (B) at (4,2);
	\coordinate (C) at (4,0);
	\coordinate (D) at (0,2);
	% Winkel
	\pic [draw, angle, "$\alpha$"{orange}, angle radius=1cm] {angle = C--A--B};
	% Scheitelwinkel / Gegenwinkel
	\coordinate (-B) at ($-1*(B)$);
	\coordinate (-C) at ($-1*(C)$);
	\pic [draw, angle=red, "$\alpha$"{red}, angle radius=1cm] {angle = {-C}--A--{-B}};
	% Stufenwinkel
	\coordinate (2B) at ($2*(B)$);
	\coordinate (2B-D) at ($(2B)-(D)$);
	\pic [draw, angle=blue, "$\alpha$"{blue}, angle radius=1cm] {angle = {2B-D}--B--{2B}};
	% Wechselwinkel
	\pic [draw, angle=purple, "$\alpha$"{purple}, angle radius=1cm] {angle = D--B--A};
	% Nebenwinkel
	\pic [draw, angle=green, "$\beta$"{green}, angle radius=1cm] {angle = {-B}--A--C};
	\pic [draw, angle=green, "$\beta$"{green}, angle radius=1cm] {angle = B--A--{-C}};

	\draw[shorten >=-2cm, shorten <=-2cm] (A) edge[green!50!black] (B) edge[blue] (C);
	\draw[shorten >=-2cm, shorten <=-2cm, blue] (B) to[blue] (D);
\end{tikzpicture}
\end{document}