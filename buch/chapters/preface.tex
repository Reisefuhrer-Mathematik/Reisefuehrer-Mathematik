\documentclass[../main.tex]{subfiles}

\begin{document}

\if 0
\textbf{Inhalte des Vorwortes:}
\begin{itemize}
    \item Für wen ist dieses Buch? -- Das Buch ist kostenlos und für jeden, hey! :D
    \item Link zum Internetauftritt, falls vorhanden
    \item Nutzung: Nachschlagewerk, Selbstlernen, Flipped Classroom
    \item Vorstellen unserer Philosophie (Beispiele erst! -> Zitat: Def zuerst ist scheiße)
    \item Erklärung der Buchelemente (Satz, Definition, Zusatzwissen, Nutshell...)
    \item Erläuterung von Flipped Classroom
    \item Sehr großer Dank an alle Autoren und namentliche Nennung mit Bild, falls gewünscht
    \item Dank an alle Korrektoren
    \item Dank an Arne? Weil KVA so toll ist?
\end{itemize}
\fi

Leider sind Lehrer nicht immer die besten Lehrer.

Wer kennt das nicht: bei der ausgebildeten Lehrkraft ist es sehr schwer, dem Unterricht zu folgen und mitzudenken, wohingegen es der Quereinsteiger spaßig und verständlich erklärt. Natürlich ist das keine allgemeingültige Wahrheit. Es gibt auch schwarze Schafe unter den Quereinsteigern und leuchtende Musterbeispiele unter den Lehrern. Leider können derartige Unterschiede in der Qualität des Unterrichts weitreichende Folgen haben. Für 2 Schuljahre hat man einen Lehrer, mit dem man keine Wellenlänge teilt und bei dem man den Stoff nicht versteht. Nun hat er ein möglichst einfaches Fach auf Lehramt studiert, und man darf es ausbaden weil man komplett den Anschluss verloren hat.

Das Ziel dieses Werkes ist zwiefältig. Wir möchten Schülerinnen und Schülern ein \emph{kostenloses} und \emph{frei Verfügbares} Nachschlagewerk sowie Übungsmaterial bieten, um die eigenen Schwächen mit der Mathematik möglichst einfach Nachzuarbeiten. Zudem möchten wir das Konzept des \enquote{\emph{Flipped Classroom}} für Lehrer attraktiv machen.

Flipped Classroom erkennt zwei große Probleme im traditionellen Frontalunterricht: im Unterricht (während die Schüler Zugriff auf den Lehrer und Mitschüler haben) wird still gesessen und wenn der Schüler den Stoff nicht verstanden hat, ist er für die Hausaufgabe \enquote{gearscht} weil er niemanden wirklich fragen kann.
Dieses Problem wird durch Flipped Classroom gelöst, indem diese Konstellation umgedreht wird: zuhause hat der Schüler die Aufgabe, sich mit dem Material zu beschäftigen -- er muss nicht alles verstehen aber sollte ehrlich mit sich selbst sein, und sich das Material gewissenhaft anschauen -- im Klassenzimmer können die Schüler dann Fragen stellen und üben den Stoff. \textbf{Zuhause wird der Stoff (gewissenhaft) vorbereitet und im Klassenzimmer wird er geübt und damit gelernt.}

Wenn du dieses Buch liest, wird dir auffallen, dass der Fließtext immer wieder durch verschiedene, farbig hervorgehobene, Boxen unterbrochen wird. Insbesondere die \enquote{Kurz und knapp} Boxen sind dabei wahrscheinlich interessant. Wenn du ein Unterkapitel bereits verstanden hast, dann solltest du nicht alles nochmal lesen müssen, um es wieder aufzuarbeiten. Die \enquote{Kurz und knapp} Boxen geben dir eine kurze Zusammenfassung der wichtigsten Erkenntnisse des jeweiligen Unterkapitels. Hast du den Schulstoff so weit verstanden und möchtest mehr erfahren oder bist du einfach neugierig? Dann kannst du dir die \enquote{Tiefer in den Hasenbau} Boxen durchlesen. Hier behandelte Themen sind nicht relevant für den Rest des Buches aber sollen interessierten Schülern erlauben, sich selbst herauszufordern.

Bleibt zuletzt noch die Danksagung, die dem Leser nichts bedeutet aber für uns ein wichtiges Zeichen unserer Wertschätzung darstellt.
Dank gebührt allen Autoren, die ihre Freizeit aufgebracht haben, um sich viele Gedanken zu machen und viel Mühe in dieses Lehrbuch und Übungskonzept zu stecken. Ebenso bedanken wir uns bei allen Lehrern und Korrektoren. Insbesondere möchten wir uns bei Arne bedanken, der uns das Konzept des Flipped Classroom gepaart mit minimaler Hilfe in Aktion gezeigt hat. Der größte Dank gebührt aber selbstredend den Katzen der Autoren, ohne die es schlicht nicht möglich gewesen wäre, sich für ein solches Projekt zu motivieren.

{\hfill \Large Gutes Gelingen!}

\bigskip
% Ich hätte nichts dagegen, wenn wir dieses Zitat tatsächlich einfach so lassen
\epigraph{\enquote{Hier kommt ein echt tolles Zitat hin. Es wird ein großartiges Zitat sein. Das beste Zitat, das je jemand gewählt hat. Hoffentlich vergessen wir nicht, hier etwas einzufügen.}}{-- Jemand Wichtiges}

\end{document}