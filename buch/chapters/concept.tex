\documentclass[../main.tex]{subfiles}

\begin{document}
\lipsum[2]

Über das gesamte Buch verteilt sind verschiedene, farbig hervorgehobene, Boxen zu sehen:
\begin{definition}[def:notitle]{}
    Dies ist eine Definition.
\end{definition}
\begin{definition}[def:title]{Title}
    Diese Definition hat einen Namen.
\end{definition}

\begin{example}[ex:sample]{}
    Dies ist ein Beispiel.    
\end{example}

\begin{remark}[rem:notitle]{}
    Dies ist eine Anmerkung.
\end{remark}

\begin{lemma}[lem:mylem]{}
    Dies ist eine Mathematische Aussage, auch als Satz oder -- wenn du vor deinen (danach nicht mehr) Freunden angeben möchtest -- Lemma bezeichnet.
\end{lemma}

\begin{theorem}[thm:mytheorem]{}
    Dies ist ein Theorem. Es ist ähnlich zu einem Lemma aber meist umfassender. Lemmata helfen häufig, um Theoreme zu Beweisen.
\end{theorem}

\begin{corollary}[cor:mycorollary]{}
    Dies ist eine direkte Folgerung aus einer vorangegangenen Aussage, auch Korollar genannt.
\end{corollary}

\begin{advanced}{Beispiel}
    In dieser Umgebung wird zusätzliches Wissen präsentiert.
\end{advanced}
\begin{nutshell}{Beispiel}
    Dies ist eine Kurzzusammenfassung der wichtigsten vorangegangenen Aussagen.
\end{nutshell}

%\begin{tikzpicture}[semithick]
%    \draw[orange, fill=orange!10!white] (0,0) circle (1cm);
%    \fill[black] (-.4,.4) circle (.5mm);
%    \fill[black] (.4,.4) circle (.5mm);
%    \draw[black] (-.2, -.4) -- (.2, -.4);
%    \draw[blue, fill=blue!10!white] (-.6, -3) -- (.6, -3) -- (0, -1) -- cycle;
%\end{tikzpicture}

Referenzen mit \texttt{cleveref}: \cref{def:notitle,def:title,ex:sample,rem:notitle,lem:mylem,thm:mytheorem,cor:mycorollary}.

\end{document}