\documentclass[../../main.tex]{subfiles}

\begin{document}
    \textbf{Variablen} erweitern die Mathematik so, dass wir eine \enquote{Schablone} für Rechnungen vorgeben können, die wir später durch
    Zahlen ausfüllen können. Wenn ein Kaugummi 10 Cent kostet, dann kosten 2 Kaugummis 20 Cent, 3 Kaugummis 30 Cent, usw.
    Wir können damit ganz einfach sehen, dass wir den Preis von mehreren Kaugummis also berechnen können mit
    \[\mtext{Anzahl Kaugummis} \cdot 10\text{ct}.\]
    \mtext{Anzahl Kaugummis} ist bereits eine Variable (da wir ihr einen beliebigen Wert zuweisen können bzw. ihr Wert variabel ist).
    Weil Mathematiker schreibfaul sind (was gut ist) und Leerzeichen oder lange Bezeichnungen in Rechnungen unübersichtlich sind,
    wählt man in der Regel einzelne Buchstaben oder Abkürzungen für Variablen. Wir könnten also auch schreiben:
    \[\mtext{Anz}\cdot 10\text{ct} \text{  oder  } n\cdot 10\text{ct} \text{  oder  } x\cdot 10\text{ct}.\]
    Eine Variable ist also nichts anderes als ein Platzhalter für einen Zahlenwert.
    Damit können wir insbesondere auch bekannte Tatsachen mathematisch formulieren.
    \begin{example}{}
        \parpic[r]{
            \begin{tikzpicture}
                \fill[orange!10, draw=orange, text=orange] (0,0) rectangle (2cm, 1cm) node[pos=.5]{$A$};
                \draw[decorate, decoration = {calligraphic brace, raise=5pt, amplitude=5pt}]
                    (0,0) -- (0,1)
                    node[pos=0.5,left=10pt,black]{$h$};
                \draw[decorate, decoration = {calligraphic brace, raise=5pt, amplitude=5pt}]
                    (0,1) -- (2,1)
                    node[pos=0.5,above=10pt,black]{$b$};
            \end{tikzpicture}
        }
        Die Fläche eines Rechteckes ist seine Breite multipliziert mit der Höhe.
        Als eine Formel mit Variablen können wir das ausdrücken durch
        \[\mtext{Fläche} = \mtext{Breite} \cdot \mtext{Höhe}.\]
        \picskip{0}
        Oder, wenn wir einzelne Buchstaben als Bezeichnungen für unsere Variablen wählen, dann könnten wir denselben Zusammenhang schreiben als
        \[A = b\cdot h\]
        wobei $A$ die Fläche (vom Englischen \enquote{Area}), $b$ die Breite und $h$ die Höhe ist.

        Hier müssen wir (genau wie immer, wenn der Variablenname nicht klar sagt, wofür die Variable steht) angeben, für welchen Sachverhalt die Variable steht
    \end{example}

    Das nächste Beispiel führt uns auf eine interessante Beobachtung, wenn wir mit Variablen rechnen.\todo{vlt schon grob teasern?}
    \begin{example}{}
        Im folgenden Bild siehst du zwei Rechtecke, die dieselbe Höhe haben und nebeneinander liegen. Wir möchten einen Term für ihre Gesamtfläche finden.
        \begin{center}
            \begin{tikzpicture}
                \fill[orange!10, draw=orange, text=orange] (0,0) rectangle (2cm, 1cm) node[pos=.5]{$A_1$};
                \fill[blue!10, draw=blue, text=blue] (2cm,0cm) rectangle (3cm, 1cm) node[pos=.5]{$A_2$};
                \draw[decorate, decoration = {calligraphic brace, raise=5pt, amplitude=5pt}]
                    (0,0) -- (0,1)
                    node[pos=0.5,left=10pt,black]{$h$};
                \draw[decorate, decoration = {calligraphic brace, raise=5pt, amplitude=5pt}]
                    (0,1) -- (2,1)
                    node[pos=0.5,above=10pt,black]{$b_1$};
                \draw[decorate, decoration = {calligraphic brace, raise=5pt, amplitude=5pt}]
                    (2,1) -- (3,1)
                    node[pos=0.5,above=10pt,black]{$b_2$};
            \end{tikzpicture}
        \end{center}
        Die Namen $b_1$, $b_2$, $A_1$ und $A_2$ sind in der Darstellung sicher aufgefallen, wegen der niedergestellten Zahlen.
        Damit ist keine neue Regel verbunden, wir haben einfach gewählt, die Zahlen für die Bezeichner niedergestellt zu schreiben. $A_1$ zum Beispiel kann gelesen werden als \enquote{Fläche von Rechteck 1}.
        Wir könnten natürlich ebenso $\mtext{Fläche von Rechteck 1}$ oder $A1$ oder ähnliches wählen. Analoges gilt auch für die $b$s.
        Wir haben uns für die dargestellten Bezeichner entschieden, da wir das Gefühl haben, dass sie klar darstellen, für welche Eigenschafften der Rechtecke sie stehen.

        Wenn wir uns überlegen, wie wir die Gesamtfläche, nennen wir sie $A_\mtext{Ges}$, berechnen können, finden wir recht schnell zwei Formeln:
        \begin{itemize}[nosep]
            \item Wir überlegen uns, dass $A_\mtext{Ges}$ einfach die Summe von $A_1$ und $A_2$ ist:
                \[A_\mtext{Ges} = A_1 + A_2.\]
                Wir kennen zudem aus dem Beispiel oben die Formeln für $A_1$ und $A_2$ und können also schreiben, dass
                \[A_\mtext{Ges} = h\cdot b_1 + h \cdot b_2.\]
            \item Alternativ können wir uns überlegen, dass die beiden Rechtecke nebeneinander ebenfalls ein Rechteck der Breite $b_1+b_2$ bilden.
                Mit der Formel von oben folgt nun, dass
                \[A_\mtext{Ges} = h\cdot (b_1 + b_2).\]
        \end{itemize}
        Wir haben also zwei unterschiedliche Terme für dieselbe Berechnung gefunden.
        Da beides gleich $A_\mtext{Ges}$ ist, können wir sogar folgern, dass
        \[h\cdot b_1 + h \cdot b_2 = h\cdot (b_1 + b_2).\]
    \end{example}

    \todo{welche Beobachtung? (steht im Beispiel, kann hier aber nochmal genannt werden. Die Redundanz hilft vermutlich)}
    Diese Beobachtung kann verwunderlich sein. Wenn wir mit Zahlen rechnen, dann bekommen wir immer ein eindeutiges Ergebnis (eine Zahl).
    Mit Variablen kommen wir aber auf unterschiedliche Terme, die dasselbe beschreiben.
    Insbesondere könnte uns
    \[h\cdot b_1 + h \cdot b_2 = h\cdot (b_1 + b_2)\]
    auch bekannt vorkommen, da es genau das Distributivgesetz ist (\mayberef).
    Dadurch, dass wir allgemeine Terme über Variablen für den Flächeninhalt eines Rechteckes aufgestellt haben,
    waren wir also in der Lage, auf eine neue, ebenso allgemeingültige Gesetzmäßigkeit (das Distributivgesetz) zu schließen.
    Während diese Erkenntnis mehr Teil der Aussagenlogik (\mayberef) sein wird und hier erstmal nicht explizit behandelt wird,
    motiviert es vor allem das Interesse daran, sich mit Variablen weiter auseinander zu setzen.

    Wir hoffen, dass diese kurze Einleitung in die Variablen dir gezeigt hat, dass man mit Variablen rechnet wie mit jeder beliebigen Zahl.
    Variablen sind also nichts neues. Allerdings helfen sie uns, allgemeine Aussagen zu formulieren, die wir anschließend verwenden können,
    um neues Wissen zu erschließen.

    Der Rest dieses Kapitels wird zuletzt die Begriffe des \enquote{Ausklammerns} und \enquote{Ausmultiplizierens} einführen,
    (hierbei handelt es sich lediglich um neue Begriffe für bereits bekannte Sachen)
    sowie besprechen, wie man Terme über Variablen übersichtlich aufschreiben kann, indem man Terme zusammenfasst.
\end{document}