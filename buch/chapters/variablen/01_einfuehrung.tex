\documentclass[../../main.tex]{subfiles}

\begin{document}
    \textbf{Variablen} erweitern die Mathematik so, dass wir eine \enquote{Schablone} für Rechnungen vorgeben können, die wir später durch
    Zahlen ausfüllen können. Wenn ein Kaugummi 10 Cent kostet, dann kosten 2 Kaugummis 20 Cent, 3 Kaugummis 30 Cent, usw.
    Wir können damit ganz einfach sehen, dass wir den Preis von mehreren Kaugummis also berechnen können mit
    \[\mtext{Anzahl Kaugummis} \cdot 10\text{ct}.\]
    \mtext{Anzahl Kaugummis} ist bereits eine Variable (da wir ihr einen beliebigen Wert zuweisen können bzw. ihr Wert variabel ist).
    Weil Mathematiker schreibfaul sind (was gut ist) und Leerzeichen oder lange Bezeichnungen in Rechnungen unübersichtlich sind,
    wählt man in der Regel einzelne Buchstaben oder Abkürzungen für Variablen. Wir könnten also auch schreiben:
    \[\mtext{Anz}\cdot 10\text{ct} \text{  oder  } n\cdot 10\text{ct} \text{  oder  } x\cdot 10\text{ct}.\]
    Eine Variable ist also nichts anderes als ein Platzhalter für einen Zahlenwert.
    Damit können wir insbesondere auch bekannte Tatsachen mathematisch formulieren.
    \begin{example}{}
        \parpic[r]{
            \begin{tikzpicture}
                \fill[orange!10, draw=orange, text=orange] (0,0) rectangle (2cm, 1cm) node[pos=.5]{$A$};
                \draw[decorate, decoration = {calligraphic brace, raise=5pt, amplitude=5pt}]
                    (0,0) -- (0,1)
                    node[pos=0.5,left=10pt,black]{$h$};
                \draw[decorate, decoration = {calligraphic brace, raise=5pt, amplitude=5pt}]
                    (0,1) -- (2,1)
                    node[pos=0.5,above=10pt,black]{$b$};
            \end{tikzpicture}
        }
        Die Fläche eines Rechteckes ist seine Breite multipliziert mit der Höhe.
        Als eine Formel mit Variablen können wir das ausdrücken durch
        \[\mtext{Fläche} = \mtext{Breite} \cdot \mtext{Höhe}.\]
        \picskip{0}
        Oder, wenn wir einzelne Buchstaben als Bezeichnungen für unsere Variablen wählen, dann könnten wir denselben Zusammenhang schreiben als
        \[A = b\cdot h\]
        wobei $A$ die Fläche (vom Englischen \enquote{Area}), $b$ die Breite und $h$ die Höhe ist.

        Hier müssen wir (genau wie immer, wenn der Variablenname nicht klar sagt, wofür die Variable steht) angeben, für welchen Sachverhalt die Variable steht
    \end{example}

    Das nächste Beispiel führt uns auf eine interessante Beobachtung, wenn wir mit Variablen rechnen.
    Wir sehen nämlich, dass wir dieselbe Rechnung durch unterschiedliche Terme ausdrücken können.
    \begin{example}{}
        Im folgenden Bild siehst du zwei Rechtecke, die dieselbe Höhe haben und nebeneinander liegen. Wir möchten einen Term für ihre Gesamtfläche finden.
        \begin{center}
            \begin{tikzpicture}
                \fill[orange!10, draw=orange, text=orange] (0,0) rectangle (2cm, 1cm) node[pos=.5]{$A_1$};
                \fill[blue!10, draw=blue, text=blue] (2cm,0cm) rectangle (3cm, 1cm) node[pos=.5]{$A_2$};
                \draw[decorate, decoration = {calligraphic brace, raise=5pt, amplitude=5pt}]
                    (0,0) -- (0,1)
                    node[pos=0.5,left=10pt,black]{$h$};
                \draw[decorate, decoration = {calligraphic brace, raise=5pt, amplitude=5pt}]
                    (0,1) -- (2,1)
                    node[pos=0.5,above=10pt,black]{$b_1$};
                \draw[decorate, decoration = {calligraphic brace, raise=5pt, amplitude=5pt}]
                    (2,1) -- (3,1)
                    node[pos=0.5,above=10pt,black]{$b_2$};
            \end{tikzpicture}
        \end{center}
        Die Namen $b_1$, $b_2$, $A_1$ und $A_2$ sind in der Darstellung sicher aufgefallen, wegen der niedergestellten Zahlen.
        Damit ist keine neue Regel verbunden, wir haben einfach gewählt, die Zahlen für die Bezeichner niedergestellt zu schreiben. $A_1$ zum Beispiel kann gelesen werden als \enquote{Fläche von Rechteck 1}.
        Wir könnten natürlich ebenso $\mtext{Fläche von Rechteck 1}$ oder $A1$ oder ähnliches wählen. Analoges gilt auch für die $b$s.
        Wir haben uns für die dargestellten Bezeichner entschieden, da wir das Gefühl haben, dass sie klar darstellen, für welche Eigenschafften der Rechtecke sie stehen.

        Wenn wir uns überlegen, wie wir die Gesamtfläche, nennen wir sie $A_\mtext{Ges}$, berechnen können, finden wir recht schnell zwei Formeln:
        \begin{itemize}[nosep]
            \item Wir überlegen uns, dass $A_\mtext{Ges}$ einfach die Summe von $A_1$ und $A_2$ ist:
                \[A_\mtext{Ges} = A_1 + A_2.\]
                Wir kennen zudem aus dem Beispiel oben die Formeln für $A_1$ und $A_2$ und können also schreiben, dass
                \[A_\mtext{Ges} = h\cdot b_1 + h \cdot b_2.\]
            \item Alternativ können wir uns überlegen, dass die beiden Rechtecke nebeneinander ebenfalls ein Rechteck der Breite $b_1+b_2$ bilden.
                Mit der Formel von oben folgt nun, dass
                \[A_\mtext{Ges} = h\cdot (b_1 + b_2).\]
        \end{itemize}
        Wir haben also zwei unterschiedliche Terme für dieselbe Berechnung gefunden.
        Da beides gleich $A_\mtext{Ges}$ ist, können wir sogar folgern, dass
        \[h\cdot b_1 + h \cdot b_2 = h\cdot (b_1 + b_2).\]
    \end{example}

    Diese Beobachtung, dass das Ergebnis unserer Rechnung zwei verschiebene Terme sein können, kann verwunderlich sein. Wenn wir mit Zahlen rechnen, dann bekommen wir immer ein eindeutiges Ergebnis (eine Zahl).
    Mit Variablen kommen wir aber auf unterschiedliche Terme, die dasselbe beschreiben.
    Insbesondere könnte uns
    \[h\cdot b_1 + h \cdot b_2 = h\cdot (b_1 + b_2)\]
    auch bekannt vorkommen, da es genau das Distributivgesetz ist (\mayberef).
    Dadurch, dass wir allgemeine Terme über Variablen für den Flächeninhalt eines Rechteckes aufgestellt haben,
    waren wir also in der Lage, auf eine neue, ebenso allgemeingültige Gesetzmäßigkeit (das Distributivgesetz) zu schließen.
    Während diese Erkenntnis mehr Teil der Aussagenlogik (\mayberef) sein wird und hier erstmal nicht explizit behandelt wird,
    motiviert es vor allem das Interesse daran, sich mit Variablen weiter auseinander zu setzen.

    Wir hoffen, dass diese kurze Einleitung in die Variablen dir gezeigt hat, dass man mit Variablen rechnet wie mit jeder beliebigen Zahl.
    Variablen sind also nichts neues. Allerdings helfen sie uns, allgemeine Aussagen zu formulieren, die wir anschließend verwenden können,
    um neues Wissen zu erschließen.

    Der Rest dieses Kapitels wird zuletzt die Begriffe des \enquote{Ausklammerns} und \enquote{Ausmultiplizierens} einführen,
    (hierbei handelt es sich lediglich um neue Begriffe für bereits bekannte Sachen)
    sowie besprechen, wie man Terme über Variablen übersichtlich aufschreiben kann, indem man Terme zusammenfasst.

    Da es auf Dauer mühselig ist, Terme wie $4\cdot a\cdot b$ auszuschreiben, führen wir zuletzt noch die Kurznotation ein,
    dass Produkte aus einbuchstabigen Variablennamen auch ohne \enquote{$\cdot$} geschrieben werden dürfen.
    Wir können (müssen aber auch nicht) also statt $4\cdot a\cdot b$ auch $4ab$ schreiben.
    Bevor dadurch komplett merkwürdige Konstrukte entstehen, fordern wir, dass Zahlen, die in dem Produkt vorkommen,
    nach vorne geschrieben werden. Aus $a\cdot b\cdot 4$ würde also nicht $ab4$ sondern auch wieder $4ab$.
    In der Regel sieht man zudem, dass die Variablen sinnvoll (zum Beispiel alphabetisch) sortiert werden.
    Dies dient aber alles lediglicht einer übersichtlichen Notation.
    
    \begin{example}{Notation}
        \begin{tabularx}{\linewidth}{YY|YY}
            \textbf{Term} & \textbf{Kurznotation} & \textbf{Term} & \textbf{Kurznotation}\\
            $x_1\cdot x_2\cdot 12$ & $12x_1x_2$ &
            $12\cdot x_2\cdot x_1$ & $12x_1x_2$\\
            $2\cdot x\cdot 3$ & $6x$ &
            $1\cdot x$ & $x$\\
            $3\cdot a + 4\cdot b$ & $3a+4b$ &
            $3\cdot \mtext{Breite} \cdot \mtext{Höhe}$ & $3\cdot \mtext{Breite} \cdot \mtext{Höhe}$
        \end{tabularx}
    \end{example}

    \begin{nutshell}{Variablen}
        Wenn wir einen allgemeinen Zusammenhang beschreiben möchten, können wir \textbf{Variablen als Platzhalter} für konkrete Werte verwenden.

        So können wir zum Beispiel für ein Rechteck sagen, dass
        \[\mtext{Fläche} = \mtext{Breite} \cdot \mtext{Höhe}.\]
        Bevor wir für die Fläche aber einen konkreten Wert berechnen können, müssen wir für die Variablen erst konkrete Werte \textbf{einsetzen}.

        Wir können denselben Sachverhalt mit verschiedenen Termen über Variablen ausdrücken.
        Dann nennen wir die beiden Terme \textbf{äquivalent}.

        Um uns die Notation von Variablentermen zu vereinfachen, darf das \enquote{$\cdot$} in Produkten mit Variablen weg gelassen werden.
    \end{nutshell}

    \subsection*{Intuition: Einheiten als Variablen}
    Ohne eine gute Intuition, fällt das Rechnen mit Einheiten häufig schwer. Fehler, die bei genauerer Überlegung
    auffliegen, passieren schnell.
    \begin{example}{}
        Wenn ein Kilometer tausend Metern entspricht (in Zeichen: $1\,\text{km} \corresponds 1000\,\text{m}$), denkt man
        leicht, dass dann auch $1\,\text{km}^2 \corresponds 1000\,\text{m}^2$.
        Dass das aber nicht stimmt, sieht man schnell, wenn man sich überlegt, dass ein Quadrat mit einer Seitenlänge
        von $1\,\text{km}$ also eine Seitenlänge von $1000\,\text{m}$ hat und somit eine Fläche von $1000^2\,\text{m}^2$.
    \end{example}

    Mit diesem kurzen Nachtrag möchten wir dir eine andere Perspektive auf Einheiten geben, indem wir Einheiten als eine
    Form von Variablen interpretieren, um das Rechnen mit Einheiten intuitiv und einfach zu machen.
    Wenn du bereits ein gutes Verständnis für das Rechnen mit Einheiten hast, kannst du diesen Abschnitt natürlich
    überspringen. Vielleicht lohnt es sich aber auch, ihn zu lesen, um Einheiten aus einer anderen Perspektive zu
    betrachten.

    Einheiten folgen häufig einem bestimmten Schema.
    \begin{example}{}
        Längen können wir unter anderem in Millimeter, Centimeter, Meter oder Kilometer angeben. Analog verwenden wir
        für Gewichte die Einheiten Milligramm, Gramm und Kilogramm und können Dateigrößen auf dem Computer in Byte,
        Kilobyte, Megabyte, Gigabyte oder auch Terrabyte angeben.

        Alle diese Einheiten haben etwas gemeinsam: Sie bestehen scheinbar aus einer Basiseinheit (Meter, Gramm, Byte),
        der gegebenenfalls eine Präfix vorgestellt wird.
        Dir ist sicher bekannt, dass \enquote{Kilo} Tausend, \enquote{Dezi} Zehntel, \enquote{Centi} Hundertstel, und
        \enquote{Milli} Tausendstel bedeutet, usw.\footnote{Diese Namen werden auch \enquote{SI-Präfixe}
        genannt}.

        Auch wenn diese Einheiten selten verwendet werden, könnten wir auch Dezimeter oder Megameter als
        Längeneinheiten, oder Gewichte beispielsweise in Centigramm angeben.
    \end{example}

    Wir können uns also vorstellen, dass das Präfix $\text{k}$ aus $\text{km}$ eine Art Variable ist mit
    $\text{k} \corresponds 1000$. Ebenso gilt für Dezi $\text{d} \corresponds 0.1$ und so fort.

    \begin{example}{}
        An unserem einleitenden Beispiel bedeutet das also, dass wir auch rechnen können:
        \[1\,\text{km}^2 = 1\,(\text{km})^2 \corresponds 1\,(1000\,\text{m})^2 = 1000^2\,\text{m}^2.\]
    \end{example}

    Interessant wird die Anschauung, wenn wir uns somit überlegen, dass wir in Einheiten kürzen können.

    \begin{example}{Einheiten kürzen}
        Zum Beispiel können wir in der Geschwindigkeit von einem Millimeter pro Millisekunde,
        $1\,\frac{\text{mm}}{\text{ms}}$, das \enquote{Milli} kürzen
        \[
            1\frac{\text{mm}}{\text{ms}} \corresponds 1\frac{\cancel{0.001}\text{m}}{\cancel{0.001}\text{s}}
            = 1\,\frac{\text{m}}{\text{s}}
        \]
        und bekommen eine Geschwindigkeit von einem Meter pro Sekunde.        
    \end{example}

    \textbf{Aber Achtung!} Man muss immer darauf achten, dass die Einheit, die man tatsächlich verwendet, dem
    Entspricht, was man einsetzt.
    \begin{example}
        Zum Beispiel würde es keinen Sinn ergeben, aus Millimetern, also $\text{mm}$, sowas zu machen wie $\text{m}^2$
        oder sogar
        \[\text{mm} = 0.001\,\text{m} = 0.001\cdot 0.001.\]
        Die Ersetzung $1\,\text{m} \corresponds 0.001$ geht nur da, wo $\text{m}$ tatsächlich für das SI-Präfix
        \enquote{Milli} steht.
        Ein anderes Beispiel wäre eine Konzentration, wie sie in der Chemie vorkommt, mit der Einheit
        \enquote{Mol pro Milliliter}, $\frac{\text{mol}}{\text{ml}}$.
        Das $\text{m}$ im Zähler ist Teil der Einheitenbezeichnung für Mol und kann damit nicht mit dem Präfix
        $\text{m}$ im Nenner, das für \enquote{Milli} steht, gekürzt werden.

        Wir können aber dennoch Umrechnungen vornehmen, wenn wir zum Beispiel eine Geschwindigkeit haben wie
        $1\,\frac{\text{m}}{\text{ms}}$.
        Indem wir für \enquote{Milli} einsetzen, bekommen wir
        \[
            1\,\frac{\text{m}}{\text{ms}} \corresponds 1\,\frac{\text{m}}{10^{-3}\,\text{s}}
            = 1\,\frac{10^3\,\text{m}}{\text{s}} \corresponds 1\,\frac{\text{km}}{\text{s}}.
        \]
        Und wie schaut es aus, wenn wir gesagt bekommen, dass wir das in $\frac{\text{km}}{\text{h}}$ umrechnen sollen?
        Wir wissen ja, dass $1\,\text{h} \corresponds 60\,\text{min} \corresponds 60\cdot 60\,\text{s} = 3600\,\text{s}$.
        Wenn wir beide Seiten durch $3600$ dividieren, bekommen wir also: $1\,\text{s} \corresponds 3600^{-1}\,\text{h}$.
        Diesen Wert können wir für $\text{s}$ in unserer Einheit einsetzen und bekommen:
        \[
            1\,\frac{\text{km}}{\text{s}} \corresponds 1\,\frac{\text{km}}{3600^{-1}\,\text{h}}
            = 1\,\frac{3600\,\text{km}}{\text{h}} = 3600\,\frac{\text{km}}{\text{h}}.
        \]
    \end{example}
\end{document}