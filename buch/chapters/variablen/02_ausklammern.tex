\documentclass[../../main.tex]{subfiles}

\begin{document}
	\textcolor{red}{Hier entsteht Magie!}

	In der Einleitung haben wir gesehen, dass wir dieselbe Rechnung durch verschiedene Terme darstellen können.
	Denselben Effekt können wir auch für das Rechnen mit Zahlen betrachten: $2+3$ und $3+2$ sind derselbe Term
	wegen des \enquote{Vertauschungsgesetzes} (Kommutativität der Addition \mayberef).

	Betrachten wir das Distributivgesetz, dann sagt es uns, dass wir Terme der Form $a\cdot (b+c)$ auch
	darstellen können als $a\cdot b + a\cdot c$ und umgekehrt. Bisher war diese Erkenntnis vielleicht noch nicht
	besonders relevant, und hat das Kopfrechnen höchstens vereinfacht. $12\cdot 8 + 12\cdot 2$ ist wesentlich
	einfacher im Kopf zu berechnen, wenn man sieht, dass
	\[12\cdot 8 + 12\cdot 2 = 12\cdot(8+2) = 12\cdot 10 = 120.\]

	Wenn wir mit Variablen rechnen, ist das ähnlich.

	\todo{\enquote{Einsetzen} von Werten für Variablen definieren}
	
	\begin{example}{}
		Wir betrachten wieder das Beispiel aus der Einführung. Du hast zwei Rechtecke, die nebeneinander gelegt
		ein neues Rechteck bilden und möchtest nun einen Ausdruck für die Gesamtfläche angeben.
		\begin{center}
            \begin{tikzpicture}
                \fill[orange!10, draw=orange, text=orange] (0,0) rectangle (2cm, 1cm) node[pos=.5]{$A_1$};
                \fill[blue!10, draw=blue, text=blue] (2cm,0cm) rectangle (3cm, 1cm) node[pos=.5]{$A_2$};
                \draw[decorate, decoration = {calligraphic brace, raise=5pt, amplitude=5pt}]
                    (0,0) -- (0,1)
                    node[pos=0.5,left=10pt,black]{$h$};
                \draw[decorate, decoration = {calligraphic brace, raise=5pt, amplitude=5pt}]
                    (0,1) -- (2,1)
                    node[pos=0.5,above=10pt,black]{$b_1$};
                \draw[decorate, decoration = {calligraphic brace, raise=5pt, amplitude=5pt}]
                    (2,1) -- (3,1)
                    node[pos=0.5,above=10pt,black]{$b_2$};
            \end{tikzpicture}
        \end{center}
		Wir erinnern uns, dass wir die beiden Formeln $(h\cdot b_1 + h\cdot b_2)$ und $h\cdot (b_1+b_2)$
		bisher erarbeitet haben und bisher nicht sagen konnten, welche von beiden \enquote{besser} ist als die
		andere.

		Wir betrachten aber jetzt den Fall, dass wir für zwei der drei Variablen Werte bekommen haben und möchten
		jetzt für diese Gegebene Werte eine Formel angeben für die Gesamtfläche:
		\begin{itemize}
			\item Wir setzen $b_1=2$ und $b_2=3$ in beide Formeln ein. Dann bekommen wir die Terme:
				\[(h\cdot 2 + h\cdot 3) \text{ bzw. } h\cdot(2+3).\]
				Während wir im linken Term keine Teilterme mehr zusammenfassen können, wissen wir im rechten Term, dass $2+3=5$.
				Somit können wir auch schreiben
				\[(h\cdot 2 + h\cdot 3) \text{ bzw. } h\cdot 5.\]
				Hier ist aber ganz klar der rechte Term besser weil wir $h$ nur noch an einer Stelle einsetzen müssen.
				Wir wissen ja, dass wir $h$ mit dem Distributivgesetz vor die Klammern ziehen können. Daher sagen wir
				im zweiten Term, wir haben \enquote{$h$ ausgeklammert}.
			\item Wir setzen $h=3$ und $b_1=2$ in beide Formeln ein. Dann bekommen wir die Terme:
				\[(3\cdot2+3\cdot b_2) \text{ bzw. } 3\cdot(2+b_2).\]
				Hier sehen wir, dass der linke Term weiter vereinfacht werden kann:
				\[6+3\cdot b_2 \text{ bzw. } 3\cdot(2+b_2).\]
				Und in diesem Fall ist also der linke Term übersichtlicher.
		\end{itemize}
	\end{example}

	Das Beispiel zeigt uns, dass wir bei Termen mit Formeln uns überlegen können, wie wir sie darstellen können, um sie übersichtlich
	zu halten. Wir möchten an dieser Stelle aber nochmal betonen, dass diese beiden Terme, die wir betrachtet haben, immer durch das
	Distributivgesetz ineinander überführt werden können, wenn wir die Teilterme nicht weiter ausgerechnet haben.

	Konkret haben wir gesehen, dass wir Faktoren aus Termen rausziehen können:
	\[{\color{orange}a}\cdot b + {\color{orange}a}\cdot c = {\color{orange}a\cdot (}b+c{\color{orange})}.\]
	Dasselbe gilt entsprechend auch für 4 Variablen
	\[{\color{orange}a}\cdot b + {\color{orange}a}\cdot c + {\color{orange}a}\cdot d = {\color{orange}a\cdot (}b+c{\color{orange})} + {\color{orange}a}\cdot d = {\color{orange}a\cdot (}b+c+d{\color{orange})}\]
	beziehungsweise nach demselben Prinzip für beliebig viele Variablen.

	Wenn wir eine Variable, die in allen Summanden vorkommt, so aus einer Summe ziehen, dann bezeichnen wir das auch als \enquote{ausklammern}.
	In die andere Richtung bezeichnen wir den Vorgang als \enquote{ausmultiplizieren}.

	\begin{example}{}
		Alice hat $n$ Freunde. Für jeden kauft sie ein kleines Geschenk im Wert von $5\,\euro{}$ und Schokolade im Wert von $1\,\euro{}$.
		Da sie auch Hungrig ist auf Schokolade, kauft sie für sich selbst zusätzlich zwei Schokoladen. Um die Geschenke zu verpacken, kauft sie
		nocht Geschenkpapier im Wert von $1\,\euro{}$. Wie viel Geld gibt Alice aus?\\[.5em]
		Wir können den Aufgabentext direkt in einen Term übersetzen:
		\[n\cdot 5\,\euro{}+(n+2)\cdot 1\,\euro{} + 1\,\euro{}.\]
		Momentan sagt der Term damit aber leider nicht viel mehr aus als die Aufgabenstellung (lediglich etwas übersichtlicher dargestellt).
		Um die Ausgaben von Alice darstellen zu können, möchten wir den Term so umformen, dass $n$ nur noch an einer Stelle vorkommt.
		Dafür multiplizieren wir den Term $(n+2)\cdot 1\,\euro{}$ erstmal aus und können zuletzt $n$ aus den meisten Termen ausklammern.
		\textbf{Achtung. Wir können n nicht aus allen Termen ausklammern.}
		\begin{align*}
			n\cdot 5\,\euro{} + \underbrace{(n+2)\cdot 1\,\euro{}}_{n\cdot 1\,\euro{}+2\cdot 1\,\euro{}} + 1\,\euro{} &= n\cdot 5\,\euro{}+n\cdot 1\,\euro{} + 2\,\euro{}+1\,\euro{}\\
			 &= n\cdot(5\,\euro{}+1\,\euro{})+2\,\euro{}+1\,\euro{}\\
			 &= n\cdot 6\,\euro{} + 3\,\euro{}.
		\end{align*}
	\end{example}
	\todo{Schwere Aufgabe: Gebe eine Linearfaktorisierung für $(x^3-xy^2+x^2y-y^3) = (x+y)^2(x-y)$ an.}

	\begin{nutshell}{Variablen und Terme über Variablen}
		Variablen kann man sich als \enquote{Platzhalter} oder \enquote{Schablone} vorstellen: ein Term mit Variablen gilt allgemein, egal, was wir einsetzen.

		Zum Beispiel ist $A = b\cdot h$ die Rechenvorschrift, wie man aus der Breite $b$ und der Höhe $h$ eines Rechteckes dessen Fläche $A$ bestimmt.
		Möchten wir für ein konkretes Rechteck (zum Beispiel $b=3$ und $h=2$) die Fläche berechnen, dann müsen wir die Werte für die Variablen \textbf{einsetzen}:
		$A = 3\cdot 2 = 6$. Die Fläche dieses Rechteckes ist also $6$.

		Häufig können wir denselben Term auf verschiedene Weisen darstellen. Wenn ein Term nur wenige Variablen besitzt, möchten wir ihn so formulieren,
		dass er möglichst unverschachtelt ist:
		\begin{center}
			\begin{tabular}{r@{ statt }l}
				$a\cdot b + b$ & $(a+1)\cdot b$\\
				$4\cdot b$ & $3\cdot b + b$\\
				$9+12\cdot a$ & $3\cdot (3 + 4\cdot a)$
			\end{tabular}
		\end{center}

		Wenn wir eine Variable aus einem Teilterm rausziehen, nennen wir das \textbf{ausklammern}:\\
		Zum Beispiel können wir $a$ aus dem Term $a\cdot 3+7\cdot a$ ausklammern und bekommen $a\cdot(3+7)$.

		Wenn wir eine Variable, die mit einem Term multipliziert wird, stattdessen mit jedem Summanden in diesem Term multiplizieren, nennen wir das \textbf{ausmultiplizieren}:\\
		Zum Beispiel können wir $a\cdot (3+7)$ ausmultiplizieren (in diesem Beispiel lediglich wenig sinnvoll) und bekommen wieder $a\cdot 3+ a\cdot 7$.
	\end{nutshell}

	Bisher haben wir immer nur von Addition und Multiplikation gesprochen. Wie sieht das aus, wenn wir subtrahieren und / oder dividieren?
	Nicht anders! Weil eine Subtraktion mit $a$ dasselbe ist wie die Addition von $(-a)$ und die Division durch $a$ dasselbe ist wie die Multiplikation mit $\frac{1}{a}$.
	\begin{example}{}
		\begin{itemize}
			\item Klammere aus: $\left(\frac{3}{a}+\frac{b}{a}\right)$. Wir können die Division mit $a$ als Multiplikation mit $\frac{1}{a}$ betrachten und diesen Bruch entsprechend rausziehen:
			$(3+b)\cdot \frac{1}{a}$. Wir können dies natürlich auch hübscher schreiben als $\frac{3+b}{a}$.
		\end{itemize}
	\end{example}
	\todo{Schwere Aufgabe (?): Klammere aus: (3/{2a} + b/a) = (3/2+b)/a oder (3+2b)/{2a}}
\end{document}