\documentclass[../../main.tex]{subfiles}

\begin{document}

	\textcolor{red}{Hier entsteht Magie!}

	\begin{theorem}
		Für beliebige Zahlen $a$, $b$ und $c$ gilt, dass
		\[a\cdot (b+c) = a\cdot b + a\cdot c.\]
		Wenn wir aus dem Term $a\cdot b + a\cdot c$ den Term $a \cdot (b+c)$ machen, sagen wir wir \enquote{klammern $a$ aus}.
		Wenn wir in die andere Richtung aus $a \cdot (b+c)$ den Term $a\cdot b + a\cdot c$ machen, nennen wir das \enquote{ausmultiplizieren}.
	\end{theorem}
\end{document}