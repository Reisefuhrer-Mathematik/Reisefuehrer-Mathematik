\documentclass[../../main.tex]{subfiles}

\begin{document}
    \newcommand{\km}{\ensuremath{\,\mathrm{km}}}
    \newcommand{\minute}{\ensuremath{\,\mathrm{min}}}
    \newcommand{\s}{\ensuremath{\,\mathrm{s}}}

    \label{chap:einheiten}
    Wie weit ist es von hier bis Berlin? Wie viel Mehl gehört in den Kuchenteig und wie lange muss der Kuchen in den Ofen? 
    Um für solche Fragen sinnvolle Antworten zu finden, benötigen wir Einheiten. Wir schauen uns in diesem Kapitel 
    zunächst die wichtigsten Einheiten an und klären anschließend die Frage, wie du damit rechnen kannst.

    \begin{example}{Geburtstagskuchen-Beispiel}
        \parpic[r]{
            \tikz{
                \node[align=center] at (0,0) {
                    \small 250g Mehl\\
                    \small 210g Butter\\
                    \small 100g Mandeln\\
                    \small 80g Zucker\\
                    \small 2 Pck. Vanillezucker\\
                    \small Puder- und Vanillezucker};
            }
        }
        Stell dir vor, du möchtest Kekse nach deinem Lieblingsrezept backen. In deinem Lieblingsrezept findest du die rechts abgebildete Zutatenliste.
        
    \end{example}

    \begin{example}{Streckenaddition}
        Stell dir vor, du 
    \end{example}

    Zahlenterm vs. gleicher Term mit Einheiten

    Werfen wir als zunächst einmal einen Blick darauf, welche Einheiten es überhaupt gibt. Die Einheit \textbf{Gramm} für das
    Messen oder Vergleichen von Gewichten kennen wir jetzt bereits. Wenn wir angeben möchten, \emph{wie schwer} etwas ist, verwenden wir also
    die Einheit \emph{Gramm}. Du hast vermutlich auch schon häufiger gesehen, dass das Gewicht von schwereren Gegenständen in \emph{Kilogramm} oder \emph{Tonnen} angegeben wird. 

    \begin{center}
        \begin{tabular}{cccc}\toprule
            \textbf{Messgröße} & \textbf{Einheit} & \textbf{Abkürzung}\\\midrule
            Masse/Gewicht & Gramm & g\\
            Länge & Meter & m\\
            Zeit & Sekunde/Minute/Stunde & s/min/h\\
            Volumen & Liter & l\\
            \bottomrule
        \end{tabular}
    \end{center}

    Diese Liste werden wir in diesem Kapitel noch ein wenig erweitern
    % Volumen, Fläche, Geld, Prozent

    Wenn wir hingegen Längen beschreiben wollen, kommen wir mit der Einheit Gramm selbstverständlich nicht mehr weiter. Stattdessen könnten wir zum Beispiel sagen, dass der Schulrekord im Weitsprung bei 5 Metern liegt.

    \begin{advanced}{SI-Basiseinheiten}
        Beginnend mit einem Auftrag der französischen Nationalversammlung wurde im Jahr 1790 damit begonnen, ein einheitliches Einheitensystem zu entwickeln, das für eine einfachere weltweite Zusammenarbeit sorgen sollte. Das dadurch entstandene SI-Einheiten-System (\emph{Système international d'unités}) ist heute das am weitesten verbreitete Einheitensystem. Die Idee dieses Systems ist es, dass nur sieben \textbf{Basiseinheiten} benötigt werden, von denen sich alle weiteren Einheiten ableiten lassen:
        \begin{center}
            \begin{tabular}{cccc}\toprule
                \textbf{Messgröße} & \textbf{Basiseinheit} & \textbf{Abkürzung}\\\midrule
                Masse/Gewicht & Kilogramm & kg\\
                Länge & Meter & m\\
                Zeit & Sekunde & s\\
                Temperatur & Kelvin & K\\
                Lichtstärke & Candela & cd\\
                Stoffmenge & Mol & mol\\
                Elektrische Stromstärke & Ampère & A\\
                \bottomrule
            \end{tabular}
        \end{center}
        Zum Beispiel kommt in dieser Tabelle die Einheit \emph{Minute} nicht vor, denn es gilt
        $1\minute=60\s$, sodass wir immer, wenn wir $1\minute$ schreiben würden, stattdessen auch $60\s$ schreiben könnten und somit auch ohne die Einheit \emph{Minute} auskommen können. Wie wird nun aber festgelegt, wie viel genau eine Sekunde oder ein Kilogramm ist? Die durchaus kuriosen Methoden dafür stellen wir auf Seite \pageref{si-units-history} vor (ebenso wie die Methodik, mit der das heute gemacht wird).
    \end{advanced}

    Zusammenfassen von Termen nur bei gleicher Einheit

    \begin{example}{Nur gleichartige Einheiten zusammenfassen}
    \end{example}

\end{document}