\makeatletter
\renewcommand{\PackageInfo}[2]{}% Remove package information
\renewcommand{\@font@info}[1]{}% Remove font information
\renewcommand{\@latex@info}[1]{}% Remove LaTeX information
\makeatother

\documentclass{buch}
\usepackage{lmodern}
\usepackage[T1]{fontenc}
\usepackage{ae,aecompl}
\usepackage{formulary}
\usepackage{mathdef}
\usepackage{tikzdef}

\usepackage{collcell}

\usepackage{amsthm}
\usepackage[]{todonotes}
\usepackage{booktabs}
\usepackage{transparent}
\usepackage{diagbox}
\usepackage{multirow, makecell, hhline}
\usepackage{picins}
\usepackage{clipboard}
\usepackage{xifthen}
\usepackage{skak}
\usepackage{cancel}
\usepackage{xfrac}

\newclipboard{chapters}

%%%%%%%%%%%%%%%%%%%%
% Characters
%%%%%%%%%%%%%%%%%%%%
\usepackage{epsdice}

%%%%%%%%%%%%%%%%%%%%
% Quotes
%%%%%%%%%%%%%%%%%%%%
\usepackage{epigraph}
\setlength\epigraphwidth{.8\textwidth}
\setlength\epigraphrule{0pt}

%%%%%%%%%%%%%%%%%%%%
% Blind text
%%%%%%%%%%%%%%%%%%%%
\usepackage{lipsum}

\parskip=9pt plus 11 pt

\usetikzlibrary{positioning}
\tikzset{>=stealth}
\graphicspath{../}

%%%%%%%%%%%%%%%%%%%%
% Timer for each page
%%%%%%%%%%%%%%%%%%%%
\usepackage{atbegshi}
\newcommand\showtimer{%
  \message{^^Jtimer: \the\numexpr\the\pdfelapsedtime*1000/65536\relax}%
  \pdfresettimer}
\AtBeginDocument{\showtimer}
\AtBeginShipout {\showtimer}

\usepackage{scrhack} %I guess?

\newcommand{\ifdraft}[2]{#2}

\usepackage{eso-pic}
\usepackage{transparent}

\newcommand{\declchapter}[2]{\chapter{#2}\label{chap:#1}\vphantom{\Copy{chap:lbl_#1}{#2}}}

\bookseries{Reiseführer\\Mathematik}
\volume{1}
\title{}%Sehenswürdigkeiten \& Wanderrouten}
\begin{document}
    \frontmatter
    %\maketitle
    \setcounter{tocdepth}{1}
    {
      \makeatletter
      \renewcommand{\chapterlinesformat}[3]{%
        \tikz{
            \newcommand{\tikzlinghookbelly}{
              \node[
                anchor=south,
                fill=\thing@signback,
                draw=\thing@signcolour,
                rounded corners=\scalingfactor*1,
                text=white,
                line width=\scalingfactor*1.8pt,
                yshift=-.2cm] {\bfseries#3};
            }
            \bee;
        }\par
        \makeatother
    }
    \tableofcontents
    
    \mainmatter
    
    \declchapter{integrale}{Integralrechnung}
    \ldsection{Einführung}
    \ldsection{Das Riemann-Integral}
    \ldsection{Der Hauptsatz der Differential- und Integralrechnung}
    \ldsection{Von Integralen zu Flächen}
    \ldsection{Uneigentliche Integrale}
    \ldsection{Rotationsvolumen}
    \ldadvsection{Weiterführendes Wissen}
    
    \backmatter
    \printindex
    
    \subfile{formulary}
    
\end{document}