\makeatletter
\renewcommand{\PackageInfo}[2]{}% Remove package information
\renewcommand{\@font@info}[1]{}% Remove font information
\renewcommand{\@latex@info}[1]{}% Remove LaTeX information
\makeatother

\documentclass{buch}
\usepackage[T1]{fontenc}
\usepackage{ae,aecompl}
\usepackage{formulary}
\usepackage{mathdef}
\usepackage{tikzdef}

\usepackage{collcell}

\usepackage[T1]{fontenc}

\usepackage{amsthm}
\usepackage[]{todonotes}
\usepackage{booktabs}
\usepackage{enumitem}
\usepackage{transparent}
\usepackage{diagbox}
\usepackage{multirow, makecell, hhline}
\usepackage{picins}
\usepackage{clipboard}
\usepackage{xifthen}

\newclipboard{chapters}

%%%%%%%%%%%%%%%%%%%%
% Characters
%%%%%%%%%%%%%%%%%%%%
\usepackage{epsdice}

%%%%%%%%%%%%%%%%%%%%
% Quotes
%%%%%%%%%%%%%%%%%%%%
\usepackage{epigraph}
\setlength\epigraphwidth{.8\textwidth}
\setlength\epigraphrule{0pt}

%%%%%%%%%%%%%%%%%%%%
% Blind text
%%%%%%%%%%%%%%%%%%%%
\usepackage{lipsum}

\parskip=9pt plus 11 pt

\usetikzlibrary{positioning}
\tikzset{>=stealth}
\graphicspath{../}

%%%%%%%%%%%%%%%%%%%%
% Timer for each page
%%%%%%%%%%%%%%%%%%%%
\usepackage{atbegshi}
\newcommand\showtimer{%
  \message{^^Jtimer: \the\numexpr\the\pdfelapsedtime*1000/65536\relax}%
  \pdfresettimer}
\AtBeginDocument{\showtimer}
\AtBeginShipout {\showtimer}

\usepackage{scrhack} %I guess?


%\newcommand{\ifdraft}[2]{#1}
\newcommand{\ifdraft}[2]{#2}

\usepackage{eso-pic}
\usepackage{transparent}
%\AddToShipoutPictureFG{%
%    \transparent{.5}{\tikz[overlay,remember picture]{
%        \path (current page.south west) -- (current page.north east)
%     node[midway,scale=6,color=gray,sloped, align=center] {\Large ENTWURF\\{\small\today}};}}
%}

\newcommand{\declchapter}[2]{\chapter{#2}\label{chap:#1}\vphantom{\Copy{chap:lbl_#1}{#2}}}

%\bookseries{Zahlen, Formeln\\\& ganz viel Magie}
%\volume{1}
%\title{Mathematik für Uneingeweihte}
\bookseries{Reiseführer\\Mathematik}
\volume{1}
\title{}%Sehenswürdigkeiten \& Wanderrouten}
\begin{document}
    \frontmatter
    %\maketitle
    \setcounter{tocdepth}{1}
    \tableofcontents
    
    %\chapter{\prefacename}
    %\subfile{chapters/preface}
    
    %\chapter{Konzept}
    %\subfile{chapters/concept}
    %\end{train}
    
    \mainmatter
    
    %%%%%%%%%%%%%%%%%%%%%%%%%%%%%%%%%%%%%%%%%%%%%%%%%%%
    % GRUNDLAGEN
    %%%%%%%%%%%%%%%%%%%%%%%%%%%%%%%%%%%%%%%%%%%%%%%%%%%
    %\part{Grundlagen}
    
    \begin{wip}
    \declchapter{grundrechnen}{Grundrechenarten}
    
    \declchapter{teilbarkeit}{Teilbarkeit}
    
    \declchapter{einheiten}{Rechnen mit Einheiten}
    
    \declchapter{geominebene}{Geometrie in der Ebene}
    
    \declchapter{ganzzahl}{Ganze Zahlen}
    
    \declchapter{koords}{Koordinatensysteme}
    
    \declchapter{bruche}{Bruchrechnung}
    
    \declchapter{vars}{Variablen}
    \ldsection{Einführung}
    
    \declchapter{aussagenlogik}{Aussagenlogik}
    \ldsection{Einführung}
    \ldsection{Aussagen in der Mathematik}
    
    \declchapter{mengen}{Mengen}
    \ldsection{Einführung}
    \ldsection{Mengendiagramme}
    \ldsection{Teilmengen}
    \ldsection{Zahlenmengen}
    \ldsection{Intervalle}
    \ldsection{Das kartesische Produkt}
    \end{wip}
    
    \declchapter{funktionen}{Funktionen}
    \ldsection{Einführung}
    \ldsection{Abbildungsvorschriften beschreiben}
    \ldsection{Funktionsgraphen}
    \ldsection{Verkettung von Funktionen}
    \ldsection{Umkehrfunktion und Identität}
    \ldsection{Nullstellen und Achsenabschnitt}
    \ldsection{Symmetrien von Funktionsgraphen}
    \ldsection{Funktionsgraphen verschieben}
    \ldsection{Funktionsgraphen strecken und stauchen}
    \ldadvsection{Weiterführendes Wissen}
    
    %%%%%%%%%%%%%%%%%%%%%%%%%%%%%%%%%%%%%%%%%%%%%%%%%%%
    % MITTELSTUFE
    %%%%%%%%%%%%%%%%%%%%%%%%%%%%%%%%%%%%%%%%%%%%%%%%%%%
    %\part{Mittelstufenmathematik}
    
    \begin{wip}
    \declchapter{statistik}{Beschreibende Statistik}
    
    \declchapter{kombinatorik}{Kombinatorik}
    \ldsection{Einführung}
    \ldsection{Permutationen}
    \ldsection{Urnenexperimente}
    \ldsection{Eulersche Phi Funktion}
    \ldadvsection{Für Fortgeschrittene}
    
    \declchapter{wkeitstheorie}{Wahrscheinlichkeitstheorie}
    
    \declchapter{lingleichungen}{Lineare Gleichungen}
    \ldsection{Einführung}
    \ldsection{Äquivalenzumformungen}
    \ldsection{Auflösen linearer Gleichungen}
    \ldsection{Auflösen linearer Ungleichungen}
    \ldsection{Geradengleichungen}
    \ldsection{Punkte auf Funktionsgeraden}
    \ldadvsection{Weiterführendes Wissen}
    \declchapter{lgs}{Lineare Gleichungssysteme}
    
    \declchapter{quadgleichungen}{Quadratische Gleichungen}
    
    \declchapter{geomkonstr}{Geometrische Konstruktionen}
    
    \declchapter{geomraum}{Geometrie im Raum}
    
    \declchapter{kreise}{Kreise}
    
    \declchapter{potulog}{Potenz- und Logarithmenrechnung}
    
    \declchapter{trig}{Trigonometrie}
    \end{wip}
    
    \if 0
    %%%%%%%%%%%%%%%%%%%%%%%%%%%%%%%%%%%%%%%%%%%%%%%%%%%
    % OBERSTUFE
    %%%%%%%%%%%%%%%%%%%%%%%%%%%%%%%%%%%%%%%%%%%%%%%%%%%
    %\part{Oberstufenmathematik}
    
    \begin{wip}
    \declchapter{folgenundreihen}{Folgen und Reihen}
    
    \declchapter{diff}{Differentialrechnung in \Real}
    \def\Chaptername{differentialrechnung}
    \ldsection{Einführung}
    \ldsection{Tangentensteigung}
    
    \declchapter{interpol}{Interpolation}
    
    \declchapter{integrale}{Integralrechnung}
    %\ldsection{Einführung}
    
    \declchapter{vektoren}{Vektorrechnung}
    
    \declchapter{zufallsvars}{Zufallsvariablen und Verteilungen}
    \end{wip}
    
    \backmatter
    \printindex
    
    \begin{wip}
    \subfile{formulary}
    \end{wip}
    \fi
\end{document}