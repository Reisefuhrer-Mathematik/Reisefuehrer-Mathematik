\makeatletter
\renewcommand{\PackageInfo}[2]{}% Remove package information
\renewcommand{\@font@info}[1]{}% Remove font information
\renewcommand{\@latex@info}[1]{}% Remove LaTeX information
\makeatother

\documentclass{buch}
\usepackage{lmodern}
\usepackage[T1]{fontenc}
\usepackage{ae,aecompl}
\usepackage{formulary}
\usepackage{mathdef}
\usepackage{tikzdef}

\usepackage{collcell}

\usepackage{amsthm}
\usepackage[]{todonotes}
\usepackage{booktabs}
\usepackage{transparent}
\usepackage{diagbox}
\usepackage{multirow, makecell, hhline}
\usepackage{picins}
\usepackage{clipboard}
\usepackage{xifthen}
\usepackage{skak}
\usepackage{cancel}
\usepackage{xfrac}

\newclipboard{chapters}

%%%%%%%%%%%%%%%%%%%%
% Characters
%%%%%%%%%%%%%%%%%%%%
\usepackage{epsdice}

%%%%%%%%%%%%%%%%%%%%
% Quotes
%%%%%%%%%%%%%%%%%%%%
\usepackage{epigraph}
\setlength\epigraphwidth{.8\textwidth}
\setlength\epigraphrule{0pt}

%%%%%%%%%%%%%%%%%%%%
% Blind text
%%%%%%%%%%%%%%%%%%%%
\usepackage{lipsum}

\parskip=9pt plus 11 pt

\usetikzlibrary{positioning}
\tikzset{>=stealth}
\graphicspath{../}

%%%%%%%%%%%%%%%%%%%%
% Timer for each page
%%%%%%%%%%%%%%%%%%%%
\usepackage{atbegshi}
\newcommand\showtimer{%
  \message{^^Jtimer: \the\numexpr\the\pdfelapsedtime*1000/65536\relax}%
  \pdfresettimer}
\AtBeginDocument{\showtimer}
\AtBeginShipout {\showtimer}

\usepackage{scrhack} %I guess?


%\newcommand{\ifdraft}[2]{#1}
\newcommand{\ifdraft}[2]{#2}

\usepackage{eso-pic}
\usepackage{transparent}
%\AddToShipoutPictureFG{%
%    \transparent{.5}{\tikz[overlay,remember picture]{
%        \path (current page.south west) -- (current page.north east)
%     node[midway,scale=6,color=gray,sloped, align=center] {\Large ENTWURF\\{\small\today}};}}
%}

\newcommand{\declchapter}[2]{\chapter{#2}\label{chap:#1}\vphantom{\Copy{chap:lbl_#1}{#2}}}

\bookseries{Reiseführer\\Mathematik}
\volume{1}
\title{}%Sehenswürdigkeiten \& Wanderrouten}
\begin{document}
    \frontmatter
    \maketitle
    \setcounter{tocdepth}{1}
    {
      \makeatletter
      \renewcommand{\chapterlinesformat}[3]{%
        \tikz{
            \newcommand{\tikzlinghookbelly}{
              \node[
                anchor=south,
                fill=\thing@signback,
                draw=\thing@signcolour,
                rounded corners=\scalingfactor*1,
                text=white,
                line width=\scalingfactor*1.8pt,
                yshift=-.2cm] {\bfseries#3};
            }
            \bee;
        }\par
        \makeatother
    }
    \tableofcontents
    
    \chapter{\prefacename}
    \subfile{chapters/preface}
    
    \chapter{Unser Konzept}
    \subfile{chapters/concept}
    }
    
    
    \mainmatter

    \part{Grundrechenarten}
    \declchapter{grundrechnen}{Grundrechenarten}
    \ldsection{Einführung}
    \ldsection{Schriftlich addieren und subtrahieren}
    \ldsection{Schriftlich multiplizieren und dividieren}
    \ldsection{Terme auswerten}
    \ldsection{Das Kommutativgesetz}
    \ldsection{Das Assoziativgesetz}
    \ldsection{Das Distributivgesetz}
    \ldsection{Potenzen und Wurzeln}
    \ldsection{Runden und Überschlagsrechnung}

    \declchapter{teilbarkeit}{Teilbarkeit}
    \ldsection{Einführung}
    \ldsection{Teilbarkeitsregeln}
    \ldsection{Primzahlen}
    \ldsection{ggT und kgV}
    
    \declchapter{geominebene}{Geometrische Konstruktionen}
    \subfile{chapters/geominebene/00_diesunddas.tex}
    
    \declchapter{ganzzahl}{Ganze Zahlen}
    
    \declchapter{koords}{Koordinatensysteme}
    
    \declchapter{bruche}{Bruchrechnung}

    \declchapter{einheiten}{Rechnen mit Einheiten}
    
    \part{Gleichungen und Funktionen}
    \declchapter{vars}{Variablen}
    \ldsection{Einführung}
    \ldsection{Ausklammern}
    \ldsection{Ausmultiplizieren}
    \ldadvsection{Weiterführendes Wissen}

    \declchapter{aussagenlogik}{Aussagenlogik}
    \ldsection{Einführung}
    \ldsection{Aussagen in der Mathematik}
    \ldsection{Wahrheitstabellen}
    \ldsection{Äquivalenzumformungen}
    \ldsection{Logisches Schlussfolgern}
    \ldadvsection{Weiterführendes Wissen}

    \declchapter{mengen}{Mengen}
    \ldsection{Einführung}
    \ldsection{Mengendiagramme}
    \ldsection{Teilmengen}
    \ldsection{Zahlenmengen}
    \ldsection{Intervalle}
    \ldsection{Das kartesische Produkt}
    \ldadvsection{Weiterführendes Wissen}

    \declchapter{funktionen}{Funktionen}
    \ldsection{Einführung}
    \ldsection{Zuordnungsvorschriften beschreiben}
    \ldsection{Funktionsgraphen}
    \ldsection{Verkettung von Funktionen}
    \ldsection{Umkehrfunktion und Identität}
    \ldsection{Nullstellen und Achsenabschnitt}
    \ldsection{Symmetrien von Funktionsgraphen}
    \ldsection{Funktionsgraphen verschieben}
    \ldsection{Funktionsgraphen strecken und stauchen}
    \ldadvsection{Weiterführendes Wissen}
    
    \declchapter{lingleichungen}{Lineare Gleichungen}
    \ldsection{Einführung}
    \ldsection{Äquivalenzumformungen}
    \ldsection{Auflösen linearer Gleichungen}
    \ldsection{Auflösen linearer Ungleichungen}
    \ldsection{Geradengleichungen}
    \ldsection{Punkte auf Funktionsgeraden}
    \ldadvsection{Weiterführendes Wissen}
    
    \declchapter{lgs}{Lineare Gleichungssysteme}
    
    \declchapter{quadratische_gleichungen}{Quadratische Gleichungen}
    \ldsection{Einführung}
    \ldsection{Die pq-Formel}
    \ldsection{Parabeln}
    \ldsection{Linearfaktorzerlegung}

    \declchapter{potulog}{Potenz- und Logarithmenrechnung}

    \part{Geometrie}
    \declchapter{flaechen}{Flächen und Volumen}
    
    \declchapter{kreise}{Kreise}
    
    \declchapter{trig}{Trigonometrie}
    
    \declchapter{vektoren}{Vektorrechnung}
    \ldsection{Einführung}
    \ldsection{Das Skalarprodukt}
    \ldsection{Dreidimensionale Koordinatensysteme}
    \ldsection{Geraden im Raum}
    \ldsection{Ebenen im Raum}
    \ldsection{Abstände berechnen}

    \part{Analysis}
    \declchapter{grenzwerte}{Grenzwerte}
    
    \declchapter{ableitungen}{Ableitungen}
    \ldsection{Einführung}
    \ldsection{Sekantensteigung}
    \ldsection{Tangentensteigung}
    \ldsection{Ableitungsregeln}
    \ldsection{Die Potenzregel}
    \ldsection{Die Produktregel}
    \ldsection{Die Kettenregel}
    \ldsection{Ableitung von Exponentialfunktionen}
    \ldsection{Ableitung von Sinus und Cosinus}
    \ldsection{Trassierung}
    
    \declchapter{kurvendiskussion}{Kurvendiskussion}
    \ldsection{Einführung}
    \ldsection{Extrempunkte}
    \ldsection{Wendepunkte}
    \ldsection{Monotonieverhalten}

    \declchapter{wachstumsprozesse}{Wachstumsprozesse}
    \ldsection{Einführung}
    \ldsection{Lineares Wachstum}
    \ldsection{Exponentielles Wachstum}
    \ldsection{Beschränktes Wachstum}
    \ldsection{Logistisches Wachstum}

    \declchapter{integrale}{Integralrechnung}
    \ldsection{Einführung}
    \ldsection{Das Riemann-Integral}
    \ldsection{Der Hauptsatz der Differential- und Integralrechnung}
    \ldsection{Von Integralen zu Flächen}
    \ldsection{Uneigentliche Integrale}
    \ldsection{Rotationsvolumen}
    \ldadvsection{Weiterführendes Wissen}
    
    \part{Stochastik}
    \declchapter{statistik}{Beschreibende Statistik}
    
    \declchapter{kombinatorik}{Kombinatorik}
    \ldsection{Einführung}
    \ldsection{Permutationen}
    \ldsection{Urnenexperimente}
    \ldsection{Eulersche Phi Funktion}
    \ldadvsection{Weiterführendes Wissen}
    
    \declchapter{wkeitstheorie}{Wahrscheinlichkeitstheorie}

    \declchapter{zufallsvars}{Zufallsvariablen und Verteilungen}
    
    \backmatter
    \appendix
    \printindex
    
    \subfile{formulary}
    
\end{document}